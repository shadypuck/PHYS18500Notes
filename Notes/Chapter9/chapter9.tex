\documentclass[../notes.tex]{subfiles}

\pagestyle{main}
\renewcommand{\chaptermark}[1]{\markboth{\chaptername\ \thechapter\ (#1)}{}}
\setcounter{chapter}{8}

\begin{document}




\chapter{Rigid Body Motion}
\section{Introduction; Rotation About an Axis; Moments of Inertia}
\begin{itemize}
    \item \marginnote{11/3:}Announcements.
    \begin{itemize}
        \item We will now have \emph{seven} problem sets instead of \emph{eight}.
        \begin{itemize}
            \item Each problem set is now worth more (PSets still amount to 40\% of our grade).
            \item There will still be one makeup PSet at the end of the quarter.
        \end{itemize}
        \item PSet 5 is due next Friday.
    \end{itemize}
    \item Recap: Many-body motion.
    \begin{itemize}
        \item It's useful to introduce the center of mass coordinate, $\vec{R}=1/M\cdot\sum_\alpha m_\alpha\vec{r}_\alpha$, where $M=\sum_\alpha m_\alpha$.
        \item In the CM frame, $\vec{R}{\,}^*=0$ and $\vec{r}_\alpha=\vec{R}+\vec{r}_\alpha{}^*$.
        \begin{itemize}
            \item We also have $\vec{P}{\,}^*=0$, $T^*=\sum_\alpha m_\alpha(\dot{\vec{r}}_\alpha{}^*)^2/2$, and $\vec{J}{\,}^*=\sum_\alpha m_\alpha\vec{r}_\alpha{}^*\times\dot{\vec{r}}_\alpha{}^*$.
        \end{itemize}
        \item Then, going back into the lab frame, we have $\vec{P}=M\cdot\dot{\vec{R}}$, $T=M\dot{\vec{R}}^2/2+T^*$, and $\vec{J}=M\vec{R}\times\dot{\vec{R}}+\vec{J}{\,}^*$.
        \item One more note before we move onto rigid bodies: Suppose we're interested in the work, i.e., the rate of change of $T$ in the system.
        \begin{itemize}
            \item Recall that $m\ddot{\vec{r}}_\alpha=\sum_\beta\vec{F}_{\alpha\beta}+\vec{F}_\alpha$.
            \item Thus,
            \begin{equation*}
                \dot{T} = \sum_\alpha m_\alpha\dot{\vec{r}}_\alpha\cdot\ddot{\vec{r}}_\alpha
                = \sum_\alpha\sum_\beta\dot{\vec{r}}_\alpha\cdot\vec{F}_{\alpha\beta}+\sum_\alpha\dot{\vec{r}}_\alpha\cdot\vec{F}_\alpha
            \end{equation*}
            \item Note: Even letting $\vec{r}_{\alpha\beta}=\vec{r}_\alpha-\vec{r}_\beta$ and using $\vec{F}_{\alpha\beta}=-\vec{F}_{\beta\alpha}$, the left term above is often not equal to zero, i.e., there is no reason for it to vanish as in previous cases.
            \begin{itemize}
                \item This is not surprising, as it makes sense that the internal potential energy of the system would change in many cases.
            \end{itemize}
            \item However, if the $\vec{F}_{\alpha\beta}$ are conservative, then
            \begin{equation*}
                \dot{\vec{r}}_{\alpha\beta}\cdot\vec{F}_{\alpha\beta} = -\dv{t}V_{\text{int},\alpha\beta}
            \end{equation*}
            is the rate of internal forces doing work.
            \item Consequence: The rate of change of the kinetic plus internal potential energy is equal to the rate at which the external forces do work. That is,
            \begin{equation*}
                \dv{t}(T+V_\text{int}) = \sum_\alpha\dot{\vec{r}}_\alpha\cdot\vec{F}_\alpha
            \end{equation*}
            \item Additionally, we can find the rate of change of energy relative to the center of mass. In particular, in the CM frame, we have
            \begin{equation*}
                \dv{t}(\frac{1}{2}M\dot{\vec{R}}^2) = M\dot{\vec{R}}\cdot\ddot{\vec{R}}
                = \dot{\vec{R}}\cdot\sum_\alpha\vec{F}_\alpha
            \end{equation*}
            \item Subtracting the above equation from the one above it, we obtain
            \begin{align*}
                \dv{t}(T^*+V_\text{int}) &= \dv{t}(T-\frac{1}{2}M\dot{\vec{R}}^2+V_\text{int})\\
                &= \sum_\alpha\dot{\vec{r}}_\alpha\cdot\vec{F}_\alpha-\dot{\vec{R}}\cdot\sum_\alpha\vec{F}_\alpha\\
                &= \sum_\alpha\dot{\vec{r}}_\alpha{}^*\cdot\vec{F}_\alpha
            \end{align*}
            \item Note that in the leftmost term above, we are differentiating the total energy in the CM frame with respect to time. But since the time rate of change of energy is power, what we have expressed is the power.
        \end{itemize}
        \item Comparing this to $\dot{\vec{J}}{\,}^*=\sum_\alpha\vec{r}_\alpha{}^*\times\vec{F}_\alpha$, we see that we have a similar structure.
    \end{itemize}
    \item Today.
    \begin{itemize}
        \item Rigid bodies (a special case of many-body motion in which the particles are fixed relative to each other).
        \item Motion about an axis.
    \end{itemize}
    \item Today, we will primarily focus on rotation about an axis.
    \item The setup is as follows.
    \begin{itemize}
        \item We choose rotation to be in the $\hat{z}$ direction. We choose a shape (whatever we want), and it is rotating about this $\hat{z}$ axis.
        \item If is often useful to use cylindrical coordinates $(\rho,\phi,z)$ here because of the axial symmetry.
        \begin{itemize}
            \item Conversions: $x=\rho\cos\phi$, $y=\rho\sin\phi$, and $z=z$.
            \item Note that $\vec{r}=z\hat{z}+\rho\hat{\rho}$, much like in Figure \ref{fig:Vrotation}.
        \end{itemize}
        \item Recall that $\dv*{\vec{r}}{t}=\vec{\omega}\times\vec{r}=\dot{\vec{r}}$.
        \item We can now calculate our $\vec{J}$. It is equal to
        \begin{equation*}
            \vec{J} = \sum_\alpha m_\alpha\vec{r}_\alpha\times\dot{\vec{r}}_\alpha
            = \sum_\alpha m_\alpha\vec{r}_\alpha\times(\vec{\omega}\times\vec{r}_\alpha)
        \end{equation*}
        \item Expanding out the cross product, we obtain
        \begin{equation*}
            \begin{pmatrix}
                \hat{\rho} & \hat{\phi} & \hat{z}\\
                0 & 0 & \omega\\
                \rho & 0 & z\\
            \end{pmatrix}
            = \omega\rho\hat{\phi}
        \end{equation*}
        \item Expanding out our second cross product, we obtain
        \begin{equation*}
            \begin{pmatrix}
                \hat{\rho} & \hat{\phi} & \hat{z}\\
                \rho & 0 & z\\
                0 & \rho\omega & 0\\
            \end{pmatrix}
            = -z\rho\omega\hat{\rho}+\rho^2\omega\hat{z}
        \end{equation*}
    \end{itemize}
    \item Thus, we have that
    \begin{align*}
        \vec{J} &= \sum_\alpha m_\alpha(\rho_\alpha^2\omega\hat{z}-z_\alpha\omega\rho_\alpha\hat{\rho})\\
        &= \sum_\alpha m_\alpha[\rho_\alpha^2\omega\hat{z}-z_\alpha\omega(\rho_\alpha\cos\phi\hat{x}+\rho_\alpha\sin\phi\hat{y})]\\
        &= \omega\left( \sum_\alpha m_\alpha\rho_\alpha^2 \right)\hat{z}-\left( \omega\sum_\alpha m_\alpha z_\alpha x_\alpha \right)\hat{x}-\left( \omega\sum_\alpha m_\alpha z_\alpha y_\alpha \right)\hat{y}
    \end{align*}
    \begin{itemize}
        \item We can get this into a more familiar term via \textbf{moments of inertia}.
    \end{itemize}
    \item \textbf{Moment of inertia} (about the $z$-axis). \emph{Denoted by} $\bm{I_{zz}}$. \emph{Given by}
    \begin{equation*}
        I_{zz} = \sum_\alpha m_\alpha\rho_\alpha^2
        = \sum_\alpha m_\alpha(x_\alpha^2+y_\alpha^2)
    \end{equation*}
    \begin{itemize}
        \item In general, these are \textbf{second} moments about an axis. This just reflects the fact that the axial distance is \emph{squred}.
    \end{itemize}
    \item \textbf{Products of inertia}. Examples.
    \begin{itemize}
        \item $I_{xz}=-\sum_\alpha m_\alpha x_\alpha z_\alpha$.
        \item $I_{yz}=-\sum_\alpha m_\alpha y_\alpha z_\alpha$.
    \end{itemize}
    \item It follows from these definitions that, for $\vec{\omega}=\omega\hat{z}$, we have
    \begin{align*}
        J_z &= I_{zz}\omega&
        J_y &= I_{yz}\omega&
        J_x &= I_{xz}\omega
    \end{align*}
    \begin{itemize}
        \item Note that if $\vec{\omega}=\omega\hat{x}$, we have
    \end{itemize}
    \begin{align*}
        J_z &= I_{zx}\omega&
        J_y &= I_{yx}\omega&
        J_x &= I_{xx}\omega
    \end{align*}
    \item If we have $\vec{\omega}=\omega_x\hat{x}+\omega_y\hat{y}+\omega_z\hat{z}$, then the contributions to angular momentum add via
    \begin{equation*}
        \begin{bmatrix}
            J_x\\
            J_y\\
            J_z\\
        \end{bmatrix}
        = \underbrace{
            \begin{bmatrix}
                I_{xx} & I_{xy} & I_{xz}\\
                I_{yx} & I_{yy} & I_{yz}\\
                I_{zx} & I_{zy} & I_{zz}\\
            \end{bmatrix}
        }_I
        \begin{bmatrix}
            \omega_x\\
            \omega_y\\
            \omega_z\\
        \end{bmatrix}
    \end{equation*}
    \begin{itemize}
        \item $I$ is the \textbf{moment of inertia tensor}.
        \item It follows that, for example,
        \begin{equation*}
            J_x = I_{xx}\omega_x+I_{xy}\omega_y+I_{xz}\omega_z
        \end{equation*}
    \end{itemize}
    \item What's a tensor?
    \begin{itemize}
        \item It's like a matrix with a tiny bit more structure.
        \item For now, think of it as a $3\times 3$ matrix, and we'll talk more about it a little bit more next time.
    \end{itemize}
    \item Consider again $\vec{\omega}=\omega\hat{z}$.
    \begin{itemize}
        \item Then
        \begin{equation*}
            J_z = I_{zz}\omega
            = \sum_\alpha m_\alpha\rho_\alpha^2\omega
        \end{equation*}
        \item It follows that
        \begin{equation*}
            \dot{\vec{J}} = \sum_\alpha\vec{r}_\alpha\times\vec{F}_\alpha
        \end{equation*}
        \item Computing the cross product, we have
        \begin{equation*}
            \begin{pmatrix}
                \hat{\rho} & \hat{\phi} & \hat{z}\\
                \rho_\alpha & 0 & z_\alpha\\
                F_\rho & F_\phi & F_z\\
            \end{pmatrix}
            = -F_\phi z_\alpha\hat{\rho}+\rho_\alpha F_\phi\hat{z}
        \end{equation*}
        \item Then
        \begin{equation*}
            \dot{J}_z = I_{zz}\dot{\omega}
            = \sum_\alpha\rho_\alpha F_\phi
        \end{equation*}
        \begin{itemize}
            \item This is the equation of motion for rigid bodies.
            \item It gives $\omega(t)$ in terms of force $F_\phi$.
        \end{itemize}
    \end{itemize}
    \item Example: Equilibrium.
    \begin{figure}[h!]
        \centering
        \includegraphics[width=0.22\linewidth]{../ExtFiles/rectangularLamina.png}
        \caption{The rectangular lamina.}
        \label{fig:rectangularLamina}
    \end{figure}
    \begin{itemize}
        \item The \textbf{rectangular lamina}.
        \item We're pulling on two corners, and if it's in equilibrium, the thing is not rotating. This means that
        \begin{align*}
            bF-aMg &= 0\\
            F &= \frac{a}{b}Mg
        \end{align*}
    \end{itemize}
    \item Kinetic energy.
    \begin{itemize}
        \item We have that
        \begin{equation*}
            T = \sum_\alpha\frac{1}{2}m_\alpha(\rho_\alpha\omega)^2
            = \frac{1}{2}I\omega^2
        \end{equation*}
        \item It follows that the time rate of change of the kinetic energy is
        \begin{equation*}
            \dot{T} = I\omega\dot{\omega}
            = \sum_\alpha\omega\rho_\alpha F_\phi
            = \sum_\alpha(\rho\dot{\phi})F_\phi
            = \sum_\alpha\dot{\vec{r}}_\alpha\cdot\vec{F}_\alpha
        \end{equation*}
        \item Thus, in this case, the internal forces do no work (which in some sense makes sense for a rigid body).
        \item Thus, the KE is just related to these external forces as shown above.
    \end{itemize}
    \item We'll talk about pivot points next time.
\end{itemize}



\section{Euler's Angles; Freely Rotating Symmetric Body}
\begin{itemize}
    \item \marginnote{11/6:}Announcements.
    \begin{itemize}
        \item Our exams are graded; we can pick them up after class.
        \begin{itemize}
            \item High: 96\%.
            \item Median: 71\%.
        \end{itemize}
        \item Our course grades will be curved.
        \begin{itemize}
            \item $\text{A}^-$/$\text{B}^+$ cutoff is likely 83\%.
            \item $\text{B}^-$/$\text{C}^+$ cutoff is likely 60\%.
        \end{itemize}
        \item Office hours are back in her office today.
        \item Where we're going.
        \begin{itemize}
            \item Next week: Hamiltonians and conservation laws.
            \item Then Thanksgiving.
            \item Then a bit of dynamical systems.
        \end{itemize}
    \end{itemize}
    \item Recap.
    \begin{itemize}
        \item Rigid bodies --- rotation about a fixed axis.
        \item Moments and products of inertia.
        \begin{itemize}
            \item What is a tensor?
        \end{itemize}
    \end{itemize}
    \item Addressing a question from last time: Why do we call $T^*+V_\text{int}$ the "total energy" in the CM frame?
    \begin{itemize}
        \item It's tautological: This is the only possible definition of "total energy" in the CM frame.
        \item More specifically, recall that $\dv*{t}(T+V_\text{int})=\sum_\alpha\dot{\vec{r}}_\alpha\cdot\vec{F}_\alpha$ and $\dv*{t}(T^*+V_\text{int})=\sum_\alpha\dot{\vec{r}}_\alpha{}^*\cdot\vec{F}_\alpha$.
        \begin{itemize}
            \item If the $\vec{F}_\alpha$ are \emph{conservative}, then we can define $V_\text{ext}$ via
            \begin{equation*}
                -\dv{t}(V_\text{ext}(\{\vec{r}_\alpha\})) = -\sum_{\alpha,i}\pdv{V_\text{ext}}{r_{\alpha i}}\dv{r_{\alpha i}}{t}
                = -\sum_\alpha\dot{\vec{r}}_\alpha\cdot\vec{F}_\alpha
            \end{equation*}
            \item Plugging the above into the expression for $\dv*{t}(T+V_\text{int})$ given above yields
            \begin{equation*}
                \dv{t}(T+V_\text{int}+V_\text{ext}) = 0
            \end{equation*}
            \item But this is exactly the condition we expect for \emph{conservative} external forces.
        \end{itemize}
        \item Visualizing the system also helps make this definition of total energy more clear.
        \begin{itemize}
            \item Recall that the system is like a bunch of particles connected by springs, all of which are connected to some external potential like gravity.
            \item When we talk about the "total energy" in the CM frame, we're essentially just "diagonalizing" the system between external and internal forces.
        \end{itemize}
    \end{itemize}
    \item Back to rigid bodies now.
    \item Rigid body motion is completely specified by the following two equations of motion.
    \begin{enumerate}
        \item $\dot{\vec{P}}=M\ddot{\vec{R}}=\sum_\alpha\vec{F}_\alpha$.
        \begin{itemize}
            \item Looks like a particle of mass $M$ at the CM.
        \end{itemize}
        \item $\dot{\vec{J}}=\sum_\alpha\vec{r}_\alpha\times\vec{F}_\alpha$.
    \end{enumerate}
    \item Recap.
    \begin{itemize}
        \item Last time, we found that there's a huge simplification we can make because all the particles in a rigid body are locked together.
        \begin{itemize}
            \item The simplification is that $\vec{J}=\overleftrightarrow{I}\vec{\omega}$, where $\overleftrightarrow{I}$ is the moment of inertia tensor from last time.
            \item Jerison writes out the matrix formula all over again.
            \item Point to emphasize: $\overleftrightarrow{I}$ is an intrinsic property of the rigid body, and it plays the role of mass.
            \item If we have a continuous object, the sums over indices $\alpha$ turn into an integral! Recall this from prior courses.
            \item Compare to $\vec{P}=M\dot{\vec{R}}$ to see that there is a similar structure in the above equation.
        \end{itemize}
        \item Special case: Rotation about a fixed axis.
        \begin{itemize}
            \item We're headed toward the \textbf{compound pendulum}.
            \item For such a problem, we use cylindrical coordinates.
            \begin{itemize}
                \item Jerison redefines the coordinate conversions.
            \end{itemize}
            \item We take $\vec{\omega}$ to lie in the $\khat$ direction via $\vec{\omega}=\omega\khat$.
            \item The moment we're most concerned with is $I_{zz}$, defined as previously. Differentiating gets us to $J_z=I_{zz}\omega_z$ and $\dot{J}_z=I_{zz}\dot{\omega}$.
            \item From here, we can define the kinetic energy
            \begin{equation*}
                T = \sum_\alpha\frac{1}{2}m_\alpha\dot{\vec{r}}_\alpha{}^2
                = \sum_\alpha\frac{1}{2}m_\alpha(\rho_\alpha\omega)^2
                = \frac{1}{2}I_{zz}\omega^2
            \end{equation*}
            where we recall that $\dot{\vec{r}}_\alpha=\vec{\omega}\times\vec{r}_\alpha=\rho_\alpha\omega\,\hat{\phi}$.
        \end{itemize}
    \end{itemize}
    \item The EOMs for this system are given by $\dot{\vec{J}}=\sum_\alpha\vec{r}_\alpha\times\vec{F}_\alpha$.
    \begin{itemize}
        \item We're mostly interested in the $z$ component, i.e., $\dot{J}_z=\sum_\alpha\rho_\alpha F_\phi$.
        \item Sometimes, it can be useful to separate out the forces into axial forces and other forces via
        \begin{equation*}
            \dot{\vec{P}} = M\ddot{\vec{R}}
            = \vec{Q}+\sum_\alpha\vec{F}_\alpha
        \end{equation*}
        \item To make calculations, it will additionally be useful to have the following expression. For a rotating body, $\dot{\vec{R}}$ is given via $\dot{\vec{R}}=\vec{\omega}\times\vec{R}$ and $\ddot{\vec{R}}=\dot{\vec{\omega}}\times\vec{R}+\vec{\omega}\times\dot{\vec{R}}=\dot{\vec{\omega}}\times\vec{R}+\vec{\omega}\times(\vec{\omega}\times\vec{R})$.
    \end{itemize}
    \item This is true in general; if we specialize to our case of rotation about an axis\dots
    \begin{itemize}
        \item We first choose coordinates such that $z_\text{cm}=0$.
        \item Since this is rotation about an axis, the above equation simplifies to
        \begin{equation*}
            \ddot{\vec{R}} = R\dot{\omega}\,\hat{\phi}-\omega^2R\,\hat{\phi}
            = R\ddot{\phi}\,\hat{\phi}-\dot{\phi}^2R\,\hat{\rho}
        \end{equation*}
        \item In the right term above, the left term is tangential acceleration and the right term is centripetal acceleration.
    \end{itemize}
    \item Example: Compound pendulum.
    \begin{figure}[h!]
        \centering
        \begin{tikzpicture}
            \small
            \draw [stealth-stealth] (2,0) node[right]{$y$} -- (0,0) coordinate (o) -- (0,-2) coordinate (x) node[below]{$x$};
            \fill circle (1.5pt);
    
            \footnotesize
            \coordinate (c) at (-50:0.8);
            \draw [pux,thick,rotate around={-50:(c)}] (c) ellipse (1.5cm and 1cm);
            \draw [->] (0,0) -- node[right,yshift=1mm]{$\vec{R}$} (c);
            \draw [->] (c) -- node[right]{$\vec{F}$} ++(0,-1);
            \pic [draw,angle radius=4mm,angle eccentricity=1.4,pic text={$\phi$}] {angle=x--o--c};
        \end{tikzpicture}
        \caption{Compound pendulum.}
        \label{fig:compoundPendulum}
    \end{figure}
    \begin{itemize}
        \item We want to look at the force on the pivot.
        \item We define a new coordinate system as in Figure \ref{fig:compoundPendulum}. Explicitly, $\hat{x}$ points straight downwards and $\hat{y}$ points straight rightwards.
        \item We put our pendulum's center of mass such that it rotates through angle $\phi$.
        \item At this point, we have
        \begin{align*}
            T &= \frac{1}{2}I_{zz}\dot{\phi}^2&
            V &= M\vec{g}\cdot\vec{R} = -MgR\cos\phi
        \end{align*}
        \item Thus, our Lagrangian is
        \begin{equation*}
            L = T-V = \frac{1}{2}I_{zz}\dot{\phi}^2+MgR\cos\phi
        \end{equation*}
        \item It follows that our EOM is
        \begin{align*}
            I\ddot{\phi} &= -MgR\sin\phi\\
            \ddot{\phi} &= -\frac{MgR}{I}\sin\phi\\
            &= -\frac{g}{\ell}\sin\phi
        \end{align*}
        where $\ell=I/MR$.
        \begin{itemize}
            \item $\ell$ defines the \textbf{equivalent simple pendulum}.
        \end{itemize}
        \item From here, we can solve for the force on the pivot as a function of $\phi$ (we could also go through $\phi(t)$, and solve for $F(t)$ if we desired).
        \begin{itemize}
            \item We start with the conservation of energy
            \begin{equation*}
                \frac{1}{2}I\dot{\phi}^2-MgR\cos\phi = E
            \end{equation*}
            \item It follows that
            \begin{equation*}
                \dot{\phi}^2 = \frac{E+MgR\cos\phi}{I/2}
                = \frac{2E}{Mr\ell}+\frac{2g}{\ell}\cos\phi
            \end{equation*}
            \item We want to solve for $\vec{Q}$ from $M\ddot{\vec{R}}=\vec{Q}+\sum_\alpha\vec{F}_\alpha$.
            \item Here, the only relevant external force is our gravitational force $Mg\cos\phi\,\hat{\rho}-Mg\sin\phi\,\hat{\phi}$.
            \item We also found previously that $\ddot{\vec{R}}=R\ddot{\phi}\,\hat{\phi}-\dot{\phi}^2R\,\hat{\rho}$. Thus,
            \begin{equation*}
                MR\ddot{\phi}\,\hat{\phi}-MR\dot{\phi}^2\,\hat{\rho} = \vec{Q}+Mg\cos\phi\,\hat{\rho}-Mg\sin\phi\,\hat{\phi}
            \end{equation*}
            \item Splitting this vector equation into scalar equations, we obtain
            \begin{align*}
                Q_\rho &= -MR\dot{\phi}^2-Mg\cos\phi&
                Q_z &= 0&
                Q_\phi &= MR\ddot{\phi}+Mg\sin\phi
            \end{align*}
            \item Substituting from the conservation of energy, we obtain
            \begin{align*}
                Q_\rho &= -\frac{2E}{\ell}-Mg\left( 1+\frac{2R}{\ell} \right)\cos\phi&
                Q_z &= 0&
                Q_\phi &= Mg\left( 1-\frac{R}{\ell} \right)\sin\phi
            \end{align*}
            \item These are the final formulae for the forces on pivot as a function of $\phi$.
        \end{itemize}
    \end{itemize}
    \item \textbf{Equivalent simple pendulum}: The simple pendulum having the same equation of motion as our extended body.
    \item What happens in a similar system when we have a "sudden blow" or impulse?
    \begin{figure}[h!]
        \centering
        \begin{tikzpicture}
            \small
            \draw [stealth-stealth] (2,0) node[right]{$y$} -- (1.5,0) -- (1.5,-0.5) node[below]{$x$};
            \node at (1.7,-0.2) {$+$};
    
            \footnotesize
            \fill
                (0,0) circle (1.5pt)
                (0,-1.6) circle (1.5pt)
            ;
            \coordinate (c) at (0,-0.8);
            \draw [pux,thick,rotate around={-90:(c)}] (c) ellipse (1.5cm and 1cm);
            \draw [dashed] (0,0) -- node[right]{$d$} ($(0,0)!2!(c)$);
    
            \draw [->] (-2.3,-1.6) -- node[above]{$\vec{K}$} ++(0.9,0);
            \draw [->,shorten >=2pt] (0.7,0) -- node[above]{$\vec{S}$} (0,0);
        \end{tikzpicture}
        \caption{The "sweet spot" of a compound pendulum.}
        \label{fig:compoundPendulum2}
    \end{figure}
    \begin{itemize}
        \item Such pendulums have a sweet spot or equilibrium where the CM is just hanging down.
        \item We imagine that we kick the pendulum with impulse $\vec{K}$ in the $\hat{y}$ direction (using our modified coordinate system), as shown above.
        \item We have that $K\hat{y}=\vec{K}=\vec{F}\Delta t$.
        \item Let $\vec{S}=\vec{Q}\Delta t$.
        \item What we'll see is that there is a special value of $d$ (between the pivot and CM) for which $\vec{\rho}$ vanishes!
        \item During the short interval,
        \begin{equation*}
            I\ddot{\phi} = -MgR\sin\phi+Fd
        \end{equation*}
        \item We make the approximation that $\ddot{\phi}$ is constant during $\Delta t$ and that $\sin\phi=0$.
        \item It follows that
        \begin{equation*}
            \omega_\text{final} = \ddot{\phi}\Delta t
            = F\Delta t\frac{d}{I}
            = \frac{Kd}{I}
        \end{equation*}
        \item Additionally, we have that $\dot{\vec{P}}=\vec{Q}+\vec{F}$ so that
        \begin{equation*}
            P_\text{final} = \dot{P}\Delta t
            = -Q\Delta t+F\Delta t
            = -S+K
        \end{equation*}
        \item But we also know that
        \begin{equation*}
            P_\text{final} = M\dot{R}_\text{final}
            = M\omega_\text{final}R
            = \frac{MKdR}{I}
        \end{equation*}
        \item Thus, putting everything together, we obtain
        \begin{align*}
            \frac{MKdR}{I} &= -S+K\\
            S &= K\left( 1-\frac{MdR}{I} \right)
        \end{align*}
        \item Thus, $S$ vanishes if we choose $d=\ell=I/MR$.
        \item Takeaway: Regardless of the shape of our pendulum, if we hit it at the distance of the equivalent simple pendulum, we'll have no impulse on the pivot.
        \item This is the "sweet spot" of our baseball bat or whatever.
    \end{itemize}
\end{itemize}



\section{Office Hours (Jerison)}
\begin{itemize}
    \item \marginnote{11/6:}The final will slant toward the second half of the course, but everything is fair game.
    \item Is there an abstract environment in which we can view mass vs. angular mass and momentum vs. angular momentum, etc. as special cases of the same generalized construct?
    \begin{itemize}
        \item Yes.
        \item One answer.
        \begin{itemize}
            \item We can get this mapping from a speed-type thing to a momentum-type thing with linear operators.
            \item A tensor is a mathematical object with some kind of geometrical meaning independent of the coordinate basis.
        \end{itemize}
        \item Another answer.
        \begin{itemize}
            \item These are both examples of equations of motion that come from the Lagrangian (think \emph{generalized} mass, \emph{generalized} momentum, \emph{generalized} force, etc.).
        \end{itemize}
    \end{itemize}
    \item Could you post the KE of a free particle derivation?
    \item There will not be another \emph{in-class} review session, but she will hold one outside of class.
    \item We will get to Euler angles on Friday.
\end{itemize}



\section{Moment of Inertia Tensor; Principal Axis Rotation}
\begin{itemize}
    \item \marginnote{11/8:}Outline.
    \begin{itemize}
        \item Moment of inertia tensor.
        \begin{itemize}
            \item What is a tensor?
            \item Principal axes.
            \item Calculating moments of inertia.
        \end{itemize}
        \item Rotation about a principal axis.
        \begin{itemize}
            \item Precession.
        \end{itemize}
    \end{itemize}
    \item Next time.
    \begin{itemize}
        \item Stability of rotation about a principal axis.
        \item Euler angles.
        \item Lagrangian for rigid bodies.
    \end{itemize}
    \item Recall.
    \begin{itemize}
        \item Our EOMs are
        \begin{align*}
            \dot{\vec{P}} &= M\ddot{\vec{R}} = \sum_\alpha\vec{F}_\alpha&
            \dot{\vec{J}} &= \sum_\alpha\vec{r}_\alpha\times\vec{F}_\alpha
        \end{align*}
        \item Last time, we talked about rotation about a fixed axis.
        \item We've also seen that more generally, if $\vec{\omega}=\omega_x\ihat+\omega_y\jhat+\omega_z\khat$, then the angular momentum is given by
        \begin{equation*}
            \vec{J} = \overleftrightarrow{I}\vec{\omega}
        \end{equation*}
    \end{itemize}
    \item \textbf{Tensor}: A mathematical object that has geometric meaning independent of the coordinate basis.
    \item What is a tensor?
    \begin{itemize}
        \item She won't belabor the point because most of this machinery is orthogonal to our present aims.
        \item The "geometric meaning" alluded to in the definition has to be some kind of multilinear relationship, usually between vectors.
        \item In particular, $\overleftrightarrow{I}$ is an intrinsic property of the rigid body and its geometry.
        \begin{itemize}
            \item Its \emph{numerical} representation will change with the basis, though.
        \end{itemize}
        \item To calculate it, we need to be able to define it in a particular basis.
        \begin{itemize}
            \item The tensor comes prepackaged with (1) a definition in one basis and (2) a rule about how to change bases.
        \end{itemize}
        \item So, in our specific example, $\overleftrightarrow{I}$ is the linear operator that takes $\vec{\omega}$ and returns to you $\vec{J}$ for your rigid body.
        \item The rule to calculate entries of $\overleftrightarrow{I}$ is: Start with the $3\times 3$ matrix and then employ
        \begin{align*}
            I_{xx} &= \iiint\rho_m(\vec{r})(z^2+y^2)&
            I_{xy} &= -\iiint\rho_m(\vec{r})xy&
        \end{align*}
        and the like where herein, $\rho_m$ is the density mass/volume, not the radial coordinate.
        \item Change of basis rule: If you have a change of basis matrix $R$, then $\overleftrightarrow{I}$ in your new basis looks like $R^{-1}IR$.
        \item Note that $\overleftrightarrow{I}$ is called a $\binom{1}{1}$ tensor since it has 1 \textbf{contravariant} and 1 \textbf{covariant} dimension, meaning that it is like a regular matrix with 1 dimension that transforms as row vectors and 1 dimension that transforms as column vectors.
        \item Other examples of tensors.
        \begin{itemize}
            \item Scalars: Rank 0 tensors (same in any dimension).
            \item Vectors: Rank 1 tensors (can be row or column vectors).
            \item Metrics: There are $\binom{0}{2}$ tensors which do \emph{not} transform as matrices, even though they are arrays of numbers.
        \end{itemize}
        \item Note that since $I_{xy}=I_{yx}$, etc., $\overleftrightarrow{I}$ is \textbf{symmetric}.
        \begin{itemize}
            \item This implies that $\overleftrightarrow{I}$ has three real eigenvalues.
            \item Moreover, the eigenvectors of $\overleftrightarrow{I}$ are orthogonal.
            \item Thus, the eigenvectors of $\overleftrightarrow{I}$ are called the \textbf{principal axes} $\vec{e}_1,\vec{e}_2,\vec{e}_3$. Thus, in principle, we can find these for any object we choose, even though in any object we study, it will be obvious which axes are which.
            \item In the special basis of the principal axes, $\overleftrightarrow{I}$ is diagonal, i.e., $\overleftrightarrow{I}=\diag(I_{xx},I_{yy},I_{zz})$. It follows that
            \begin{equation*}
                \vec{J} = I_1\omega_1\vec{e}_1+I_2\omega_2\vec{e}_2+I_3\omega_3\vec{e}_3
            \end{equation*}
        \end{itemize}
        \item We don't need to worry about any of this stuff if we don't want to.
    \end{itemize}
    \item All these tensor machinations help with defining\dots
    \begin{itemize}
        \item The kinetic energy as:
        \begin{equation*}
            T = \sum_\alpha\frac{1}{2}m_\alpha\dot{\vec{r}}_\alpha{}^2
            = \sum_\alpha\frac{1}{2}m_\alpha(\vec{\omega}\times\vec{r}_\alpha)^2
            = \sum_\alpha\frac{1}{2}m_\alpha[\omega^2r_\alpha^2-(\vec{\omega}\cdot\vec{r}_\alpha)^2]
        \end{equation*}
        \item The angular momentum as:
        \begin{equation*}
            \vec{J} = \sum_\alpha m_\alpha\vec{r}_\alpha\times\dot{\vec{r}}_\alpha
            = \sum_\alpha m_\alpha\vec{r}_\alpha\times(\vec{\omega}\times\vec{r}_\alpha)
            = \sum_\alpha m_\alpha[r_\alpha^2\vec{\omega}-(\vec{r}_\alpha\cdot\vec{\omega})\vec{r}_\alpha]
        \end{equation*}
        \item Comparing the above two results, we obtain
        \begin{equation*}
            T = \frac{1}{2}\vec{\omega}\cdot\vec{J}
        \end{equation*}
        \item In particular, in the basis of principal axes,
        \begin{equation*}
            T = \frac{1}{2}I_1\omega_1^2+\frac{1}{2}I_2\omega_2^2+\frac{1}{2}I_3\omega_3^2
        \end{equation*}
        \item We can use the above to get the Lagrangian for general rigid body motion.
        \item A few notes on this.
        \begin{itemize}
            \item $\vec{e}_1,\vec{e}_2,\vec{e}_3$ rotate with the body.
            \item $\vec{J}=\overleftrightarrow{I}\vec{\omega}$ implies that in general, $\vec{J}$ is not parallel to $\vec{\omega}$. However, if $\vec{\omega}$ is along $\vec{e}_1,\vec{e}_2,\vec{e}_3$, then $\vec{J}$ is parallel to $\vec{\omega}$.
        \end{itemize}
    \end{itemize}
    \item \textbf{Symmetric body}: A rigid body for which two of the moments of inertia (usually taken to be $I_1,I_2$) are equal.
    \item \textbf{Totally symmetric body}: A rigid body for which all three of the moments of inertia are equal.
    \item Examples of (totally) symmetric bodies.
    \begin{itemize}
        \item A cylinder and square pyramid are both symmetric.
        \item A sphere and cube are both totally symmetric.
    \end{itemize}
    \item We'll mostly be dealing with \emph{symmetric} bodies.
    \item In this case:
    \begin{itemize}
        \item We have that
        \begin{equation*}
            \vec{J} = I_1(\omega_1\vec{e}_1+\omega_2\vec{e}_2)+I_3\omega_3\vec{e}_3
        \end{equation*}
        \item Thus, any axis in place of $\vec{e}_1,\vec{e}_2$ is a principal axis; we can choose any pair of orthogonal vectors herein.
    \end{itemize}
    \item In the case of a totally symmetric object, any axis ais a principal axis and $\vec{J}$ is always parallel to $\vec{\omega}$.
    \item Calculating $\overleftrightarrow{I}$.
    \begin{enumerate}
        \item If we take $\vec{r}=\vec{R}+\vec{r}{\,}^*$, then
        \begin{equation*}
            \sum_\alpha m_\alpha x^* = \sum_\alpha m_\alpha y^* = \sum_\alpha m_\alpha z^* = 0
        \end{equation*}
        \begin{itemize}
            \item Let $\vec{R}=(X,Y,Z)$.
            \item The above identities imply that the cross terms work out as follows.
            \begin{equation*}
                I_{xy} = \sum_\alpha m_\alpha(X+x^*)(Y+y^*)
                = -MXY-\sum_\alpha m_\alpha x_\alpha^*y_\alpha^*
            \end{equation*}
            \item Similarly, for the moments of inertia,
            \begin{equation*}
                I_{xx} = M(Y^2+Z^2)+I_{xx}^*
            \end{equation*}
            \begin{itemize}
                \item This decomposes the moment of inertia into the sum of the moment of the CM about your origin and the moment of inertia relative to $\vec{R}$.
                \item This is the \textbf{parallel axis theorem}.
            \end{itemize}
        \end{itemize}
        \item Objects with 3 perpendicular symmetry planes.
        \begin{itemize}
            \item Picture a cylinder or an ellipsoid with uniform density and three axes $a,b,c$.
            \item Then
            \begin{align*}
                I_1^* &= M(\lambda_yb^2+\lambda_zc^2)&
                I_2^* &= M(\lambda_xa^2+\lambda_zc^2)&
                I_3^* &= M(\lambda_xa^2+\lambda_yb^2)
            \end{align*}
            where\dots
            \begin{itemize}
                \item $\lambda_x=\lambda_y=\lambda_z=1/5$ for an ellipsoid;
                \item $\lambda_x=\lambda_y=\lambda_z=1/3$ for a parallelipiped;
                \item $\lambda_x=\lambda_y=1/4$ and $\lambda_z=1/3$ for a cylinder.
            \end{itemize}
            \item The derivation of the above results is on \textcite[209-11]{bib:KibbleBerkshire}.
            \begin{itemize}
                \item We should look through this as we may be expected to do the integrals!
            \end{itemize}
            \item What are the $\lambda$'s?
            \begin{itemize}
                \item It's just a number that has to do with the geometry of the subscripted axis.
            \end{itemize}
        \end{itemize}
    \end{enumerate}
    \item An interesting case: The effect of a small force on an axis; \textbf{precession}.
    \begin{itemize}
        \item Imagine an object that is spinning fairly rapidly about one of its axes.
        \item Assume that we have a symmetric body and that initially, $\vec{\omega}=\omega\vec{e}_3$.
        \item It follows that initially, $\vec{J}=I_3\omega_3\vec{e}_3$.
        \item In the case of no external forces, we have
        \begin{equation*}
            \dot{\vec{J}} = I_3\dot{\vec{\omega}}_3
            = \sum\vec{r}_\alpha\times\vec{F}_\alpha
            = 0
        \end{equation*}
        \item Now imagine we exert a small force $\vec{F}$ at a distance $\vec{r}$ up the axis from the CM/origin.
        \item It follows that $\dot{\vec{J}}=I_3\dot{\vec{\omega}}=\vec{r}\times\vec{F}$.
        \item Thus, $\dot{\vec{J}}$ is perpendicular to $\vec{\omega}$ and $\vec{\omega}$ changes direction, so the system turns.
        \item Under gravity, the wheel turns right.
        \item \emph{Mysterious picture}
    \end{itemize}
    \item At this point, we can analyze the motion of a top/gyroscope!
    \begin{figure}[h!]
        \centering
        \includegraphics[width=0.2\linewidth]{../ExtFiles/topGyroscope.png}
        \caption{A spinning top/gyroscope.}
        \label{fig:topGyroscope}
    \end{figure}
    \begin{itemize}
        \item We have that
        \begin{align*}
            I_3\dot{\vec{\omega}} &= R\vec{e}_3\times(-Mg\khat)\\
            I_3\omega\dot{\vec{e}}_3 &= MgR\khat\times\vec{e}_3\\
            \dot{\vec{e}}_3 &= \frac{MgR}{I_3\omega}\khat\times\vec{e}_3
        \end{align*}
        \item Defining $\vec{\Omega}=\frac{MgR}{I_3\omega}\khat$, we have that
        \begin{equation*}
            \dot{\vec{e}}_3 = \vec{\Omega}\times\vec{e}_3
        \end{equation*}
        \item Thus, $\vec{e}_3$ rotates about the $\khat$ axis (direction of $\vec{\Omega}$) at rate $\Omega$. This is precession!
        \item We make the approximation that the value for $\Omega\ll\omega$, or $I_3\omega^2/2\gg MgR$.
        \item We are making the approximation that $\vec{J}$ points in the $\vec{\omega}$ direction ($\vec{e}_3$ direction), which is not quite true due to the $\Omega$ contribution.
    \end{itemize}
\end{itemize}



\section{Euler's Angles; Freely Rotating Symmetric Body}
\begin{itemize}
    \item \marginnote{11/10:}Recap.
    \begin{itemize}
        \item Stability of rotation about a principal axis.
    \end{itemize}
    \item Today.
    \begin{itemize}
        \item Euler angles.
        \item Freely rotating body.
    \end{itemize}
    \item Recall.
    \begin{itemize}
        \item Last time, we talked about the moment of inertia tensor $\overleftrightarrow{I}$.
        \item Before you diagonalize it, this $3\times 3$ matrix has an element like $I_{xy}$ in each slot.
        \item Moreover, since it is a real symmetric matrix, the moment of inertia tensor is orthonormally diagonalizable.
        \begin{itemize}
            \item We call it's eigenvectors the principal axes.
        \end{itemize}
        \item In general, we will deal with nice symmetric objects like the cylinder, which you can just look at and see its principal axes.
        \begin{itemize}
            \item Moreover, in the particular case of the cylinder, \emph{symmetric} has the additional meaning that $I_1=I_2$.
            \item In this case, we can choose any two orthogonal vectors in the span of $\vec{e}_1,\vec{e}_2$ to be the principal axes.
        \end{itemize}
        \item Note that to find the principal axes rigorously, the rule is that the cross terms (i.e., those $I_{xy}$ in which the two subscripted variables differ and which thus do not lie along the diagonal of $\overleftrightarrow{I}$) equal zero.
        \begin{itemize}
            \item This occurs when integrating $m_\alpha xy$ over the whole object yields zero.
        \end{itemize}
        \item In the principal axes basis, $\overleftrightarrow{I}=\diag(I_1,I_2,I_3)$.
        \begin{itemize}
            \item Calculate $I_1,I_2,I_3$ either by choosing the principal axes from the beginning or by choosing nonstandard axes and diagonalizing.
        \end{itemize}
        \item Specific example: The rotating top.
        \begin{itemize}
            \item We often want to use the pivot point at the origin (which may well not be the CM of the system).
            \item To find the moment of inertia for bodies like this, we usually use the parallel axis theorem.
            \item Beware, though, that the principal axes at the CM and a pivot point need not be parallel. However, they are parallel (and thus can be taken to be identical) if the new origin is on a principal axis that passes through the COM.
        \end{itemize}
    \end{itemize}
    \item To start today, we generalize rotation.
    \begin{itemize}
        \item What if we can have any instantaneous angular velocity $\vec{\omega}$?
        \item The angular momentum in the basis of the principal axes will still be
        \begin{equation*}
            \vec{J} = I_1\omega_1\hat{e}_1+I_2\omega_2\hat{e}_2+I_3\omega_3\hat{e}_3
        \end{equation*}
        \begin{itemize}
            \item Recall that $\hat{e}_1,\hat{e}_2,\hat{e}_3$ rotate with the body.
        \end{itemize}
        \item To find our EOM, we start with our previously discovered EOMs.
        \begin{equation*}
            \left( \dv{\vec{J}}{t} \right)_\text{inertial} = \sum_\alpha\vec{r}_\alpha\times\vec{F}_\alpha
            = \vec{G}
            = \dot{\vec{J}}+\vec{\omega}\times\vec{J}
        \end{equation*}
        \begin{itemize}
            \item In particular, $\vec{G}$ is the net external torque and $\dot{\vec{J}}$ is the rate of change of the angular momentum within the rotating frame.
        \end{itemize}
        \item In this scenario, $\dot{\vec{J}}$ is easily found by differentiating the equation two lines above:
        \begin{equation*}
            \dot{\vec{J}} = I_1\dot{\omega}_1\hat{e}_1+I_2\dot{\omega}_2\hat{e}_2+I_3\dot{\omega}_3\hat{e}_3
        \end{equation*}
        \item It follows by combining the above two equations that the componentwise EOMs are
        \begin{align*}
            I_1\dot{\omega}_1+(I_3-I_2)\omega_2\omega_3 &= G_1\\
            I_2\dot{\omega}_2+(I_1-I_3)\omega_3\omega_1 &= G_2\\
            I_3\dot{\omega}_3+(I_2-I_1)\omega_1\omega_2 &= G_3
        \end{align*}
        \begin{itemize}
            \item We will discuss all of these next time.
        \end{itemize}
    \end{itemize}
    \item We now discuss a special case of the above motion.
    \item No external torques: The situation wherein $\vec{G}=0$.
    \begin{itemize}
        \item Suppose that we initially have some $\omega_3$ but that $\omega_1=\omega_2=0$.
        \begin{itemize}
            \item This is rotation about just one principal axis.
        \end{itemize}
        \item It follows that $\omega_1,\omega_2,\omega_3$ are constant and hence rotation continues about the same axis.
    \end{itemize}
    \item When is rotation about a principal axis stable?
    \begin{itemize}
        \item Suppose that $\vec{\omega}=\omega\hat{e}_3$, but this time, a small perturbtaion introduces angular momentum about one or more of the other axes.
        \begin{itemize}
            \item Mathematically, we assume $\omega_1,\omega_2\ll\omega_3$.
            \item Thus, we neglect terms that contain a product of $\omega_1$ and $\omega_2$.
        \end{itemize}
        \item Under these constraints, our EOMs become
        \begin{align*}
            I_1\dot{\omega}_1+(I_3-I_2)\omega_2\omega_3 &= 0\\
            I_2\dot{\omega}_2+(I_1-I_3)\omega_3\omega_1 &= 0\\
            I_3\dot{\omega}_3 &= 0
        \end{align*}
        \item The last line above implies that $\omega_3$ is constant.
        \item This leaves us with the task of solving the two remaining first-order, coupled ODEs.
        \item Try the ansatzs
        \begin{align*}
            \omega_1 &= a_1\e[pt]&
            \omega_2 &= a_2\e[pt]
        \end{align*}
        \item Then we get the following system of equations.
        \begin{equation*}
            \begin{cases}
                I_1pa_1\e[pt]+(I_3-I_2)a_2\e[pt]\omega_3 = 0\\
                I_2pa_2\e[pt]+(I_1-I_3)\omega_3a_1\e[pt] = 0
            \end{cases}
            \quad\Longrightarrow\quad
            \begin{cases}
                I_1pa_1+(I_3-I_2)a_2\omega_3 = 0\\
                I_2pa_2+(I_1-I_3)\omega_3a_1 = 0
            \end{cases}
        \end{equation*}
        \item We can solve this for two separate forms of the ratio $a_1/a_2$:
        \begin{align*}
            \frac{a_1}{a_2} &= \frac{-(I_3-I_2)\omega_3}{I_1p}&
            \frac{a_1}{a_2} &= \frac{I_2p}{-(I_1-I_3)\omega_3}
        \end{align*}
        \item It follows by transitivity that
        \begin{align*}
            \frac{I_2p}{-(I_1-I_3)\omega_3} &= \frac{-(I_3-I_2)\omega_3}{I_1p}\\
            I_1I_2p^2 &= \omega_3^2(I_3-I_2)(I_1-I_3)
        \end{align*}
        \item Thus, if $(I_3-I_2)(I_1-I_3)>0$, then $p>0$ and the rotation is unstable.
        \item On the other hand, if the term is less than zero, then $p$ is imaginary, so the rotation is purely oscillatory and hence stable.
        \item Takeaway:
        \begin{itemize}
            \item If $I_3$ is the smallest or largest of the moments, then the rotation is stable.
            \item If $I_3$ is the middle moment, the the rotation is unstable.
        \end{itemize}
    \end{itemize}
    \item Example of the above.
    \begin{itemize}
        \item Consider a rectangular prism with longest axis $a$, second longest $b$, and third longest.
        \item We can calculate that $\hat{e}_3\parallel c$, $\hat{e}_1\parallel a$, and $\hat{e}_2\parallel b$.
        \item Now calculate $I_1,I_2,I_3$.
        \begin{align*}
            I_3 &= M\left( \frac{a^2}{3}+\frac{b^2}{3} \right)&
            I_2 &= M\left( \frac{a^2}{3}+\frac{c^2}{3} \right)&
            I_1 &= M\left( \frac{b^2}{3}+\frac{c^2}{3} \right)
        \end{align*}
        \begin{itemize}
            \item It follows that $I_3$ is largest, $I_2$ is middle, and $I_1$ is smallest.
            \item Note that the $1/3$ comes from integrating $x^2$.
        \end{itemize}
        \item Thus, if the prism is rotating around the smallest axis to begin with, it will remain stably spinning around that axis.
        \item Rotating head over heels one is unstable.
        \item And the frisbee one (rotating around the largest axis) is also stable.
    \end{itemize}
    \item Euler angles.
    \begin{figure}[h!]
        \centering
        \includegraphics[width=0.47\linewidth]{../ExtFiles/EulerAngles.png}
        \caption{Euler angles.}
        \label{fig:EulerAngles}
    \end{figure}
    \begin{itemize}
        \item A method of specifying the orientation of an object in space that uses three angles.
        \item For rotation about the CM, these three angles will be our three DOFs for the system.
        \item Goal: Write $\vec{J},T$ in terms of these angles.
        \item Suppose our object starts such that it is oriented along $\ihat,\jhat,\khat$. We now want to go to an arbitrary new orientation. We do so in three steps.
        \begin{enumerate}
            \item Rotate it through an angle $\phi$ about $\khat$. Then
            \begin{equation*}
                \ihat,\jhat,\khat \mapsto \hat{e}_1'',\hat{e}_2',\khat
            \end{equation*}
            \item Rotate it through an angle $\theta$ about $\hat{e}_2'$. Then
            \begin{equation*}
                \hat{e}_1'',\hat{e}_2',\khat \mapsto \hat{e}_1',\hat{e}_2',\hat{e}_3
            \end{equation*}
            \item Finally, rotate it about an angle $\psi$ about $\hat{e}_3$. Then
            \begin{equation*}
                \hat{e}_1',\hat{e}_2',\hat{e}_3 \mapsto \hat{e}_1,\hat{e}_2,\hat{e}_3
            \end{equation*}
        \end{enumerate}
        \item It follows based on these definitions (see reasoning in \textcite{bib:KibbleBerkshire}) that
        \begin{equation*}
            \vec{\omega} = \dot{\phi}\khat+\dot{\theta}\hat{e}_2'+\dot{\psi}\hat{e}_3
        \end{equation*}
        \item But these bases are not ideal since these aren't our principal axis basis. Thus, we wish to define $\vec{\omega}$ in the principal axis basis.
        \item In the restrictive case of a symmetric body, $I_1=I_2$. Thus, we can choose $\hat{e}_1:=\hat{e}_1'$ and $\hat{e}_2:=\hat{e}_2'$ because we can choose \emph{any} vectors in this plane, as stated above.
        \item Additionally, we have that $\khat=-\sin\theta\,\hat{e}_1'+\cos\theta\,\hat{e}_3$.
        \item Thus,
        \begin{equation*}
            \vec{\omega} = \dot{\phi}(-\sin\theta\,\hat{e}_1'+\cos\theta\,\hat{e}_3)+\dot{\theta}\,\hat{e}_2'+\dot{\psi}\,\hat{e}_3
            = -\dot{\phi}\sin\theta\,\hat{e}_1'+\dot{\theta}\,\hat{e}_2'+(\dot{\psi}+\dot{\phi}\cos\theta)\,\hat{e}_3
        \end{equation*}
        \item Therefore,
        \begin{equation*}
            \vec{J} = -I_1\dot{\phi}\sin\theta\,\hat{e}_1'+I_1\dot{\theta}\,\hat{e}_2'+I_3(\dot{\psi}+\dot{\phi}\cos\theta)\,\hat{e}_3
        \end{equation*}
        and
        \begin{equation*}
            T = \frac{1}{2}I\vec{\omega}^2
            = \frac{1}{2}I_1\dot{\phi}^2\sin^2\theta+\frac{1}{2}I_1\dot{\theta}^2+\frac{1}{2}I_3(\dot{\psi}+\dot{\phi}\cos\theta)^2
        \end{equation*}
    \end{itemize}
\end{itemize}



\section{Free Rotation; Hamilton's Equations}
\begin{itemize}
    \item \marginnote{11/13:}Outline.
    \begin{itemize}
        \item Free rotation.
        \begin{itemize}
            \item Lagrangian + precession under gravity.
        \end{itemize}
        \item Hamiltonian.
    \end{itemize}
    \item Last time.
    \begin{itemize}
        \item We defined the Euler angles $\theta,\phi,\psi$ so that $\vec{\omega}=\dot{\phi}\khat+\dot{\theta}\hat{e}_2'+\dot{\psi}\hat{e}_3$.
        \item For a symmetric body, $I_1=I_2$. Thus, we had $\vec{\omega}=-\dot{\phi}\sin\theta\,\hat{e}_1'+\dot{\theta}\,\hat{e}_2'+(\dot{\psi}+\dot{\phi}\cos\theta)\,\hat{e}_3$
        \begin{itemize}
            \item $\hat{e}_1',\hat{e}_2',\hat{e}_3$ are the principal axes of the object.
        \end{itemize}
        \item With $\vec{\omega}$ in terms of our principal axes basis, it was easy to write down expressions for $\vec{J}$ and $T$.
    \end{itemize}
    \item We now investigate the motion of such a freely rotating system in a couple of cases.
    \item Case 1: No external forces.
    \begin{figure}[h!]
        \centering
        \includegraphics[width=0.43\linewidth]{../ExtFiles/FreeRot1.png}
        \caption{Free rotation under no external forces.}
        \label{fig:FreeRot1}
    \end{figure}
    \begin{itemize}
        \item In this case, $\vec{J}$ is conserved, so we have
        \begin{equation*}
            \vec{J} = J\khat=-J\sin\theta\,\hat{e}_1'+J\cos\theta\,\hat{e}_3
        \end{equation*}
        \item By comparing this with last class's equation defining $\vec{J}$ in terms of the Euler angles, we obtain the componentwise equations
        \begin{align*}
            I_1\dot{\phi}\sin\theta &= J\sin\theta\\
            I_1\dot{\theta} &= 0\\
            I_3(\dot{\psi}+\dot{\phi}\cos\theta) &= J\cos\theta
        \end{align*}
        \item The middle equation above implies that $\theta$ is constant, from which it follows that $J\sin\theta$ and $J\cos\theta$ are constant.
        \item Thus, we can solve for\dots
        \begin{align*}
            \dot{\phi} &= \frac{J}{I_1}&
            \dot{\psi} &= \frac{J\cos\theta}{I_3}-\frac{J}{I_1}\cos\theta
        \end{align*}
        where all of the terms on the right above are constant.
        \item It follows that in this case, $\hat{e}_3$ is fixed at angle $\theta$ with respect to $\vec{J}$.
        \item Moreover, $\vec{\omega}$ is at a fixed angle with respect to $\khat$, precessing around $\khat$ with rate $\dot{\phi}$.
        \item It follows that
        \begin{align*}
            \vec{\omega} &= -\dot{\phi}\sin\theta\hat{e}_1'+(\dot{\psi}+\dot{\phi}\cos\theta)\hat{e}_3\\
            &= \frac{J\sin\theta}{I_1}\hat{e}_1'+\frac{J\cos\theta}{I_3}\hat{e}_3\\
            &= \sin\beta\hat{e}_1+\cos\beta\hat{e}_3
        \end{align*}
        \item It follows that
        \begin{equation*}
            \tan\beta = \frac{I_3}{I_1}\tan\theta
        \end{equation*}
        \item The body cone "rolls around" the \textbf{space cone}; that is, we can check that $\dot{\psi}\sin\beta=\dot{\phi}\sin(\theta-\beta)$.
        \item The net motion is that the body is rotating on its \textbf{body cone} and also rotating about the axis.
    \end{itemize}
    \item Case 2: Gravity as an external force.
    \begin{itemize}
        \item In this case, it's easier to write down a Lagrangian.
        \item Luckily, we already have the kinetic energy, so
        \begin{equation*}
            L = \frac{1}{2}I_1\dot{\phi}^2\sin^2\theta+\frac{1}{2}I_1\dot{\theta}^2+\frac{1}{2}I_3(\dot{\psi}+\dot{\phi}\cos\theta)^2-MgR\cos\theta
        \end{equation*}
        \item Thus, our Euler-Lagrange equations will be
        \begin{align*}
            \dv{t}(I_1\dot{\theta}) &= I_1\dot{\phi}^2\sin\theta\cos\theta-I_3(\dot{\psi}+\dot{\phi}\cos\theta)\dot{\phi}\sin\theta\tag{$\theta$}\\
            \dv{t}[\underbrace{I_1\dot{\phi}\sin^2\theta+I_3(\dot{\psi}+\dot{\phi}\cos\theta)\cos\theta}_{P_\phi}] &= 0\tag{$\phi$}\\
            \dv{t}[\underbrace{I_3(\dot{\psi}+\dot{\phi}\cos\theta)}_{P_\psi}] &= 0\tag{$\psi$}
        \end{align*}
        \begin{itemize}
            \item Note that $P_\phi,P_\psi$ are generalized momenta.
        \end{itemize}
        \item It follows that
        \begin{equation*}
            \omega_3 = \dot{\psi}+\dot{\phi}\cos\theta
        \end{equation*}
        which is constant from Equation ($\psi$) above.
        \item What are the conditions for steady procession at fixed angle $\theta$?
        \begin{itemize}
            \item If $\theta$ is constant, $\dot{\psi},\dot{\phi}$ are constant.
            \item Let $\Omega:=\dot{\phi}$ be the precession rate.
            \item Then it follows by Equation ($\theta$) above that for $\dot{\theta}=0$, we must assume $\sin\theta\neq 0$.
            \item Substituting the definition of $\Omega$ into Equation ($\theta$), we have
            \begin{align*}
                0 &= I_1\Omega^2\cos\theta-I_3\omega_3\Omega+MgR\\
                \Omega &= \frac{I_3\omega_3\pm\sqrt{I_3^2\omega_3^2-4I_1\cos\theta MgR}}{2I_1\cos\theta}
            \end{align*}
            \item Thus, for real $\Omega$, we need $I_3^2\omega_3^2-4I_1\cos\theta MgR>0$.
            \item Thus, there is a minimum rotation speed $\omega_3$ to get steady precession for a given $\theta$ given by
            \begin{equation*}
                I_3^2\omega_3^2 = 4I_1\cos\theta MgR
            \end{equation*}
            \item Takeaway: The smaller the angle of inclination, the faster you have to be spinning to get steady procession at that rate.
            \item Next time, we'll analyze some even more general cases using the Hamiltonian.
        \end{itemize}
    \end{itemize}
    \item Problems with translation and rotation.
    \begin{itemize}
        \item Recall from our discussion of many-body systems that
        \begin{equation*}
            T = \frac{1}{2}M(\dot{X}^2+\dot{Y}^2+\dot{Z}^2)+T^*
        \end{equation*}
        where $\vec{R}=(X,Y,Z)$ is the center of mass and $T^*$ is the kinetic energy in the CM frame.
        \item For any general system, this is equal to
        \begin{equation*}
            T^* = \sum_\alpha\frac{1}{2}m_\alpha(\dot{\vec{r}}_\alpha{}^*)^2
        \end{equation*}
        \item Additionally, for a rigid body,
        \begin{equation*}
            T^* = \frac{1}{2}I_1^*\omega_1^2+\frac{1}{2}I_2^*\omega_2^2+\frac{1}{2}I_3^*\omega_3^2
        \end{equation*}
        \begin{itemize}
            \item Note that $I_1^*$ is the moment of inertia about principal axis 1 with CM at the origin.
            \item Explicitly,
            \begin{equation*}
                I_1^* = \iiint\rho_m(\vec{r}{\,}^*)(z^2+y^2)
            \end{equation*}
        \end{itemize}
    \end{itemize}
    \item We now leap to Chapter 12 to talk about the Hamiltonian!
\end{itemize}



\section{Chapter 8: Many-Body Systems}
\emph{From \textcite{bib:KibbleBerkshire}.}
\begin{itemize}
    \item \marginnote{11/3:}Wrapping up Section 8.4.
\end{itemize}



\section{Chapter 9: Rigid Bodies}
\emph{From \textcite{bib:KibbleBerkshire}.}
\begin{itemize}
    \item Covered a smattering of results from various sections.
\end{itemize}




\end{document}