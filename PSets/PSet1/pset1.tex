\documentclass[../psets.tex]{subfiles}

\pagestyle{main}
\renewcommand{\leftmark}{Problem Set \thesection}

\begin{document}




\section{Linear Motion}
\begin{enumerate}
    \item \marginnote{10/6:}One particle of mass $m$ is subject to force
    \begin{equation*}
        F =
        \begin{cases}
            -b & x>0\\
            b & x<0
        \end{cases}
    \end{equation*}
    A second particle is subject to force $F=-kx$.
    \begin{enumerate}
        \item Find the potential energy functions for each force. (1 pt)
        \begin{proof}
            First particle: Over $(0,\infty)$, we have $V=-\int_0^x-b\dd{x}=bx$. Similarly, over $(-\infty,0)$, we have $V=-\int_0^xb\dd{x}=-bx$. These two piecewise parts of the potential energy function can be unified in closed form as follows, where the domain is understood to be the given domain $\mathbb{R}\setminus\{0\}$.
            \begin{equation*}
                \boxed{V = b|x|}
            \end{equation*}
            Second particle:
            \begin{align*}
                V &= -\int_0^xF(x')\dd{x'}\\
                &= -\int_0^x-kx'\dd{x'}\\
                \Aboxed{V &= \frac{1}{2}kx^2}
            \end{align*}
        \end{proof}
        \item Find the trajectory $x(t)$ for each particle during the first period, assuming it is released at the origin at $t=0$ at velocity $v>0$. Describe the motion of each particle, and sketch each trajectory $x(t)$. Solve for the period and the points $x^*$ where each particle is stationary. (6 pts)
        \begin{proof}
            First particle:
            \begin{align*}
                m\ddot{x} &= -b\\
                \dv{\dot{x}}{t} &= -\frac{b}{m}\\
                \int_v^{\dot{x}}\dd{\dot{x}'} &= \int_0^t-\frac{b}{m}\dd{t}\\
                \dv{x}{t} &= -\frac{b}{m}t+v\\
                \int_0^{x(t)}\dd{x'} &= \int_0^t\left( -\frac{b}{m}t'+v \right)\dd{t}\\
                x(t) &= -\frac{b}{2m}t^2+vt
            \end{align*}
            It follows that at time
            \begin{align*}
                0 &= -\frac{b}{2m}t+v\\
                t &= \frac{2mv}{b}
            \end{align*}
            the first particle will return to the origin with velocity $-v$. Then by symmetry, over the domain $t\in(2mv/b,4mv/b)$, we will have
            \begin{equation*}
                x(t) = \frac{b}{2m}(t-2mv/b)^2-v(t-2mv/b)
            \end{equation*}
            Thus, the complete trajectory of the first particle during its first period under the stated assumptions is
            \begin{equation*}
                \boxed{
                    x_1(t) =
                    \begin{cases}
                        -\frac{b}{2m}t^2+vt & t\in[0,2mv/b]\\
                        \frac{b}{2m}(t-2mv/b)^2-v(t-2mv/b)  & t\in(2mv/b,4mv/b]
                    \end{cases}
                }
            \end{equation*}
            Second particle: From class, we know that the trajectory of the second particle during its first period under the stated assumptions is
            \begin{equation*}
                \boxed{x_2(t) = \frac{v}{\omega}\sin(\omega t)}
            \end{equation*}
            where $\omega=\sqrt{k/m}$.\par\medskip
            Both particles are perpetually falling toward the origin. Whenever they pass it, they start accelerating in the opposite direction. This motion occurs symmetrically on both sides of the origin, forever. Particle 1 falls as if drawn toward the origin by a constant gravitational field (that is, parabolically), and Particle 2 falls under a linear restoring force (that is, sinusoidally).

            \emph{trajectories sketch}

            As stated above, the period of the first particle is
            \begin{equation*}
                \boxed{\tau_1 = \frac{4mv}{b}}
            \end{equation*}
            From class, the period of the second particle is
            \begin{equation*}
                \boxed{\tau_2 = \frac{2\pi}{\omega}}
            \end{equation*}
            where $\omega$ is defined as above.\par\medskip
            The total energy of the system is wholly kinetic when the particle is at the origin. Thus, the total energy of each system is $mv^2/2$. Additionally, the particle is stationary under such monotonic concave potentials at the points where kinetic energy is converted entirely to potential. That is, for the first particle, where
            \begin{align*}
                \frac{1}{2}mv^2 &= b|x_1^*|\\
                \Aboxed{x_1^* &= \pm\frac{mv^2}{2b}}
            \end{align*}
            and for the second particle, where
            \begin{align*}
                \frac{1}{2}mv^2 &= \frac{1}{2}k(x_2^*)^2\\
                \Aboxed{x_2^* &= \pm v\sqrt{\frac{m}{k}}}
            \end{align*}
        \end{proof}
        \item Solve for $v$ such that the trajectories have the same period. Which particle travels further? Given this $v$, how many times do the two particles' trajectories cross during one period? (3 pts)
        \begin{proof}
            We want $v$ such that $\tau_1=\tau_2$. Plugging from part (B) and solving, we obtain
            \begin{align*}
                \tau_1 &= \tau_2\\
                \frac{4mv}{b} &= \frac{2\pi}{\omega}\\
                \Aboxed{v &= \frac{\pi b}{2m\omega}}
            \end{align*}
            Using this $v$, we can take the ratio
            \begin{align*}
                \frac{x_1^*}{x_2^*} &= \frac{mv^2/2b}{v\sqrt{m/k}}\\
                &= \frac{v\sqrt{mk}}{2b}\\
                &= \frac{\pi b\sqrt{mk}}{4bm\sqrt{k/m}}\\
                &= \frac{\pi}{4}
            \end{align*}
            Thus, since the ratio is less than one, \fbox{the second particle travels further.}\par\medskip
            Additionally, since there will always be a region near zero where the second particle is under a smaller magnitude of force than the first particle, the second particle will decelerate slower than the first one when $t$ is small. Thus, the second particle both travels further and gets farther away from the origin more quickly, implying that the first particle cannot catch up to it before both particles come to rest at their maximum distance from the origin. Therefore, the trajectories cross only \fbox{twice} during each period, specifically during their passes by the origin (at the beginning and middle of the period).
        \end{proof}
    \end{enumerate}
    \item The potential energy of a particle of mass $m$ is
    \begin{equation*}
        V(x) = E((\mu_1x+a)(\mu_2x-b))^2
    \end{equation*}
    where $E>0$ is a constant with units of energy, and $\mu_1,\mu_2,a,b>0$.
    \begin{enumerate}
        \item Sketch the potential energy function. Identify and label the locations of any minima. (3 pts)
        \begin{proof}
            ${\color{white}hi}$
            \begin{center}
                \begin{tikzpicture}
                    \footnotesize
                    \draw
                        (-1.5,0) -- (1.5,0) node[right]{$x$}
                        (0,-0.5) -- (0,1.5) node[above]{$V(x)$}
                    ;
                    
                    \draw [rex,thick] plot[domain=-1.376:0.625] (\x,{((\x+1)*(2*\x-0.5))^2});

                    \fill (-1,0) circle (2pt) node[below=2pt]{$-\frac{a}{\mu_1}$};
                    \fill (0.25,0) circle (2pt) node[below=2pt]{$\frac{b}{\mu_2}$};
                \end{tikzpicture}
            \end{center}
        \end{proof}
        \item Write expressions for the potential energy a distance $\delta x$ from each minimum, up to second order in $\delta x$. (2 pts)
        \begin{proof}
            Let's begin with the minimum at $-a/\mu_1$. The Taylor expansion about $x=-a/\mu_1$ to second order is
            \begin{equation*}
                \tilde{V}(\delta x) = V\left( -\frac{a}{\mu_1} \right)+V'\left( -\frac{a}{\mu_1} \right)\delta x+\frac{1}{2}V''\left( -\frac{a}{\mu_1} \right)(\delta x)^2
            \end{equation*}
            As in class, we can qualitatively inspect the graph from part (a) to learn that $V(-a/\mu_1)=V'(-a/\mu_1)=0$. Additionally, we can calculate that
            \begin{align*}
                V''\left( -\frac{a}{\mu_1} \right) &= \eval{\dv[2]{x}(E((\mu_1x+a)(\mu_2x-b))^2)}_{-\frac{a}{\mu_1}}\\
                % &= \eval{\dv{x}(\dv{x}(E((\mu_1x+a)(\mu_2x-b))^2))}_{-\frac{a}{\mu_1}}\\
                % &= \eval{\dv{x}(2E((\mu_1x+a)(\mu_2x-b)))\cdot\dv{x}((\mu_1x+a)(\mu_2x-b))}_{-\frac{a}{\mu_1}}\\
                % &= \eval{\dv{x}(2E((\mu_1x+a)(\mu_2x-b)))\cdot\dv{x}(\mu_1\mu_2x^2+(a\mu_2-b\mu_1)x-ab)}_{-\frac{a}{\mu_1}}\\
                % &= \eval{\dv{x}(2E((\mu_1x+a)(\mu_2x-b)))\cdot(2\mu_1\mu_2x+a\mu_2-b\mu_1)}_{-\frac{a}{\mu_1}}
                &= \eval{\dv[2]{x}(E(\mu_1\mu_2x^2+(a\mu_2-b\mu_1)x-ab)^2)}_{-\frac{a}{\mu_1}}\\
                &= \eval{\dv[2]{x}(E(\mu_1^2\mu_2^2x^4+2(a\mu_1\mu_2^2-b\mu_1^2\mu_2)x^3+((a\mu_2-b\mu_1)^2-2ab\mu_1\mu_2)x^2+\cdots))}_{-\frac{a}{\mu_1}}\\
                &= \eval{E(12\mu_1^2\mu_2^2x^2+12(a\mu_1\mu_2^2-b\mu_1^2\mu_2)x+2((a\mu_2-b\mu_1)^2-2ab\mu_1\mu_2))}_{-\frac{a}{\mu_1}}\\
                &= E(12a^2\mu_2^2-12(a^2\mu_2^2-ab\mu_1\mu_2)+2((a\mu_2-b\mu_1)^2-2ab\mu_1\mu_2))\\
                &= E(2a^2\mu_2^2+4ab\mu_1\mu_2+2b^2\mu_1^2)\\
                &= 2E(a\mu_2+b\mu_1)^2
            \end{align*}
            Therefore, the desired expression for the potential energy a distance $\delta x$ from the minimum at $x=-a/\mu_1$ up to second order in $\delta x$ is
            \begin{equation*}
                \boxed{\tilde{V}(\delta x) = E(a\mu_2+b\mu_1)^2(\delta x)^2}
            \end{equation*}
            In fact, because $V''(x)$ is a parabola with the same bilateral symmetry as $V(x)$, we have that $V''(-a/\mu_1)=V''(b/\mu_2)$. Therefore, the above expression is actually applicable the minimum at $x=b/\mu_2$ as well.
        \end{proof}
        \item For each minimum, what condition should $\delta x$ fulfill for this approximation to be valid? (i.e., $\delta x$ should be small compared to what length scale?) (3 pts)
        \begin{proof}
            Since the constraint derived for the validity of the SHM approximation in class relied only on the fact that we were expanding a Taylor series (i.e., did not rely on any characteristics of the Taylor series specific to the SHM), we can use the same constraint here. Explicitly, we want (with a change of variables)
            \begin{equation*}
                |\delta x| \ll \left| \frac{V''(-a/\mu_1)}{V'''(-a/\mu_1)} \right|
            \end{equation*}
            $V''(-a/\mu_1)$ was computed in part (B). Thus, $V'''(-a/\mu_1)$ can be computed by picking up with the expression for the second derivative \emph{before} evaluation in the work from part (B). Explicitly,
            \begin{align*}
                V'''\left( -\frac{a}{\mu_1} \right) &= \eval{\dv{x}(E(12\mu_1^2\mu_2^2x^2+12(a\mu_1\mu_2^2-b\mu_1^2\mu_2)x+2((a\mu_2-b\mu_1)^2-2ab\mu_1\mu_2)))}_{-\frac{a}{\mu_1}}\\
                &= \eval{E(24\mu_1^2\mu_2^2x+12(a\mu_1\mu_2^2-b\mu_1^2\mu_2))}_{-\frac{a}{\mu_1}}\\
                &= E(-24a\mu_1\mu_2^2+12(a\mu_1\mu_2^2-b\mu_1^2\mu_2))\\
                &= E(-12a\mu_1\mu_2^2-12b\mu_1^2\mu_2)\\
                &= -12\mu_1\mu_2E(a\mu_2+b\mu_1)
            \end{align*}
            Therefore, the desired condition is
            \begin{equation*}
                \boxed{|\delta x| \ll \frac{a\mu_2+b\mu_1}{6\mu_1\mu_2}}
            \end{equation*}
            Moreover, as in part (B), because $V'''(x)$ is an odd function about the line of reflection of $V(x)$, we have that $V'''(-a/\mu_1)=-V'''(b/\mu_2)$. Therefore, since we take an absolute value of the constraint into which we plug $V'''(b/\mu_2)$, the above expression is actually applicable to the minimum at $x=b/\mu_2$ as well.
        \end{proof}
        \item For each minimum, use your approximate potential energy function to specify the trajectory $x(t)$ of a particle of mass $m$ released from rest a distance $\delta x$ away from the minimum. (2 pts)
        \begin{proof}
            Since the approximate potential energy function is parabolic, the desired trajectory will be sinusoidal. Thus, to find said trajectory, first plug $\tilde{V}(\delta x)$ into
            \begin{equation*}
                -\dv{\tilde{V}}{(\tilde{\delta x})} = F = m(\ddot{\tilde{\delta x}})\footnotemark
            \end{equation*}
            \footnotetext{Note that in the above expression, $\tilde{\delta x}$ takes the place of the independent variable $\delta x$ used in parts (B)-(C) because the notation "$\delta x$" is now taken by a constant introduced in the problem statement for this part.}%
            Then extract a value for $k$, use the initial conditions to solve for $C$ and $D$, and plug into the general solution from class. Let's begin.\par
            As outlined above, start with
            \begin{align*}
                m(\ddot{\tilde{\delta x}}) &= -\dv{(\tilde{\delta x})}(E(a\mu_2+b\mu_1)^2(\tilde{\delta x})^2)\\
                &= -2E(a\mu_2+b\mu_1)^2\tilde{\delta x}\\
                m(\ddot{\tilde{\delta x}})+\underbrace{2E(a\mu_2+b\mu_1)^2}_k\tilde{\delta x} &= 0
            \end{align*}
            Thus, we have that $\omega=\sqrt{2E(a\mu_2+b\mu_1)^2/m}$, $C=x_0=\delta x$, and $D=v_0/\omega=0/\omega=0$. Therefore, we have that
            \begin{equation*}
                \tilde{\delta x}(t) = \delta x\cos(t\sqrt{\frac{2E(a\mu_2+b\mu_1)^2}{m}})
            \end{equation*}
            Finally, we can apply the coordinate transformations
            \begin{align*}
                x_{-a/\mu_1} &= \tilde{\delta x}-\frac{a}{\mu_1}\\
                x_{b/\mu_2} &= \tilde{\delta x}+\frac{b}{\mu_2}
            \end{align*}
            which can be inferred from the sketch in part (A). Given these, we can state the final trajectories for particle of mass $m$ released from rest a distance $\delta x$ from $x=-a/\mu_1$ and $x=b/\mu_2$, respectively, as
            \begin{align*}
                \Aboxed{x_{-a/\mu_1}(t) &= \delta x\cos(t\sqrt{\frac{2E(a\mu_2+b\mu_1)^2}{m}})-\frac{a}{\mu_1}}&
                \Aboxed{x_{b/\mu_2} &= \delta x\cos(t\sqrt{\frac{2E(a\mu_2+b\mu_1)^2}{m}})+\frac{b}{\mu_2}}
            \end{align*}
        \end{proof}
    \end{enumerate}
    \item \textcite{bib:KibbleBerkshire}, Q2.13. A particle falling under gravity is subject to a retarding force proportional to its velocity.
    \begin{enumerate}
        \item Find its position as a function of time, if it starts from rest. (7 pts)
        \begin{proof}
            We have that
            \begin{align*}
                % \begin{align*}
                %     \sum F &= m\ddot{x}\\
                %     F_d-F_g &= m\ddot{x}\\
                %     k\dot{x}-mg &= m\dv{\dot{x}}{t}\\
                %     \int_0^t\dd{t} &= \int_0^{\dot{x}}\frac{1}{k\dot{x}'/m-g}\dd{\dot{x}'}\\
                %     t &= \frac{m}{k}\ln(\frac{k\dot{x}}{m}-g)-\frac{m}{k}\ln(-g)\\
                %     \e[-kt/m] &= 1-\frac{k\dot{x}}{mg}\\
                %     \dot{x} &= \frac{mg}{k}\left( 1-\e[-kt/m] \right)
                % \end{align*}
                \sum F &= m\ddot{x}\\
                F_g-F_d &= m\ddot{x}\\
                mg-k\dot{x} &= m\dv{\dot{x}}{t}\\
                \int_0^t\dd{t} &= \int_0^{\dot{x}}\frac{1}{g-k\dot{x}'/m}\dd{\dot{x}'}\\
                t &= -\frac{m}{k}\ln(g-\frac{k\dot{x}}{m})+\frac{m}{k}\ln(g)\\
                \e[-kt/m] &= 1-\frac{k\dot{x}}{mg}\\
                \dot{x} &= \frac{mg}{k}\left( 1-\e[-kt/m] \right)
            \end{align*}
            where $k$ is the proportionality constant between the retarding force and the velocity. It follows that
            \begin{align*}
                \int_0^x\dd{x} &= \frac{mg}{k}-\frac{mg}{k}\int_0^t\e[-kt/m]\dd{t}\\
                x(t) &= \frac{mg}{k}-\frac{mg}{k}\left( -\frac{m}{k}\e[-kt/m]+\frac{m}{k} \right)\\
                \Aboxed{x(t) &= \frac{m^2g}{k^2}\e[-kt/m]-\frac{m^2g}{k^2}+\frac{mg}{k}}
            \end{align*}
        \end{proof}
        \item Show that it will eventually reach a terminal velocity, and solve for this velocity. (3 pts)
        \begin{proof}
            As $t\to\infty$, $\e[-kt/m]\to 0$, leaving
            \begin{equation*}
                \boxed{\dot{x}_f = \frac{mg}{k}}
            \end{equation*}
        \end{proof}
    \end{enumerate}
    \item Suppose we have an oscillator with negative damping described by
    \begin{equation*}
        m\ddot{x}+\lambda\dot{x}+kx = 0
    \end{equation*}
    where $\lambda<0$ and $k>0$.
    \begin{enumerate}
        \item Solve for $x(t)$ for the particle, if it begins at velocity $v$ at the origin. (4 pts)
        \begin{proof}
            % Use the solution equations from class for the appropriate case.

            % $\gamma=\lambda/2m<0$ and $\omega=\sqrt{k/m}>0$.

            % Three cases; positive zooming away, positive oscillation, and positive acceleration away.


            Let $-\gamma=\lambda/2m$ and $\omega=\sqrt{k/m}$ so that we may rewrite the equation as
            \begin{equation*}
                \ddot{x}-2\gamma\dot{x}+\omega_0^2x = 0
            \end{equation*}
            Use $x=\e[pt]$ as an ansatz to find that
            \begin{align*}
                0 &= p^2-2\gamma p+\omega_0^2\\
                p &= \gamma\pm\sqrt{\gamma^2-\omega_0^2}
            \end{align*}
            We now divide into three cases.\par
            Case 1 ($\gamma>\omega_0$): In this case, we have two real roots that are both positive real numbers by the form of $p$. Define
            \begin{equation*}
                \gamma_\pm = \gamma\pm\sqrt{\gamma^2-\omega_0^2}
            \end{equation*}
            Thus, we can write the general solution as
            \begin{equation*}
                x(t) = \frac{1}{2}A\e[\gamma_+t]+\frac{1}{2}B\e[\gamma_-t]
            \end{equation*}
            To apply the initial conditions, first take a derivative to get
            \begin{equation*}
                \dot{x}(t) = \frac{1}{2}A\gamma_+\e[\gamma_+t]+\frac{1}{2}B\gamma_-\e[\gamma_-t]
            \end{equation*}
            Now, solve the system of equations
            \begin{equation*}
                \begin{cases}
                    x(0) = \frac{1}{2}A\e[\gamma_+\cdot 0]+\frac{1}{2}B\e[\gamma_-\cdot 0]\\
                    \dot{x}(0) = \frac{1}{2}A\gamma_+\e[\gamma_+\cdot 0]+\frac{1}{2}B\gamma_-\e[\gamma_-\cdot 0]
                \end{cases}
                \quad\longrightarrow\quad
                \begin{cases}
                    0 = A+B\\
                    2v = A\gamma_++B\gamma_-
                \end{cases}
            \end{equation*}
            to get
            \begin{equation*}
                \boxed{x(t) = \frac{v}{\gamma_+-\gamma_-}(\e[\gamma_+t]-\e[\gamma_-t])}
            \end{equation*}
            Case 3 ($\gamma<\omega_0$): In this case, we'll have two complex roots. Define
            \begin{equation*}
                \omega = \sqrt{\omega_0-\gamma^2}
            \end{equation*}
            and write $p=\gamma\pm i\omega$. It follows that the general solution is
            \begin{align*}
                x(t) &= \frac{1}{2}A\e[\gamma t+i\omega t]+\frac{1}{2}B\e[\gamma t-i\omega t]\\
                &= a\e[\gamma t]\cos(\omega t-\theta)
            \end{align*}
            Adjusting for the initial conditions, we get
            \begin{equation*}
                \boxed{x(t) = \frac{v}{\omega}\e[\gamma t]\sin(\omega t)}
            \end{equation*}
            Case 3 ($\gamma=\omega_0$): In this case, we'll use an additional ansatz to get to the general solution
            \begin{equation*}
                x(t) = (a+bt)\e[\gamma t]
            \end{equation*}
            Solving in the initial conditions yields
            \begin{equation*}
                \boxed{x(t) = vt\e[\gamma t]}
            \end{equation*}
        \end{proof}
        \item Describe the behavior of the particle. Under what conditions does it oscillate? Sketch the possible trajectories. (4 pts)
        \begin{proof}
            Define $-\gamma=\lambda/2m$ and $\omega_0=\sqrt{k/m}$. Then it oscillates when $|\gamma|<|\omega_0|$, i.e., when
            \begin{equation*}
                \boxed{|\lambda| < 2\sqrt{km}}
            \end{equation*}
            If $\lambda$ gets too large negatively, the system can reach critical damping or overdamping, at which point the particle will just take off to $\infty$ and never again return to the origin.
        \end{proof}
        \item In which case does the particle gain energy the fastest for large times? Explain. (2 pts)
        \begin{proof}
            % Negative damping means the oscillation is growing!
            % Looking at $\dv*{E}{t}$ in the long run.
            % $-\lambda\dot{x}-kx=F=-\dv*{V}{t}$.
            % Gain energy? As in kinetic energy or potential energy? Or both? We should be able to calculate $T=\frac{1}{2}m\dot{x}^2$ over time given the trajectory in part (A) and then $V=-\int_0^xF(x')\dd{x'}$ and relate that as well.

            % Probably can just give a qualitative explanation that it's critical.

            % However, total energy is always kinetic at the origin, so we can just use $T$ at every periodic time that $x(t)=0$!


            In the \fbox{critical damping} case. Unlike in the other two cases, where some energy is lost due to imperfect synchronization of the oscillator and this pseudo-driving force, here, all of the energy that can be gained is gained. In particular, critical damping is faster than underdamping because $\gamma$ is just really small here ($<\omega_0$), and critical damping is faster than overdamping because $\gamma_-$ dominates in the long run in that case.
        \end{proof}
    \end{enumerate}
    \item \textcite{bib:KibbleBerkshire}, Q2.25. For an oscillator under periodic force $F(t)=F_1\cos(\omega_1t)$\dots
    \begin{enumerate}
        \item Calculate the \textbf{power} (defined as the rate at which the force does work). (4 pts)
        \begin{proof}
            % We need the Fourier integral.

            % $P=F\dot{x}$.
            % $P=F_1\cos(\omega_1 t)\dot{x}$.

            % $\int F\dd{x}$

            % Take velocity from the particular solution only, neglecting the transient term.

            From lecture, we know a particular solution of the driven, damped harmonic oscillator. It follows from the definition of power that we have
            \begin{align*}
                P &= F\dot{x}\\
                &= F_1\cos(\omega_1t)\dv{t}(a_1\cos(\omega_1t-\theta_1))\\
                \Aboxed{P &= -a_1\omega_1F_1\cos(\omega_1t)\sin(\omega_1t-\theta_1)}
            \end{align*}
        \end{proof}
        \item Show that the \textbf{average power} (defined as the time average over a complete cycle) is $P=m\omega_1^2a_1^2/\gamma$, and hence verify that it is equal to the average rate at which energy is dissipated against the resistive force. (3 pts)
        \begin{proof}
            Let $\tau$ be the period of the oscillator. Then the average power $\bar{P}$ of the oscillator is given by
            \begin{equation*}
                \bar{P} = \frac{1}{\tau}\int_0^\tau P\dd{t}
            \end{equation*}
            Plugging in and solving, we can get to the following.
            \begin{align*}
                \bar{P} &= \frac{1}{\tau}\int_0^\tau F_1\cos(\omega_1t)\cdot -a_1\omega_1\sin(\omega_1t-\theta_1)\dd{t}\\
                &= -\frac{a_1\omega_1F_1}{\tau}\int_0^\tau\cos(\omega_1t)\sin(\omega_1t-\theta_1)\dd{t}\\
                &= -\frac{a_1\omega_1F_1}{\tau}\int_0^\tau\cos(\omega_1t)(\sin(\omega_1t)\cos\theta_1-\cos(\omega_1t)\sin\theta_1)\dd{t}\\
                &= -\frac{a_1\omega_1F_1}{\tau}\int_0^\tau\cos(\omega_1t)\sin(\omega_1t)\cos\theta_1\dd{t}+\frac{a_1\omega_1F_1}{\tau}\int_0^\tau\cos(\omega_1t)\cos(\omega_1t)\sin\theta_1\dd{t}\\
                &= -a_1\omega_1F_1\cos\theta_1\cdot\frac{1}{\tau}\int_0^\tau\cos(\omega_1t)\sin(\omega_1t)\dd{t}+a_1\omega_1F_1\sin\theta_1\cdot\frac{1}{\tau}\int_0^\tau\cos^2(\omega_1t)\dd{t}
            \end{align*}
            At this point, we invoke the laws that
            \begin{align*}
                \int_0^\tau\cos(\omega_1t)\sin(\omega_1t)\dd{t} &= 0&
                \frac{1}{\tau}\int_0^\tau\cos^2(\omega_1t)\dd{t} &= \frac{1}{2}
            \end{align*}
            This simplifies the above expression to
            \begin{equation*}
                \bar{P} = \frac{a_1\omega_1F_1\sin\theta_1}{2}
            \end{equation*}
            But we're not quite done. Recalling that
            \begin{align*}
                \tan\theta_1 &= \frac{2\gamma\omega_1}{\omega_0^2-\omega_1^2}&
                \sin(\tan^{-1}(x)) &= \frac{x}{\sqrt{x^2+1}}
            \end{align*}
            we can learn that
            \begin{equation*}
                \sin\theta_1 = \frac{\frac{2\gamma\omega_1}{\omega_0^2-\omega_1^2}}{\sqrt{\left( \frac{2\gamma\omega_1}{\omega_0^2-\omega_1^2} \right)^2+1}}
                = \frac{2ma_1\omega_1}{F_1\gamma}
            \end{equation*}
            Therefore, we have that
            \begin{align*}
                \bar{P} &= \frac{a_1\omega_1F_1\sin\theta_1}{2}\\
                &= \frac{a_1\omega_1F_1}{2}\cdot\frac{2ma_1\omega_1}{F_1\gamma}\\
                \bar{P} &= \frac{ma_1^2\omega_1^2}{\gamma}
            \end{align*}
            as desired.
        \end{proof}
        \item Show that the power $P$ --- as a function of $\omega_1$ --- is at a maximum at $\omega_1=\omega_0$. Also find the values of $\omega_1$ for which it has half its maximum value. (3 pts)
        % \begin{proof}
        %     Fourier integrals?
        % \end{proof}
    \end{enumerate}
    \item \textcite{bib:KibbleBerkshire}, Q2.32. Find the Green's function of an oscillator in the case $\gamma>\omega_0$. Use it to solve the problem of an oscillator that is initially in equilibrium, and is subjected from $t=0$ to a force increasing linearly with time via $F=ct$.
    % \begin{proof}
        
    % \end{proof}
    \item How long did you spend on this problem set?
    \begin{proof}
        About 10 hours.
    \end{proof}
\end{enumerate}




\end{document}