\documentclass[../psets.tex]{subfiles}

\pagestyle{main}
\renewcommand{\leftmark}{Problem Set \thesection}

\begin{document}




\section{Linear Motion}
\begin{enumerate}
    \item \marginnote{10/6:}One particle of mass $m$ is subject to force
    \begin{equation*}
        F =
        \begin{cases}
            -b & x>0\\
            b & x<0
        \end{cases}
    \end{equation*}
    A second particle is subject to force $F=-kx$.
    \begin{enumerate}
        \item Find the potential energy functions for each force. (1 pt)
        \item Find the trajectory $x(t)$ for each particle during the first period, assuming it is released at the origin at $t=0$ at velocity $v>0$. Describe the motion of each particle, and sketch each trajectory $x(t)$. Solve for the period and the points $x^*$ where each particle is stationary. (6 pts)
        \item Solve for $v$ such that the trajectories have the same period. Which particle travels further? Given this $v$, how many times do the two particles' trajectories cross during one period? (3 pts)
    \end{enumerate}
    \item The potential energy of a particle of mass $m$ is
    \begin{equation*}
        V(x) = E((\mu_1x+a)(\mu_2x-b))^2
    \end{equation*}
    where $E>0$ is a constant with units of energy, and $\mu_1,\mu_2,a,b>0$.
    \begin{enumerate}
        \item Sketch the potential energy function. Identify and label the locations of any minima. (3 pts)
        \item Write expressions for the potential energy a distance $\delta x$ from each minimum, up to second order in $\delta x$. (2 pts)
        \item For each minimum, what condition should $\delta x$ fulfill for this approximation to be valid? (i.e., $\delta x$ should be small compared to what length scale?) (3 pts)
        \item For each minimum, use your approximate potential energy function to specify the trajectory $x(t)$ of a particle of mass $m$ released from rest a distance $\delta x$ away from the minimum. (2 pts)
    \end{enumerate}
    \item \textcite{bib:KibbleBerkshire}, Q2.13. A particle falling under gravity is subject to a retarding force proportional to its velocity.
    \begin{enumerate}
        \item Find its position as a function of time, if it starts from rest. (7 pts)
        \item Show that it will eventually reach a terminal velocity, and solve for this velocity. (3 pts)
    \end{enumerate}
    \item Suppose we have an oscillator with negative damping described by
    \begin{equation*}
        m\ddot{x}+\lambda\dot{x}+kx = 0
    \end{equation*}
    where $\lambda<0$ and $k>0$.
    \begin{enumerate}
        \item Solve for $x(t)$ for the particle, if it begins at velocity $v$ at the origin. (4 pts)
        \item Describe the behavior of the particle. Under what conditions does it oscillate? Sketch the possible trajectories. (4 pts)
        \item In which case does the particle gain energy the fastest for large times? Explain. (2 pts)
    \end{enumerate}
    \item \textcite{bib:KibbleBerkshire}, Q2.25. For an oscillator under periodic force $F(t)=F_1\cos(\omega_1t)$\dots
    \begin{enumerate}
        \item Calculate the \textbf{power} (defined as the rate at which the force does work). (4 pts)
        \item Show that the \textbf{average power} (defined as the time average over a complete cycle) is $P=m\omega_1^2a_1^2/\gamma$, and hence verify that it is equal to the average rate at which energy is dissipated against the resistive force. (3 pts)
        \item Show that the power $P$ --- as a function of $\omega_1$ --- is at a maximum at $\omega_1=\omega_0$. Also find the values of $\omega_1$ for which it has half its maximum value. (3 pts)
    \end{enumerate}
    \item \textcite{bib:KibbleBerkshire}, Q2.32. Find the Green's function of an oscillator in the case $\gamma>\omega_0$. Use it to solve the problem of an oscillator that is initially in equilibrium, and is subjected from $t=0$ to a force increasing linearly with time via $F=ct$.
    \item How long did you spend on this problem set?
\end{enumerate}




\end{document}