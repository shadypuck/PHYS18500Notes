\documentclass[../notes.tex]{subfiles}

\pagestyle{main}
\renewcommand{\chaptermark}[1]{\markboth{\chaptername\ \thechapter\ (#1)}{}}
\setcounter{chapter}{3}

\begin{document}




\chapter{Central Conservative Forces}
\section{Conservation Laws, Radial Energy Equation, Orbits}
\begin{itemize}
    \item \marginnote{10/16:}Review.
    \begin{itemize}
        \item The Lagrangian for a free particle.
        \item We have that space is isotropic and homogeneous, and time is homogeneous.
        \item $L(v^2)$ or $L(v)$ implies that the equations of motion are invariant under the velocity boost.
        \item Recall that $v=\sqrt{v^2}=\sqrt{v_x^2+v_y^2+v_z^2}$.
        \item From here, we get to $L=\frac{1}{2}mv^2$
    \end{itemize}
    \item What we've said on 3D central conservative forces thus far.
    \begin{itemize}
        \item Consider a particle in 3D at position $\vec{r}$ being acted on by external forces $\vec{F}(\vec{r})$.
        \item In spherical coordinates, we have
        \begin{align*}
            x &= r\sin\theta\cos\phi&
            y &= r\sin\theta\sin\phi&
            z &= r\cos\theta
        \end{align*}
        \begin{itemize}
            \item $\theta$ is the \textbf{polar} angle.
            \item $\phi$ is the \textbf{azimuthal} angle.
        \end{itemize}
        \item Special case: \emph{Central} force.
        \begin{itemize}
            \item \emph{Central} force: Acts in a direction parallel to $\vec{r}$.
            \item Thus, if $\vec{F}$ is central, then $\vec{G}=\vec{r}\times\vec{F}=0$. It follows that $\vec{J}=\vec{r}\times\vec{p}$ is conserved.
        \end{itemize}
        \item Special case: \emph{Conservative} force.
        \begin{itemize}
            \item Condition: $\vec{\nabla}\times\vec{F}=0$.
            \item In this case, there exists a scalar function $V$ such that $\vec{F}=-\vec{\nabla}V$.
            \item Equivalently, in spherical coordinates,
            \begin{align*}
                F_r &= -\pdv{V}{r}&
                F_\theta &= -\frac{1}{r}\pdv{V}{\theta}&
                F_\phi &= -\frac{1}{r\sin\theta}\pdv{V}{\phi}
            \end{align*}
            \item Thus, since $F_\theta=F_\phi=0$, it follows that $V=V(r)$ is not dependent on $\theta$ or $\phi$. Mathematically,
            \begin{equation*}
                \vec{F} = -\pdv{F}{r}\hat{r}
            \end{equation*}
        \end{itemize}
    \end{itemize}
    \item Recall: Uniform circular motion.
    \begin{itemize}
        \item In plane polar coordinates, we have
        \begin{equation*}
            \vec{F} = m\ddot{\vec{r}} = m[(\ddot{r}-r\dot{\theta}^2)\hat{r}+(r\ddot{\theta}+2\dot{r}\dot{\theta})\hat{\theta}]
        \end{equation*}
        \item In uniform circular motion, $\dot{\theta}=\omega$ and $r=R$, so we get
        \begin{equation*}
            \vec{F} = mR\omega^2\hat{r}
            = \frac{mv^2}{R}\hat{r}
        \end{equation*}
        \begin{itemize}
            \item Note that to get from the second expression above to the third one, we substitute the definition of angular velocity: $\omega=v/R$.
        \end{itemize}
    \end{itemize}
    \item We are now ready to treat the case of the \emph{central conservative} force.
    \begin{itemize}
        \item Herein, we get a lot of conservation laws!
        \begin{enumerate}
            \item Energy is conserved:
            \begin{equation*}
                \frac{1}{2}m\dot{\vec{r}}{\,}^2+V(r) = E = \text{constant}
            \end{equation*}
            \begin{itemize}
                \item Note that this is a scalar equation.
            \end{itemize}
            \item Angular momentum is conserved:
            \begin{equation*}
                m\vec{r}\times\dot{\vec{r}} = \vec{J} = \text{constant}
            \end{equation*}
            \begin{itemize}
                \item Note that this is a set of 3 vector equations.
            \end{itemize}
        \end{enumerate}
        \item Letting $r,\theta$ be our plane polar coordinates, we can rewrite equation (1) above as follows.
        \begin{equation*}
            \frac{1}{2}m(\dot{r}^2+r^2\dot{\theta}^2)+V(r) = E
        \end{equation*}
        \item Similarly, we can rewrite equation (2) above as follows.
        \begin{align*}
            \vec{J} &= mr\hat{r}\times(\underbrace{\dot{r}}_{v_r}\hat{r}+\underbrace{r\dot{\theta}}_{v_\theta}\hat{\theta})\\
            J &= mr^2\dot{\theta}
        \end{align*}
        \begin{itemize}
            \item Note that $J$ is a scalar here.
        \end{itemize}
        \item Since $\dot{\theta}$ is a function of $r$, we get orbits??
        \item In particular, if we plug $\dot{\theta}=J/mr^2$ into the original conservation of energy equation, we get the \textbf{radial energy equation}.
    \end{itemize}
    \item \textbf{Radial energy equation}: The equation defined as follows. \emph{Given by}
    \begin{equation*}
        \frac{1}{2}m\dot{r}^2+\frac{J^2}{2mr^2}+V(r) = E
    \end{equation*}
    \begin{itemize}
        \item Note that this looks a lot like the original energy conservation law once we define the \textbf{effective potential energy}.
    \end{itemize}
    \item \textbf{Effective potential energy}: The following expression, which treats a radial particle as if it were a one-dimensional particle, i.e., in a rotating reference frame. \emph{Denoted by} $\bm{U(r)}$. \emph{Given by}
    \begin{equation*}
        U(r) = \frac{J^2}{2mr^2}+V(r)
    \end{equation*}
    \item Example: $V(r)=kr^2/2$.
    \begin{itemize}
        \item Then $U(r)=J^2/2mr^2+kr^2/2$. We get a shape that is a blend of a parabola but that goes up super steeply as we approach the axis.
        \item We have a PE function that looks like a parabola, but gets steeper close to the origin; this gives us two turn about points.
    \end{itemize}
    \item Most important example: The inverse square law.
    \begin{figure}[h!]
        \centering
        \begin{subfigure}[b]{0.3\linewidth}
            \centering
            \begin{tikzpicture}
                \small
                \draw
                    (0,-1) -- (0,3) node[above]{$U(r)$}
                    (0,0) -- (3,0) node[right]{$r$}
                ;
    
                \footnotesize
                \draw (0.924,0.1) -- ++(0,-0.2) node[below]{$r_1$};
                \draw [densely dashed] (0,2) node[left]{$E$} -- ++(3,0);
    
                \draw [blx,thick] plot[domain=0.69:3,smooth] (\x,{0.6/(\x)^2+1.2/\x});
            \end{tikzpicture}
            \caption{$k>0$.}
            \label{fig:invSqPota}
        \end{subfigure}
        \begin{subfigure}[b]{0.3\linewidth}
            \centering
            \begin{tikzpicture}
                \small
                \draw
                    (0,-1) -- (0,3) node[above]{$U(r)$}
                    (0,0) -- (3,0) node[right]{$r$}
                ;
    
                \footnotesize
                \draw (0.5,0.1) -- ++(0,-0.2) node[below,xshift=-1pt]{$\frac{\ell}{2}$};
                \draw (1,0.1) -- ++(0,-0.2) node[below]{$\ell$};
                \draw (0.1,-0.6) -- ++(-0.2,0) node[left]{$-\frac{|k|}{2\ell}$};
    
                \draw [blx,thick] plot[domain=0.29:3,samples=100,smooth] (\x,{0.6/(\x)^2-1.2/\x});
            \end{tikzpicture}
            \caption{$k<0$.}
            \label{fig:invSqPotb}
        \end{subfigure}
        \caption{Potentials under the inverse square law.}
        \label{fig:invSqPot}
    \end{figure}
    \begin{itemize}
        \item Attractive and repulsive case.
        \item Occurs when $\vec{F}=k\hat{r}/r^2$.
        \item $k>0$ is repulsive (think like charges).
        \item $k<0$ is attractive (think gravity or opposite charges).
        \item Repulsive case ($k>0$):
        \begin{itemize}
            \item We have
            \begin{equation*}
                U(r) = \frac{J^2}{2mr^2}+\frac{k}{r}
            \end{equation*}
            \item Thus, we get a point of closest approach as dictated by the energy $E$, but that's it.
        \end{itemize}
        \item Attractive case:
        \begin{itemize}
            \item We have
            \begin{equation*}
                U(r) = \frac{J^2}{2mr^2}+\frac{k}{r}
            \end{equation*}
            once again.
            \item If we define the \textbf{length scale}, then we obtain
            \begin{equation*}
                U(r) = |k|\left( \frac{\ell}{2r^2}-\frac{1}{r} \right)
            \end{equation*}
            \item It follows that, as in Figure \ref{fig:invSqPotb}, the effective potential crosses $y=0$ at $\ell/2$ and has minimum at $y=-|k|/2\ell$.
            \item Additionally, there are four possible types of trajectories depending on the value of $E$.
            \begin{enumerate}
                \item ($E=U_\text{min}=-|k|/2\ell$): $\vec{r}=0$, and we get uniform circular motion with $r=\vec{l}$. The kinetic energy is
                \begin{equation*}
                    \frac{1}{2}mv^2 = T
                    = E-V
                    = -\frac{|k|}{2\ell}-\frac{k}{\ell}
                    = \frac{|k|}{2\ell}
                \end{equation*}
                so that the speed is
                \begin{equation*}
                    v = \sqrt{\frac{|k|}{m\ell}}
                \end{equation*}
                \item ($-|k|/2\ell<E<0$): Bounded orbit between $r_1<r<r_2$. The shape is an \emph{ellipse}, as we will later prove.
                \item ($E=0$): The orbit is a parabola: It comes in, slingshots around, and just escapes back to $\infty$.
                \item ($E>0$): The orbit is a hyperbola.
            \end{enumerate}
        \end{itemize}
    \end{itemize}
    \item \textbf{Length scale}: The distance from the origin at which the particle orbits stably. \emph{Denoted by} $\bm{\ell}$. \emph{Given by}
    \begin{equation*}
        \ell = J^2/m|k|
    \end{equation*}
    \item We find the orbits by eliminating time from the radial energy equation.
    \begin{itemize}
        \item Recall that
        \begin{equation*}
            \frac{1}{2}m\dot{r}^2+\frac{J^2}{2mr^2}+V(r) = E
        \end{equation*}
        \item Now substitute in $u=1/r$ and its consequence $\dv*{u}{\theta}=(-1/r^2)\dv*{r}{\theta}$. Note, of course, that we are just encoding all of the information in $r$ in this "$u$."
        \item It follows that
        \begin{equation*}
            \dot{r} = \dv{r}{\theta}\dot{\theta}
            = -r^2\dot{\theta}\dv{u}{\theta}
            = -\frac{J}{m}\dv{u}{\theta}
        \end{equation*}
        \item Returning the substitution into the radial energy equation, we obtain
        \begin{equation*}
            \frac{J^2}{2m}\left( \dv{u}{\theta} \right)^2+\frac{J^2}{2m}u^2+V(r) = E
        \end{equation*}
        \item Evidently, this equation relates $u$ to $\theta$ for a given potential energy function $V$!
        \item We can use this equation to solve for the $V(u)$ that gives us an orbit $u(\theta)$, and (even easier) we can solve for the orbit given $V(u)$. Depending on how complicated this is, we may not be able to solve the ODE. But we \emph{can} solve it in several cool cases.
    \end{itemize}
    \item We'll start next time with orbits of the inverse square law.
\end{itemize}




\end{document}