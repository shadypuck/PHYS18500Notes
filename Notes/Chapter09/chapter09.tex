\documentclass[../notes.tex]{subfiles}

\pagestyle{main}
\renewcommand{\chaptermark}[1]{\markboth{\chaptername\ \thechapter\ (#1)}{}}
\setcounter{chapter}{8}

\begin{document}




\chapter{Rigid Body Motion}
\section{Introduction; Rotation About an Axis; Moments of Inertia}
\begin{itemize}
    \item \marginnote{11/3:}Today.
    \begin{itemize}
        \item Rigid bodies (special case of many-body motion in which particles are fixed relative to each other).
        \item Motion about an axis.
    \end{itemize}
    \item Today, we will primarily focus on rotation about an axis.
    \item The setup is as follows.
    \begin{itemize}
        \item We choose rotation to be in the $\hat{z}$ direction. This means that we choose a shape (whatever we want) and let it rotate about this $\hat{z}$ axis.
        \item If is often useful to use cylindrical coordinates $(\rho,\phi,z)$ here because of the axial symmetry.
        \begin{itemize}
            \item Conversions: $x=\rho\cos\phi$, $y=\rho\sin\phi$, and $z=z$.
        \end{itemize}
        \item Note that $\vec{r}=z\hat{z}+\rho\hat{\rho}$ (much like in Figure \ref{fig:Vrotation}) and recall that $\dv*{\vec{r}}{t}=\vec{\omega}\times\vec{r}=\dot{\vec{r}}$.
        \item We can now calculate our $\vec{J}$. It is equal to
        \begin{equation*}
            \vec{J} = \sum_\alpha m_\alpha\vec{r}_\alpha\times\dot{\vec{r}}_\alpha
            = \sum_\alpha m_\alpha\vec{r}_\alpha\times(\vec{\omega}\times\vec{r}_\alpha)
        \end{equation*}
        \item Expanding out the cross product in parentheses, we obtain
        \begin{equation*}
            \begin{pmatrix}
                \hat{\rho} & \hat{\phi} & \hat{z}\\
                0 & 0 & \omega\\
                \rho & 0 & z\\
            \end{pmatrix}
            = \omega\rho\,\hat{\phi}
        \end{equation*}
        \item Expanding out our second cross product, we obtain
        \begin{equation*}
            \begin{pmatrix}
                \hat{\rho} & \hat{\phi} & \hat{z}\\
                \rho & 0 & z\\
                0 & \rho\omega & 0\\
            \end{pmatrix}
            = -z\rho\omega\,\hat{\rho}+\rho^2\omega\,\hat{z}
        \end{equation*}
        \item Thus, we have that
        \begin{align*}
            \vec{J} &= \sum_\alpha m_\alpha(\rho_\alpha^2\omega\hat{z}-z_\alpha\omega\rho_\alpha\hat{\rho})\\
            &= \sum_\alpha m_\alpha[\rho_\alpha^2\omega\hat{z}-z_\alpha\omega(\rho_\alpha\cos\phi\,\hat{x}+\rho_\alpha\sin\phi\,\hat{y})]\\
            &= \omega\left( \sum_\alpha m_\alpha\rho_\alpha^2 \right)\hat{z}+\omega\left( -\sum_\alpha m_\alpha z_\alpha x_\alpha \right)\hat{x}+\omega\left( -\sum_\alpha m_\alpha z_\alpha y_\alpha \right)\hat{y}
        \end{align*}
        \item We can get this into a more familiar form via \textbf{moments of inertia}.
    \end{itemize}
    \pagebreak
    \item \textbf{Moment of inertia} (about the $z$-axis). \emph{Denoted by} $\bm{I_{zz}}$. \emph{Given by}
    \begin{equation*}
        I_{zz} = \sum_\alpha m_\alpha\rho_\alpha^2
        = \sum_\alpha m_\alpha(x_\alpha^2+y_\alpha^2)
    \end{equation*}
    \begin{itemize}
        \item In general, these are \textbf{second} moments about an axis. This name just reflects the fact that the axial distance $\rho_\alpha$ is \emph{squared}.
    \end{itemize}
    \item \textbf{Products of inertia}. \emph{Examples}.
    \begin{align*}
        I_{xz} &= -\sum_\alpha m_\alpha x_\alpha z_\alpha&
        I_{yz} &= -\sum_\alpha m_\alpha y_\alpha z_\alpha
    \end{align*}
    \item It follows from these definitions and the above expression for $\vec{J}$ that, for $\vec{\omega}=\omega\hat{z}$, we have
    \begin{align*}
        J_z &= I_{zz}\omega&
        J_y &= I_{yz}\omega&
        J_x &= I_{xz}\omega
    \end{align*}
    \begin{itemize}
        \item Note that if $\vec{\omega}=\omega\,\hat{x}$, we have
    \end{itemize}
    \begin{align*}
        J_z &= I_{zx}\omega&
        J_y &= I_{yx}\omega&
        J_x &= I_{xx}\omega
    \end{align*}
    \item If we have $\vec{\omega}=\omega_x\,\hat{x}+\omega_y\,\hat{y}+\omega_z\,\hat{z}$, then the contributions to angular momentum add as a linear combination via
    \begin{equation*}
        \begin{bmatrix}
            J_x\\
            J_y\\
            J_z\\
        \end{bmatrix}
        = \underbrace{
            \begin{bmatrix}
                I_{xx} & I_{xy} & I_{xz}\\
                I_{yx} & I_{yy} & I_{yz}\\
                I_{zx} & I_{zy} & I_{zz}\\
            \end{bmatrix}
        }_{\overleftrightarrow{I}}
        \begin{bmatrix}
            \omega_x\\
            \omega_y\\
            \omega_z\\
        \end{bmatrix}
    \end{equation*}
    \begin{itemize}
        \item $\overleftrightarrow{I}$ is the \textbf{moment of inertia tensor}.
        \item It follows that, for example,
        \begin{equation*}
            J_x = I_{xx}\omega_x+I_{xy}\omega_y+I_{xz}\omega_z
        \end{equation*}
        \item Since $I_{xy}=I_{yx}$, for example, $\overleftrightarrow{I}$ is a symmetric matrix.
    \end{itemize}
    \item What's a tensor?
    \begin{itemize}
        \item It's like a matrix with a tiny bit more structure.
        \item For now, think of it as a $3\times 3$ matrix, and we'll talk more about it a little bit more next time.
    \end{itemize}
    \item Consider again $\vec{\omega}=\omega\,\hat{z}$.
    \begin{itemize}
        \item We know that
        \begin{equation*}
            J_z = I_{zz}\omega
            = \sum_\alpha m_\alpha\rho_\alpha^2\omega
        \end{equation*}
        \item Additionally, recall that
        \begin{equation*}
            \dot{\vec{J}} = \sum_\alpha\vec{r}_\alpha\times\vec{F}_\alpha
        \end{equation*}
        \item Computing one of the cross products in the above sum, we have
        \begin{equation*}
            \begin{pmatrix}
                \hat{\rho} & \hat{\phi} & \hat{z}\\
                \rho_\alpha & 0 & z_\alpha\\
                F_\rho & F_\phi & F_z\\
            \end{pmatrix}
            = -F_\phi z_\alpha\,\hat{\rho}+\rho_\alpha F_\phi\,\hat{z}\footnotemark
        \end{equation*}
        \footnotetext{Why is there not a $\hat{\phi}$ term in this cross product??}
        \item Then we have a second expression for $\dot{J}_z$, in addition to the one obtained by taking derivatives of both sides of the expression $J_z=I_{zz}\omega$:
        \begin{equation*}
            \dot{J}_z = I_{zz}\dot{\omega}
            = \sum_\alpha\rho_\alpha F_\phi
            % = \sum_\alpha\rho_\alpha(m_\alpha a_\phi)
            % = \sum_\alpha\rho_\alpha(m_\alpha\dot{v}_\phi)
            % = \sum_\alpha\rho_\alpha(m_\alpha\rho_\alpha\dot{\omega})
            % = \sum_\alpha m_\alpha\rho_\alpha^2\dot{\omega}
        \end{equation*}
        \begin{itemize}
            \item This equation determines the rate of change of angular velocity, and hence may be called the equation of motion of the rotating body.
            \item It gives $\omega(t)$ in terms of force $F_\phi$.
        \end{itemize}
    \end{itemize}
    \item Example of using $\dot{J}_z=\sum_\alpha\rho_\alpha F_\phi$ analogously to Newton's second law.
    \begin{figure}[H]
        \centering
        \includegraphics[width=0.22\linewidth]{../ExtFiles/rectangularLamina.png}
        \caption{The rectangular lamina.}
        \label{fig:rectangularLamina}
    \end{figure}
    \begin{itemize}
        \item Let the shape depicted in Figure \ref{fig:rectangularLamina} be in equilibrium, i.e., $\dot{\omega}=0$ so $\dot{J}_z=0$.
        \item This shape is called the \textbf{rectangular lamina}. It is of size $a\times b$, of negligible mass, pivoted at one quarter, carrying a weight $Mg$ at one corner, and supported by a horizontal force $\vec{F}$.
        \item We're pulling on two corners, and if it's in equilibrium, the thing is not rotating.
        \item This means that the force $F$ with which we have to pull on the top-left corner in order for the shape to stay in equilibrium is
        \begin{align*}
            \sum_\alpha\rho_\alpha F_\phi &= 0\\
            bF-aMg &= 0\\
            F &= \frac{a}{b}Mg
        \end{align*}
    \end{itemize}
    \item Kinetic energy.
    \begin{itemize}
        \item We have that
        \begin{equation*}
            T = \sum_\alpha\frac{1}{2}m_\alpha(\rho_\alpha\omega)^2
            = \frac{1}{2}I_{zz}\omega^2
        \end{equation*}
        \item It follows that the time rate of change of the kinetic energy is
        \begin{equation*}
            \dot{T} = I_{zz}\omega\dot{\omega}
            = \sum_\alpha\omega\rho_\alpha F_\phi
            = \sum_\alpha(\rho\dot{\phi})F_\phi
            = \sum_\alpha\dot{\vec{r}}_\alpha\cdot\vec{F}_\alpha
        \end{equation*}
        \item Since the internal forces do not appear in the above expression, we have proven that they cannot do any work on the rigid body (which also makes intuitive sense for a rigid body).
        \item Indeed, the change in the kinetic energy is only related to the external forces, as shown above.
    \end{itemize}
    \item We'll talk about pivot points next time.
\end{itemize}



\section{Center of Mass Acceleration; Compound Pendulum}
\begin{itemize}
    \item \marginnote{11/6:}Announcements.
    \begin{itemize}
        \item Our exams are graded; we can pick them up after class.
        \begin{itemize}
            \item High: 96\%.
            \item Median: 71\%.
        \end{itemize}
        \item Our course grades will be curved.
        \begin{itemize}
            \item $\text{A}^-$/$\text{B}^+$ cutoff is likely 83\%.
            \item $\text{B}^-$/$\text{C}^+$ cutoff is likely 60\%.
        \end{itemize}
        \item Office hours are back in her office today.
        \item Where we're going.
        \begin{itemize}
            \item Next week: Hamiltonians and conservation laws.
            \item Then Thanksgiving.
            \item Then a bit of dynamical systems.
        \end{itemize}
    \end{itemize}
    \item Recap.
    \begin{itemize}
        \item Rigid bodies --- rotation about a fixed axis.
        \item Moments and products of inertia.
        \begin{itemize}
            \item What is a tensor?
        \end{itemize}
    \end{itemize}
    \item Addressing a question from last time: Why do we call $T^*+V_\text{int}$ the "total energy" in the CM frame?
    \begin{itemize}
        \item It's tautological: This is the only possible definition of "total energy" in the CM frame.
        \item More specifically, recall that $\dv*{t}(T+V_\text{int})=\sum_\alpha\dot{\vec{r}}_\alpha\cdot\vec{F}_\alpha$ and $\dv*{t}(T^*+V_\text{int})=\sum_\alpha\dot{\vec{r}}_\alpha{}^*\cdot\vec{F}_\alpha$.
        \begin{itemize}
            \item If the $\vec{F}_\alpha$ are \emph{conservative}, then we can define $V_\text{ext}$ via
            \begin{equation*}
                -\dv{t}(V_\text{ext}(\{\vec{r}_\alpha\})) = -\sum_{\alpha,i}\pdv{V_\text{ext}}{r_{\alpha i}}\dv{r_{\alpha i}}{t}
                = -\sum_\alpha\dot{\vec{r}}_\alpha\cdot\vec{F}_\alpha
            \end{equation*}
            \item Plugging the above into the expression for $\dv*{t}(T+V_\text{int})$ given above yields
            \begin{equation*}
                \dv{t}(T+V_\text{int}+V_\text{ext}) = 0
            \end{equation*}
            \item But this is exactly the condition we expect for \emph{conservative} external forces.
        \end{itemize}
        \item Visualizing the system also helps make this definition of total energy more clear.
        \begin{itemize}
            \item Recall that the system is like a bunch of particles connected by springs, all of which are connected to some external potential like gravity.
            \item When we talk about the "total energy" in the CM frame, we're essentially just "diagonalizing" the system between external and internal forces.
        \end{itemize}
    \end{itemize}
    \item Back to rigid bodies now.
    \item Rigid body motion is completely specified by the following two equations of motion.
    \begin{enumerate}
        \item $\dot{\vec{P}}=M\ddot{\vec{R}}=\sum_\alpha\vec{F}_\alpha$.
        \begin{itemize}
            \item Looks like a particle of mass $M$ at the CM.
        \end{itemize}
        \item $\dot{\vec{J}}=\sum_\alpha\vec{r}_\alpha\times\vec{F}_\alpha$.
    \end{enumerate}
    \item Recap.
    \begin{itemize}
        \item Last time, we found that there's a huge simplification we can make because all the particles in a rigid body are locked together.
        \begin{itemize}
            \item The simplification is that $\vec{J}=\overleftrightarrow{I}\vec{\omega}$, where $\overleftrightarrow{I}$ is the moment of inertia tensor.
            \begin{itemize}
                \item Jerison writes out the matrix formula all over again.
            \end{itemize}
            \item Key point: $\overleftrightarrow{I}$ is an \emph{intrinsic} property of the rigid body, playing the role of mass.
            \item If we have a continuous object, the sums over indices $\alpha$ turn into an integral! Recall this from prior courses.
            \item Compare $\vec{J}=\overleftrightarrow{I}\vec{\omega}$ to $\vec{P}=M\dot{\vec{R}}$; there is a similar structure in the equations.
        \end{itemize}
        \item Special case: Rotation about a fixed axis.
        \begin{itemize}
            \item We're headed toward the \textbf{compound pendulum}.
            \item For such a problem, we use cylindrical coordinates.
            \begin{itemize}
                \item Jerison rewrites the coordinate conversions.
            \end{itemize}
            \item We take $\vec{\omega}$ to lie in the $\khat$ direction via $\vec{\omega}=\omega\,\khat$.
            \item The moment with which we're most concerned is $I_{zz}$, defined as previously. Differentiating gets us from $J_z=I_{zz}\omega_z$ to $\dot{J}_z=I_{zz}\dot{\omega}$\footnote{Where did the subscript $z$ come from and why did it promptly disappear?? I don't think there's anything special going on here; notation is just wacky and confusing.}.
            \item From here, we can determine the kinetic energy to be $T=I_{zz}\omega^2/2$ where we recall that $\dot{\vec{r}}_\alpha=\vec{\omega}\times\vec{r}_\alpha=\rho_\alpha\omega\,\hat{\phi}$.
        \end{itemize}
        \item The EOMs for this system are given by $\dot{\vec{J}}=\sum_\alpha\vec{r}_\alpha\times\vec{F}_\alpha$.
        \begin{itemize}
            \item We're mostly interested in the $z$ component, i.e., $\dot{J}_z=\sum_\alpha\rho_\alpha F_\phi$.
        \end{itemize}
    \end{itemize}
    \item Sometimes, it can be useful to separate out the forces into axial forces and other forces via
    \begin{equation*}
        \dot{\vec{P}} = M\ddot{\vec{R}}
        = \vec{Q}+\sum_\alpha\vec{F}_\alpha
    \end{equation*}
    \begin{itemize}
        \item $\vec{Q}$ is the force on the axis and $\sum_\alpha\vec{F}_\alpha$ denotes other forces.
    \end{itemize}
    \item To make calculations, it will additionally be useful to have the following expression. For a rotating body, $\ddot{\vec{R}}$ can be found as follows: Since $\dot{\vec{R}}=\vec{\omega}\times\vec{R}$, we have that
    \begin{equation*}
        \ddot{\vec{R}} = \dot{\vec{\omega}}\times\vec{R}+\vec{\omega}\times\dot{\vec{R}}
        = \dot{\vec{\omega}}\times\vec{R}+\vec{\omega}\times(\vec{\omega}\times\vec{R})
    \end{equation*}
    \item The above expression for $\ddot{\vec{R}}$ holds true in general.
    \item If we specialize to the case of rotation about an axis, we can obtain a more tailored expression.
    \begin{itemize}
        \item First, choose the origin so that $z_\text{cm}=0$.
        \item Then the above expression simplifies to
        \begin{equation*}
            \ddot{\vec{R}} = \dot{\omega}\,\hat{z}\times R\,\hat{\rho}+\omega\,\hat{z}\times(\omega\,\hat{z}\times R\,\hat{\rho})
            = R\dot{\omega}\,\hat{\phi}-\omega^2R\,\hat{\rho}
            = R\ddot{\phi}\,\hat{\phi}-\dot{\phi}^2R\,\hat{\rho}
        \end{equation*}
        \item The right term above is tangential acceleration minus centripetal acceleration.
    \end{itemize}
    \item Example: Compound pendulum.
    \begin{figure}[H]
        \centering
        \begin{tikzpicture}
            \small
            \draw [stealth-stealth] (2,0) node[right]{$y$} -- (0,0) coordinate (o) -- (0,-2) coordinate (x) node[below]{$x$};
            \fill circle (1.5pt);
    
            \footnotesize
            \coordinate (c) at (-50:0.8);
            \draw [pux,thick,rotate around={-50:(c)}] (c) ellipse (1.5cm and 1cm);
            \draw [->] (0,0) -- node[right,yshift=1mm]{$\vec{R}$} (c);
            \draw [->] (c) -- node[right]{$\vec{F}$} ++(0,-1);
            \pic [draw,angle radius=4mm,angle eccentricity=1.4,pic text={$\phi$}] {angle=x--o--c};
        \end{tikzpicture}
        \caption{Compound pendulum.}
        \label{fig:compoundPendulum}
    \end{figure}
    \begin{itemize}
        \item We want to look at the force on the pivot.
        \item We define a new coordinate system as in Figure \ref{fig:compoundPendulum}. Explicitly, $\hat{x}$ points straight downwards and $\hat{y}$ points straight rightwards.
        \item We put our pendulum's center of mass such that it rotates through angle $\phi$.
        \item At this point, we have
        \begin{align*}
            T &= \frac{1}{2}I_{zz}\dot{\phi}^2&
            V &= M\vec{g}\cdot\vec{R} = -MgR\cos\phi
        \end{align*}
        \item Thus, our Lagrangian is
        \begin{equation*}
            L = T-V = \frac{1}{2}I_{zz}\dot{\phi}^2+MgR\cos\phi
        \end{equation*}
        \item Going forward, we will denote $I_{zz}$ with $I$.
        \item It follows from the E-L equation that our EOM is
        \begin{align*}
            I\ddot{\phi} &= -MgR\sin\phi\\
            \ddot{\phi} &= -\frac{MgR}{I}\sin\phi\\
            &= -\frac{g}{\ell}\sin\phi
        \end{align*}
        where $\ell=I/MR$.
        \begin{itemize}
            \item $\ell$ defines the \textbf{equivalent simple pendulum}.
        \end{itemize}
        \item From here, we can solve for the force on the pivot as a function of $\phi$ (we could also go through $\phi(t)$, and solve for $F(t)$ if we desired).
        \begin{itemize}
            \item We start with the conservation of energy
            \begin{equation*}
                \frac{1}{2}I\dot{\phi}^2-MgR\cos\phi = E
            \end{equation*}
            \item It follows that
            \begin{equation*}
                \dot{\phi}^2 = \frac{E+MgR\cos\phi}{I/2}
                = \frac{2E}{MR\ell}+\frac{2g}{\ell}\cos\phi
            \end{equation*}
            \item We want to solve for $\vec{Q}$ from $M\ddot{\vec{R}}=\vec{Q}+\sum_\alpha\vec{F}_\alpha$.
            \item Here, the only relevant external force is our gravitational force $Mg\cos\phi\,\hat{\rho}-Mg\sin\phi\,\hat{\phi}$.
            \item We also found previously that $\ddot{\vec{R}}=R\ddot{\phi}\,\hat{\phi}-\dot{\phi}^2R\,\hat{\rho}$. Thus,
            \begin{equation*}
                MR\ddot{\phi}\,\hat{\phi}-MR\dot{\phi}^2\,\hat{\rho} = \vec{Q}+Mg\cos\phi\,\hat{\rho}-Mg\sin\phi\,\hat{\phi}
            \end{equation*}
            \item Splitting this vector equation into scalar equations, we obtain
            \begin{align*}
                Q_\rho &= -MR\dot{\phi}^2-Mg\cos\phi&
                Q_z &= 0&
                Q_\phi &= MR\ddot{\phi}+Mg\sin\phi
            \end{align*}
            \item Substituting from the conservation of energy and EOM, we obtain
            \begin{align*}
                Q_\rho &= -\frac{2E}{\ell}-Mg\left( 1+\frac{2R}{\ell} \right)\cos\phi&
                Q_z &= 0&
                Q_\phi &= Mg\left( 1-\frac{R}{\ell} \right)\sin\phi
            \end{align*}
            \item These are the final formulae for the forces on pivot as a function of $\phi$.
        \end{itemize}
    \end{itemize}
    \item \textbf{Equivalent simple pendulum}: The simple pendulum having the same equation of motion as our extended body.
    \item What happens in a similar system if it receives a "sudden blow" or impulse?
    \begin{figure}[h!]
        \centering
        \begin{tikzpicture}
            \small
            \draw [stealth-stealth] (2,0) node[right]{$y$} -- (1.5,0) -- (1.5,-0.5) node[below]{$x$};
            \node at (1.7,-0.2) {$+$};
    
            \footnotesize
            \fill
                (0,0) circle (1.5pt)
                (0,-1.6) circle (1.5pt)
            ;
            \coordinate (c) at (0,-0.8);
            \draw [pux,thick,rotate around={-90:(c)}] (c) ellipse (1.5cm and 1cm);
            \draw [dashed] (0,0) -- node[right]{$d$} ($(0,0)!2!(c)$);
    
            \draw [->] (-2.3,-1.6) -- node[above]{$\vec{K}$} ++(0.9,0);
            \draw [->,shorten >=2pt] (0.7,0) -- node[above]{$\vec{S}$} (0,0);
        \end{tikzpicture}
        \caption{The "sweet spot" of a compound pendulum.}
        \label{fig:compoundPendulum2}
    \end{figure}
    \begin{itemize}
        \item Such pendulums have a sweet spot or equilibrium where the CM is just hanging down.
        \item We imagine that we kick the pendulum with impulse $\vec{K}$ in the $\hat{y}$ direction (using our modified coordinate system), as shown above.
        \item We have that $K\hat{y}=\vec{K}=\vec{F}\Delta t$.
        \item Let $\vec{S}=\vec{Q}\Delta t$.
        \item What we'll see is that there is a special value of $d$ (between the pivot and CM) for which $\vec{\rho}$ vanishes!
        \item During the short interval,
        \begin{equation*}
            I\ddot{\phi} = -MgR\sin\phi+Fd
        \end{equation*}
        \item We make the approximation that $\ddot{\phi}$ is constant during $\Delta t$ and that $\sin\phi=0$.
        \item It follows that
        \begin{equation*}
            \omega_\text{final} = \ddot{\phi}\Delta t
            = F\Delta t\frac{d}{I}
            = \frac{Kd}{I}
        \end{equation*}
        \item Additionally, we have that $\dot{\vec{P}}=\vec{Q}+\vec{F}$ so that
        \begin{equation*}
            P_\text{final} = \dot{P}\Delta t
            = -Q\Delta t+F\Delta t
            = -S+K
        \end{equation*}
        \item But we also know that
        \begin{equation*}
            P_\text{final} = M\dot{R}_\text{final}
            = M\omega_\text{final}R
            = \frac{MKdR}{I}
        \end{equation*}
        \item Thus, putting everything together, we obtain
        \begin{align*}
            \frac{MKdR}{I} &= -S+K\\
            S &= K\left( 1-\frac{MdR}{I} \right)
        \end{align*}
        \item Thus, $S$ vanishes if we choose $d=\ell=I/MR$.
        \item Takeaway: Regardless of the shape of our pendulum, if we hit it at the distance of the equivalent simple pendulum, we'll have no impulse on the pivot.
        \item This is the "sweet spot" of our baseball bat or whatever, the point at which we can swing the bat, hit the ball, and the maximum KE will be transferred to the ball and not to our hands (the pivot).
    \end{itemize}
\end{itemize}



\section{Office Hours (Jerison)}
\begin{itemize}
    \item \marginnote{11/6:}The final will slant toward the second half of the course, but everything is fair game.
    \item Is there an abstract environment in which we can view mass vs. angular mass and momentum vs. angular momentum, etc. as special cases of the same generalized construct?
    \begin{itemize}
        \item Yes.
        \item One answer.
        \begin{itemize}
            \item We can get this mapping from a speed-type thing to a momentum-type thing with linear operators.
            \item A tensor is a mathematical object with some kind of geometrical meaning independent of the coordinate basis.
        \end{itemize}
        \item Another answer.
        \begin{itemize}
            \item These are both examples of equations of motion that come from the Lagrangian (think \emph{generalized} mass, \emph{generalized} momentum, \emph{generalized} force, etc.).
        \end{itemize}
    \end{itemize}
    \item Could you post the KE of a free particle derivation?
    \item There will not be another \emph{in-class} review session, but she will hold one outside of class.
    \item We will get to Euler angles on Friday.
\end{itemize}



\section{Moment of Inertia Tensor; Principal Axis Rotation}
\begin{itemize}
    \item \marginnote{11/8:}Outline.
    \begin{itemize}
        \item Moment of inertia tensor.
        \begin{itemize}
            \item What is a tensor?
            \item Principal axes.
            \item Calculating moments of inertia.
        \end{itemize}
        \item Rotation about a principal axis.
        \begin{itemize}
            \item Precession.
        \end{itemize}
    \end{itemize}
    \item Next time.
    \begin{itemize}
        \item Stability of rotation about a principal axis.
        \item Euler angles.
        \item Lagrangian for rigid bodies.
    \end{itemize}
    \item Recall.
    \begin{itemize}
        \item Our EOMs are
        \begin{align*}
            \dot{\vec{P}} &= M\ddot{\vec{R}} = \sum_\alpha\vec{F}_\alpha&
            \dot{\vec{J}} &= \sum_\alpha\vec{r}_\alpha\times\vec{F}_\alpha
        \end{align*}
        \item Last time, we talked about rotation about a fixed axis.
        \item We've also seen (more generally) that if $\vec{\omega}=\omega_x\ihat+\omega_y\jhat+\omega_z\khat$, then the angular momentum is given by
        \begin{equation*}
            \vec{J} = \overleftrightarrow{I}\vec{\omega}
        \end{equation*}
    \end{itemize}
    \item \textbf{Tensor}: A mathematical object that has geometric meaning independent of the coordinate basis.
    \item What is a tensor?
    \begin{itemize}
        \item She won't belabor the point because most of this machinery is orthogonal to our present aims.
        \item The "geometric meaning" alluded to in the definition has to be some kind of multilinear relationship, usually between vectors.
        \item In particular, $\overleftrightarrow{I}$ is an intrinsic property of the rigid body and its geometry.
        \begin{itemize}
            \item Its \emph{numerical} representation will change with the basis, though.
        \end{itemize}
        \item To calculate it, we need to be able to define it in a particular basis.
        \begin{itemize}
            \item The tensor comes prepackaged with (1) a definition in one basis and (2) a rule about how to change bases.
        \end{itemize}
        \item So, in our specific example, $\overleftrightarrow{I}$ is the linear operator that takes $\vec{\omega}$ and returns to you $\vec{J}$ for your rigid body.
        \item The rule to calculate entries of $\overleftrightarrow{I}$ is: Start with the $3\times 3$ matrix and then employ
        \begin{align*}
            I_{xx} &= \iiint\rho_m(\vec{r})(z^2+y^2)&
            I_{xy} &= -\iiint\rho_m(\vec{r})xy&
        \end{align*}
        and the like where herein, $\rho_m$ is the density (i.e., mass/volume), not the radial coordinate.
        \item Change of basis rule: If you have a change-of-basis matrix $R$, then $\overleftrightarrow{I}$ in your new basis looks like $R^{-1}\overleftrightarrow{I}R$.
        \item Note that $\overleftrightarrow{I}$ is called a $\binom{1}{1}$ tensor since it has 1 \textbf{contravariant} and 1 \textbf{covariant} dimension, meaning that it is like a regular matrix with 1 dimension that transforms as row vectors and 1 dimension that transforms as column vectors.
        \item Other examples of tensors.
        \begin{itemize}
            \item Scalars: Rank 0 tensors (same in any dimension).
            \item Vectors: Rank 1 tensors (can be row or column vectors).
            \item Metrics: There are $\binom{0}{2}$ tensors which do \emph{not} transform as matrices, even though they are arrays of numbers.
        \end{itemize}
        \item We don't need to worry about any of this stuff if we don't want to.
    \end{itemize}
    \item Note that since $I_{xy}=I_{yx}$, etc., $\overleftrightarrow{I}$ is \textbf{symmetric}. Thus, by the real spectral theorem, it is orthonormally diagonalizable.
    \begin{itemize}
        \item This implies that $\overleftrightarrow{I}$ has three real eigenvalues.
        \item Moreover, the eigenvectors of $\overleftrightarrow{I}$ are orthonormal.
        \item Thus, we may use the eigenvectors of $\overleftrightarrow{I}$ to define an orthonormal basis of 3D space. We call these eigenvectors the \textbf{principal axes} $\vec{e}_1,\vec{e}_2,\vec{e}_3$. Thus, in principle, we can find these for any object we choose, even though in any object we study in this class, it will be obvious which axes are which.
        \item In the special basis of the principal axes, $\overleftrightarrow{I}$ is diagonal, i.e., $\overleftrightarrow{I}=\diag(I_{xx},I_{yy},I_{zz})$. It follows that
        \begin{equation*}
            \vec{J} = I_1\omega_1\vec{e}_1+I_2\omega_2\vec{e}_2+I_3\omega_3\vec{e}_3
        \end{equation*}
    \end{itemize}
    \item We now put some of these tensor machinations to good use.
    \begin{itemize}
        \item We begin with a couple of observations and a consequence. We then relate these back to principal axes.
        \item Observe that we can express the kinetic energy as follows.
        \begin{equation*}
            T = \sum_\alpha\frac{1}{2}m_\alpha\dot{\vec{r}}_\alpha{}^2
            = \sum_\alpha\frac{1}{2}m_\alpha(\vec{\omega}\times\vec{r}_\alpha)^2
            = \sum_\alpha\frac{1}{2}m_\alpha[\omega^2r_\alpha^2-(\vec{\omega}\cdot\vec{r}_\alpha)^2]
        \end{equation*}
        \begin{itemize}
            \item A derivation of the vector algebra identity $(\vec{u}\times\vec{v})^2=u^2v^2-(\vec{u}\cdot\vec{v})^2$ can be found in \textcite{bib:KibbleBerkshire}.
        \end{itemize}
        \item Observe that we can express the angular momentum as follows.
        \begin{equation*}
            \vec{J} = \sum_\alpha m_\alpha\vec{r}_\alpha\times\dot{\vec{r}}_\alpha
            = \sum_\alpha m_\alpha\vec{r}_\alpha\times(\vec{\omega}\times\vec{r}_\alpha)
            = \sum_\alpha m_\alpha[r_\alpha^2\vec{\omega}-(\vec{r}_\alpha\cdot\vec{\omega})\vec{r}_\alpha]
        \end{equation*}
        \item Comparing the above two results, we obtain
        \begin{equation*}
            T = \frac{1}{2}\vec{\omega}\cdot\vec{J}
        \end{equation*}
        \item In particular, in the basis of principal axes,
        \begin{equation*}
            T = \frac{1}{2}I_1\omega_1^2+\frac{1}{2}I_2\omega_2^2+\frac{1}{2}I_3\omega_3^2
        \end{equation*}
        \item We can use the above expression to get the Lagrangian for general rigid body motion.
        \item A few notes on this.
        \begin{itemize}
            \item $\vec{e}_1,\vec{e}_2,\vec{e}_3$ rotate with the body.
            \item $\vec{J}=\overleftrightarrow{I}\vec{\omega}$ implies that in general, $\vec{J}$ is not parallel to $\vec{\omega}$. However, if $\vec{\omega}$ lies along one of $\vec{e}_1,\vec{e}_2,\vec{e}_3$, then $\vec{J}$ is parallel to $\vec{\omega}$.
        \end{itemize}
    \end{itemize}
    \item We now consider rigid bodies with certain symmetries.
    \item \textbf{Symmetric body}: A rigid body for which two of the moments of inertia (usually taken to be $I_1,I_2$) are equal.
    \item \textbf{Totally symmetric body}: A rigid body for which all three of the moments of inertia are equal.
    \item Examples of (totally) symmetric bodies.
    \begin{itemize}
        \item A cylinder and square pyramid are both symmetric.
        \item A sphere and cube are both totally symmetric.
    \end{itemize}
    \item We'll mostly be dealing with \emph{symmetric} bodies.
    \item In this case:
    \begin{itemize}
        \item We have that
        \begin{equation*}
            \vec{J} = I_1(\omega_1\vec{e}_1+\omega_2\vec{e}_2)+I_3\omega_3\vec{e}_3
        \end{equation*}
        \item Thus, any orthonormal vectors in the plane defined by $\vec{e}_1,\vec{e}_2$ can serve as principal axes.
    \end{itemize}
    \item In the case of a totally symmetric object, any three orthonormal vectors serve as principal axes, and $\vec{J}$ is always parallel to $\vec{\omega}$.
    \item Calculating $\overleftrightarrow{I}$.
    \begin{enumerate}
        \item If we take $\vec{r}=\vec{R}+\vec{r}{\,}^*$, then since $\vec{R}^*=0$,
        \begin{equation*}
            \sum_\alpha m_\alpha x^* = \sum_\alpha m_\alpha y^* = \sum_\alpha m_\alpha z^* = 0
        \end{equation*}
        \begin{itemize}
            \item Let $\vec{R}=(X,Y,Z)$.
            \item The above identities imply that the cross terms work out as follows.
            \begin{equation*}
                I_{xy} = -\sum_\alpha m_\alpha(X+x^*)(Y+y^*)
                = -MXY-\sum_\alpha m_\alpha x_\alpha^*y_\alpha^*
            \end{equation*}
            \item Similarly, for the moments of inertia,
            \begin{equation*}
                I_{xx} = M(Y^2+Z^2)+I_{xx}^*
            \end{equation*}
            \item The above equation merits additional comment.
            \begin{itemize}
                \item It decomposes the moment of inertia into the sum of the moment of the center of mass about the origin and the moment of inertia relative to the center of mass $\vec{R}$.
                \item This is the \textbf{parallel axis theorem}.
            \end{itemize}
        \end{itemize}
        \item Objects with 3 perpendicular symmetry planes.
        \begin{itemize}
            \item Picture a cylinder, ellipsoid, or parallelepiped with uniform density and three axes of lengths $2a,2b,2c$.
            \item Then
            \begin{align*}
                I_1^* &= M(\lambda_yb^2+\lambda_zc^2)&
                I_2^* &= M(\lambda_xa^2+\lambda_zc^2)&
                I_3^* &= M(\lambda_xa^2+\lambda_yb^2)
            \end{align*}
            where\dots
            \begin{itemize}
                \item $\lambda_x=\lambda_y=\lambda_z=1/5$ for an ellipsoid;
                \item $\lambda_x=\lambda_y=\lambda_z=1/3$ for a parallelepiped;
                \item $\lambda_x=\lambda_y=1/4$ and $\lambda_z=1/3$ for a cylinder.
            \end{itemize}
            \item The derivation of the above results is on \textcite[209-11]{bib:KibbleBerkshire}.
            \begin{itemize}
                \item We should look through this as we may be expected to do the integrals!
                \item Known by the name, \textbf{Routh's rule}.
            \end{itemize}
            \item What are the $\lambda$'s?
            \begin{itemize}
                \item It's just a number that has to do with the geometry of the subscripted axis.
            \end{itemize}
        \end{itemize}
    \end{enumerate}
    \item An interesting case: The effect of a small force on an axis; \textbf{precession}.
    \begin{itemize}
        \item Imagine an object that is spinning fairly rapidly about one of its axes.
        \item Assume that we have a symmetric body and that initially, $\vec{\omega}=\omega\vec{e}_3$.
        \item It follows that initially, $\vec{J}=I_3\omega_3\vec{e}_3$.
        \item In the case of no external forces, we have
        \begin{equation*}
            \dot{\vec{J}} = I_3\dot{\vec{\omega}}_3
            = \sum\vec{r}_\alpha\times\vec{F}_\alpha
            = 0
        \end{equation*}
        \item Now imagine we exert a small force $\vec{F}$ at a distance $\vec{r}$ up the axis from the CM/origin.
        \item It follows that $\dot{\vec{J}}=I_3\dot{\vec{\omega}}=\vec{r}\times\vec{F}$.
        \item Thus, $\dot{\vec{J}}$ is perpendicular to $\vec{\omega}$ and $\vec{\omega}$ changes direction, so the system turns.
    \end{itemize}
    \item For example, consider a system consisting of a rolling bicycle wheel under gravity.
    \begin{figure}[h!]
        \centering
        \begin{tikzpicture}
            \footnotesize
            \begin{scope}[rotate=-30]
                \draw [->] (0,0) -- (-1.2,0) node[below right=-2pt,yshift=-5pt]{$\vec{\omega}$};
                \draw [->] (-1.2,0) -- ++(40:3mm) node[above]{$\dot{\vec{\omega}}$};
    
                \draw [white,line width=4pt] ellipse (7mm and 1cm);
                \draw [pux,thick] ellipse (7mm and 1cm);
    
                \draw [->] (0,-1) -- node[above left=-1mm]{$\vec{r}$} (0,0);
            \end{scope}
            \draw [->] (0,0) -- node[right=-2pt]{$M\vec{g}$} (0,-0.5);
        \end{tikzpicture}
        \caption{Why a bicycle wheel turns.}
        \label{fig:bicycleWheel}
    \end{figure}
    \begin{itemize}
        \item In this case, the perpendicular force drives the wheel to turn to the right, instead of immediately falling over.
        \item Keep in mind the precise placement of all the vectors in this image, especially $\vec{r}$ since the pivot point is at the bottom of the wheel (on the pavement).
    \end{itemize}
    \item At this point, we can analyze the motion of a top/gyroscope!
    \begin{figure}[H]
        \centering
        \includegraphics[width=0.25\linewidth]{../ExtFiles/topGyroscope.png}
        \caption{A spinning top/gyroscope.}
        \label{fig:topGyroscope}
    \end{figure}
    \begin{itemize}
        \item We have that
        \begin{align*}
            \dot{\vec{J}} &= \vec{r}\times\vec{F}\\
            I_3\dot{\vec{\omega}} &= R\vec{e}_3\times(-Mg\khat)\\
            I_3\omega\dot{\vec{e}}_3 &= MgR\khat\times\vec{e}_3\\
            \dot{\vec{e}}_3 &= \frac{MgR}{I_3\omega}\khat\times\vec{e}_3
        \end{align*}
        \item Define
        \begin{equation*}
            \vec{\Omega} = \frac{MgR}{I_3\omega}\khat
        \end{equation*}
        \item Then
        \begin{equation*}
            \dot{\vec{e}}_3 = \vec{\Omega}\times\vec{e}_3
        \end{equation*}
        \item Thus, $\vec{e}_3$ rotates about the $\khat$ axis (which is the direction of $\vec{\Omega}$) at rate $\Omega$. This is precession!
        \item We make the approximation that the value for $\Omega\ll\omega$, or $I_3\omega^2/2\gg MgR$.
        \item We are also making the approximation that $\vec{J}$ points in the $\vec{\omega}$ direction ($\vec{e}_3$ direction), which is not quite true due to the $\Omega$ contribution.
    \end{itemize}
\end{itemize}



\section{Euler's Angles; Freely Rotating Symmetric Body}
\begin{itemize}
    \item \marginnote{11/10:}Recap.
    \begin{itemize}
        \item Stability of rotation about a principal axis.
    \end{itemize}
    \item Today.
    \begin{itemize}
        \item Euler angles.
        \item Freely rotating body.
    \end{itemize}
    \item Recall.
    \begin{itemize}
        \item Last time, we talked about the moment of inertia tensor $\overleftrightarrow{I}$.
        \item Before you diagonalize it, this $3\times 3$ matrix has an element like $I_{xy}$ in each slot.
        \item Moreover, since it is a real symmetric matrix, the moment of inertia tensor is orthonormally diagonalizable.
        \begin{itemize}
            \item We call it's eigenvectors the principal axes.
        \end{itemize}
        \item In general, we will deal with nice symmetric objects like the cylinder, which you can just look at and see its principal axes.
        \begin{itemize}
            \item Moreover, in the particular case of the cylinder, \emph{symmetric} has the additional meaning that $I_1=I_2$.
            \item In this case, we can choose any two orthogonal vectors in the span of $\vec{e}_1,\vec{e}_2$ to be the principal axes.
        \end{itemize}
        \item Note that to find the principal axes rigorously, the rule is that the cross terms (i.e., those $I_{xy}$ in which the two subscripted variables differ and which thus do not lie along the diagonal of $\overleftrightarrow{I}$) equal zero.
        \begin{itemize}
            \item This occurs when integrating $m_\alpha xy$ over the whole object yields zero.
        \end{itemize}
        \item In the principal axes basis, $\overleftrightarrow{I}=\diag(I_1,I_2,I_3)$.
        \begin{itemize}
            \item Calculate $I_1,I_2,I_3$ either by choosing the principal axes from the beginning or by choosing nonstandard axes and diagonalizing.
        \end{itemize}
        \item Specific example: The rotating top.
        \begin{itemize}
            \item We often want to use the pivot point as the origin (which may well not be the CM of the system).
            \item To find the moment of inertia for bodies like this, we usually use the parallel axis theorem.
            \item Beware, though, that the principal axes at the CM and a pivot point need not be parallel. However, they are parallel (and thus can be taken to be identical) if the new origin is on a principal axis that passes through the center of mass.
        \end{itemize}
    \end{itemize}
    \item To start today, we generalize rotation.
    \begin{itemize}
        \item What if we can have any instantaneous angular velocity $\vec{\omega}$?
        \item The angular momentum in the basis of the principal axes will still be
        \begin{equation*}
            \vec{J} = I_1\omega_1\hat{e}_1+I_2\omega_2\hat{e}_2+I_3\omega_3\hat{e}_3
        \end{equation*}
        \begin{itemize}
            \item Recall that $\hat{e}_1,\hat{e}_2,\hat{e}_3$ rotate with the body.
        \end{itemize}
        \item To find our EOM, we start with our previously discovered EOMs.
        \begin{equation*}
            \left( \dv{\vec{J}}{t} \right)_\text{inertial} = \sum_\alpha\vec{r}_\alpha\times\vec{F}_\alpha
            = \vec{G}
            = \dot{\vec{J}}+\vec{\omega}\times\vec{J}
        \end{equation*}
        \begin{itemize}
            \item In particular, $\vec{G}$ is the net external torque and $\dot{\vec{J}}$ is the rate of change of the angular momentum within the rotating frame.
        \end{itemize}
        \item In this scenario, $\dot{\vec{J}}$ is easily found by differentiating the equation two lines above:
        \begin{equation*}
            \dot{\vec{J}} = I_1\dot{\omega}_1\hat{e}_1+I_2\dot{\omega}_2\hat{e}_2+I_3\dot{\omega}_3\hat{e}_3
        \end{equation*}
        \item It follows by combining the above three equations that
        \begin{equation*}
            \vec{G} = \left( I_1\dot{\omega}_1\hat{e}_1+I_2\dot{\omega}_2\hat{e}_2+I_3\dot{\omega}_3\hat{e}_3 \right)+\left( \omega_1\hat{e}_1+\omega_2\hat{e}_2+\omega_3\hat{e}_3 \right)\times\left( I_1\omega_1\hat{e}_1+I_2\omega_2\hat{e}_2+I_3\omega_3\hat{e}_3 \right)
        \end{equation*}
        \item Thus, evaluating the cross product, the componentwise EOMs are
        \begin{align*}
            I_1\dot{\omega}_1+(I_3-I_2)\omega_2\omega_3 &= G_1\\
            I_2\dot{\omega}_2+(I_1-I_3)\omega_3\omega_1 &= G_2\\
            I_3\dot{\omega}_3+(I_2-I_1)\omega_1\omega_2 &= G_3
        \end{align*}
        \begin{itemize}
            \item We will discuss all of these next time.
        \end{itemize}
    \end{itemize}
    \item We now discuss a special case of the above motion.
    \item No external torques: The situation wherein $\vec{G}=0$.
    \begin{itemize}
        \item Suppose that we initially have some $\omega_3$ but that $\omega_1=\omega_2=0$.
        \begin{itemize}
            \item This is rotation about just one principal axis.
        \end{itemize}
        \item It follows that $\omega_1,\omega_2,\omega_3$ are constant and hence rotation continues about the same axis.
    \end{itemize}
    \item When is rotation about a principal axis stable?
    \begin{itemize}
        \item Suppose that $\vec{\omega}=\omega\hat{e}_3$, but this time, a small perturbation introduces angular momentum about one or more of the other axes.
        \begin{itemize}
            \item Mathematically, we assume $\omega_1,\omega_2\ll\omega_3$.
            \item Thus, we neglect terms that contain a product of $\omega_1$ and $\omega_2$.
        \end{itemize}
        \item Under these constraints, our EOMs become
        \begin{align*}
            I_1\dot{\omega}_1+(I_3-I_2)\omega_2\omega_3 &= 0\\
            I_2\dot{\omega}_2+(I_1-I_3)\omega_3\omega_1 &= 0\\
            I_3\dot{\omega}_3 &= 0
        \end{align*}
        \item The last line above implies that $\omega_3$ is constant.
        \item This leaves us with the task of solving the two remaining first-order, coupled ODEs.
        \item Try the ansatzs
        \begin{align*}
            \omega_1 &= a_1\e[pt]&
            \omega_2 &= a_2\e[pt]
        \end{align*}
        \item Then we get the following system of equations.
        \begin{equation*}
            \begin{cases}
                I_1pa_1\e[pt]+(I_3-I_2)a_2\e[pt]\omega_3 = 0\\
                I_2pa_2\e[pt]+(I_1-I_3)\omega_3a_1\e[pt] = 0
            \end{cases}
            \quad\Longrightarrow\quad
            \begin{cases}
                I_1pa_1+(I_3-I_2)a_2\omega_3 = 0\\
                I_2pa_2+(I_1-I_3)\omega_3a_1 = 0
            \end{cases}
        \end{equation*}
        \item We can solve this for two separate forms of the ratio $a_1/a_2$:
        \begin{align*}
            \frac{a_1}{a_2} &= \frac{-(I_3-I_2)\omega_3}{I_1p}&
            \frac{a_1}{a_2} &= \frac{I_2p}{-(I_1-I_3)\omega_3}
        \end{align*}
        \begin{itemize}
            \item The left equation comes from dividing $I_1pa_1+(I_3-I_2)a_2\omega_3=0$ through by $a_2$ and rearranging.
            \item The right equation comes from dividing $I_2pa_2+(I_1-I_3)\omega_3a_1=0$ through by $a_2$ and rearranging.
        \end{itemize}
        \item It follows by transitivity that
        \begin{align*}
            \frac{I_2p}{-(I_1-I_3)\omega_3} &= \frac{-(I_3-I_2)\omega_3}{I_1p}\\
            I_1I_2p^2 &= \omega_3^2(I_3-I_2)(I_1-I_3)
        \end{align*}
        \item Thus, if
        \begin{equation*}
            (I_3-I_2)(I_1-I_3) > 0
        \end{equation*}
        then $p>0$ and the rotation is unstable.
        \item On the other hand, if the above term is less than zero, then $p$ is imaginary, so the rotation is purely oscillatory and hence stable.
        \item Takeaway:
        \begin{itemize}
            \item If $I_3$ is the smallest or largest of the moments (i.e., if $I_3>I_1,I_2$ or $I_1,I_2>I_3$), then the rotation is stable.
            \item If $I_3$ is the middle moment (i.e., if $I_1>I_3>I_2$ or $I_2>I_3>I_1$), the the rotation is unstable.
        \end{itemize}
        \item This is called the \textbf{tennis racket theorem}.
    \end{itemize}
    \item Example of the above.
    \begin{itemize}
        \item Consider a rectangular prism with longest axis $2a$, second longest $2b$, and third longest $2c$.
        \item We can calculate that $\hat{e}_3\parallel c$, $\hat{e}_1\parallel a$, and $\hat{e}_2\parallel b$.
        \item Now using Routh's rule, we have that
        \begin{align*}
            I_3 &= M\left( \frac{a^2}{3}+\frac{b^2}{3} \right)&
            I_2 &= M\left( \frac{a^2}{3}+\frac{c^2}{3} \right)&
            I_1 &= M\left( \frac{b^2}{3}+\frac{c^2}{3} \right)
        \end{align*}
        \begin{itemize}
            \item It follows that $I_3$ is largest, $I_2$ is middle, and $I_1$ is smallest.
            \item Note that the $1/3$ comes from integrating $x^2$.
        \end{itemize}
        \item Thus, if the prism is rotating around the smallest axis to begin with, it will remain stably spinning around that axis.
        \item If the prism is rotating head over heels, the rotation is unstable.
        \item And if the prism is rotating like a frisbee (i.e., around the largest axis), the rotation is also stable.
    \end{itemize}
    \item Euler angles.
    \begin{figure}[h!]
        \centering
        \includegraphics[width=0.47\linewidth]{../ExtFiles/EulerAngles.png}
        \caption{Euler angles.}
        \label{fig:EulerAngles}
    \end{figure}
    \begin{itemize}
        \item A method of specifying the orientation of an object in space that uses three angles.
        \item For rotation about the CM, these three angles will be our three DOFs for the system.
        \item Goal: Write $\vec{J},T$ in terms of these angles.
        \item Suppose our object starts such that it is oriented along $\ihat,\jhat,\khat$. We now want to go to an arbitrary new orientation. We do so in three steps.
        \begin{enumerate}
            \item Rotate it through an angle $\phi$ about $\khat$. Then
            \begin{equation*}
                \ihat,\jhat,\khat \mapsto \hat{e}_1'',\hat{e}_2',\khat
            \end{equation*}
            \item Rotate it through an angle $\theta$ about $\hat{e}_2'$. Then
            \begin{equation*}
                \hat{e}_1'',\hat{e}_2',\khat \mapsto \hat{e}_1',\hat{e}_2',\hat{e}_3
            \end{equation*}
            \item Finally, rotate it about an angle $\psi$ about $\hat{e}_3$. Then
            \begin{equation*}
                \hat{e}_1',\hat{e}_2',\hat{e}_3 \mapsto \hat{e}_1,\hat{e}_2,\hat{e}_3
            \end{equation*}
        \end{enumerate}
        \item It follows based on these definitions (see reasoning in \textcite{bib:KibbleBerkshire}) that
        \begin{equation*}
            \vec{\omega} = \dot{\phi}\,\khat+\dot{\theta}\,\hat{e}_2'+\dot{\psi}\,\hat{e}_3
        \end{equation*}
        \item But these bases are not ideal since these aren't our principal axis basis. Thus, we wish to define $\vec{\omega}$ in the principal axis basis.
        \item In the restrictive case of a symmetric body, $I_1=I_2$. Thus, we can choose $\hat{e}_1:=\hat{e}_1'$ and $\hat{e}_2:=\hat{e}_2'$ because we can choose \emph{any} vectors in this plane, as stated above.
        \item Additionally, we have that $\khat=-\sin\theta\,\hat{e}_1'+\cos\theta\,\hat{e}_3$.
        \item Thus,
        \begin{equation*}
            \vec{\omega} = \dot{\phi}(-\sin\theta\,\hat{e}_1'+\cos\theta\,\hat{e}_3)+\dot{\theta}\,\hat{e}_2'+\dot{\psi}\,\hat{e}_3
            = -\dot{\phi}\sin\theta\,\hat{e}_1'+\dot{\theta}\,\hat{e}_2'+(\dot{\psi}+\dot{\phi}\cos\theta)\,\hat{e}_3
        \end{equation*}
        \item Therefore, we independently have based on the above that
        \begin{equation*}
            \vec{J} = -I_1\dot{\phi}\sin\theta\,\hat{e}_1'+I_1\dot{\theta}\,\hat{e}_2'+I_3(\dot{\psi}+\dot{\phi}\cos\theta)\,\hat{e}_3
        \end{equation*}
        and
        \begin{equation*}
            T = \frac{1}{2}I\vec{\omega}^2
            = \frac{1}{2}I_1\dot{\phi}^2\sin^2\theta+\frac{1}{2}I_1\dot{\theta}^2+\frac{1}{2}I_3(\dot{\psi}+\dot{\phi}\cos\theta)^2
        \end{equation*}
    \end{itemize}
\end{itemize}



\section{Free Rotation; Hamilton's Equations}
\begin{itemize}
    \item \marginnote{11/13:}Outline.
    \begin{itemize}
        \item Free rotation.
        \begin{itemize}
            \item Lagrangian + precession under gravity.
        \end{itemize}
        \item Hamiltonian.
    \end{itemize}
    \item Last time.
    \begin{itemize}
        \item We defined the Euler angles $\theta,\phi,\psi$ so that $\vec{\omega}=\dot{\phi}\,\khat+\dot{\theta}\,\hat{e}_2'+\dot{\psi}\,\hat{e}_3$.
        \item For a symmetric body, $I_1=I_2$. Thus, we had $\vec{\omega}=-\dot{\phi}\sin\theta\,\hat{e}_1'+\dot{\theta}\,\hat{e}_2'+(\dot{\psi}+\dot{\phi}\cos\theta)\,\hat{e}_3$
        \begin{itemize}
            \item $\hat{e}_1',\hat{e}_2',\hat{e}_3$ are the principal axes of the object.
        \end{itemize}
        \item With $\vec{\omega}$ in terms of our principal axes basis, it was easy to write down expressions for $\vec{J}$ and $T$.
    \end{itemize}
    \item We now investigate the motion of such a freely rotating system in a couple of cases.
    \item Case 1: No external forces.
    \begin{figure}[h!]
        \centering
        \includegraphics[width=0.43\linewidth]{../ExtFiles/FreeRot1.png}
        \caption{Free rotation under no external forces.}
        \label{fig:FreeRot1}
    \end{figure}
    \begin{itemize}
        \item In this case, $\vec{J}$ is conserved, so we have
        \begin{equation*}
            \vec{J} = J\khat=-J\sin\theta\,\hat{e}_1'+J\cos\theta\,\hat{e}_3
        \end{equation*}
        \item By comparing this with last class's equation defining $\vec{J}$ in terms of the Euler angles, we obtain the componentwise equations
        \begin{align*}
            I_1\dot{\phi}\sin\theta &= J\sin\theta\\
            I_1\dot{\theta} &= 0\\
            I_3(\dot{\psi}+\dot{\phi}\cos\theta) &= J\cos\theta
        \end{align*}
        \item The middle equation above implies that $\theta$ is constant, from which it follows that $J\sin\theta$ and $J\cos\theta$ are constant.
        \item Thus, we can solve for\dots
        \begin{align*}
            \dot{\phi} &= \frac{J}{I_1}&
            \dot{\psi} &= \frac{J\cos\theta}{I_3}-\frac{J}{I_1}\cos\theta
        \end{align*}
        where all of the terms on the right above are constant.
        \item It follows that in this case, $\hat{e}_3$ is fixed at angle $\theta$ with respect to $\vec{J}$.
        \item Moreover, $\vec{\omega}$ (which depends on the three fixed quantities $\dot{\theta},\dot{\phi},\dot{\psi}$) is at a fixed angle with respect to $\khat$, precessing around $\khat$ with rate $\dot{\phi}$.
        \item It follows that
        \begin{align*}
            \vec{\omega} &= -\dot{\phi}\sin\theta\,\hat{e}_1'+(\dot{\psi}+\dot{\phi}\cos\theta)\,\hat{e}_3\\
            &= \frac{J\sin\theta}{I_1}\,\hat{e}_1'+\frac{J\cos\theta}{I_3}\,\hat{e}_3
        \end{align*}
        \item Separately, we may read from Figure \ref{fig:FreeRot1} that
        \begin{equation*}
            \vec{\omega} = \sin\beta\,\hat{e}_1'+\cos\beta\,\hat{e}_3
        \end{equation*}
        \item It follows by comparing the above two equations that
        \begin{equation*}
            \tan\beta = \frac{\sin\beta}{\cos\beta}
            = \frac{\frac{J\sin\theta}{I_1}}{\frac{J\cos\theta}{I_3}}
            = \frac{I_3}{I_1}\tan\theta
        \end{equation*}
        \item The \textbf{body cone} "rolls around" the \textbf{space cone}; that is, we can check that
        \begin{equation*}
            \dot{\psi}\sin\beta = \dot{\phi}\sin(\theta-\beta)
        \end{equation*}
        \begin{itemize}
            \item In particular, we have that
            \begin{align*}
                \dot{\psi}\sin\beta &= \dot{\psi}\cdot\frac{J\sin\theta}{I_1}\\
                &= \left( \frac{J\cos\theta}{I_3}-\frac{J}{I_1}\cos\theta \right)\cdot\dot{\phi}\sin\theta\\
                &= \dot{\phi}\left( \sin\theta\cdot\frac{J\cos\theta}{I_3}-\cos\theta\cdot\frac{J\sin\theta}{I_1} \right)\\
                &= \dot{\phi}(\sin\theta\cdot\cos\beta-\cos\theta\cdot\sin\beta)\\
                &= \dot{\phi}\sin(\theta-\beta)
            \end{align*}
        \end{itemize}
        \item The net motion is that the body is rotating on its body cone and also rotating about the axis.
    \end{itemize}
\end{itemize}



\section{Chapter 9: Rigid Bodies}
\emph{From \textcite{bib:KibbleBerkshire}.}
\begin{itemize}
    \item Covered a smattering of results from various sections.
    \item \marginnote{12/4:}A necessary and sufficient condition for equilibrium: "The sum of the forces and the sum of their moments are both zero" \parencite[198]{bib:KibbleBerkshire}.
    \begin{itemize}
        \item We see this mathematically from the equations
        \begin{align*}
            \dot{\vec{P}} &= M\ddot{\vec{R}} = \sum\vec{F}&
            \dot{\vec{J}} &= \sum\vec{r}\times\vec{F}
        \end{align*}
    \end{itemize}
    \item \textbf{Lamina}: A plane, two-dimensional object.
    \item Reconsider Figure \ref{fig:rectangularLamina} and the associated discussion.
    \begin{itemize}
        \item Letting $\vec{Q}$, once again, be the force on the pivot, we see that for equilibrium, we have
        \begin{equation*}
            \vec{Q} = (F,Mg,0)
        \end{equation*}
    \end{itemize}
    \item \textbf{Compound pendulum}: A rigid body pivoted about a horizontal axis and moving under gravity.
    \begin{itemize}
        \item An alternate way to obtain the energy conservation equation
        \begin{equation*}
            E = \frac{1}{2}I\dot{\phi}^2-MgR\cos\phi
        \end{equation*}
        is by multiplying the equation of motion $I\ddot{\phi}=-MgR\sin\phi$ by $\dot{\phi}$ and integrating\footnote{This is very similar to the trick used in the 10/13 lecture in the Lagrange undetermined multiplier example.}.
    \end{itemize}
    \item A sudden blow.
    \begin{itemize}
        \item Integrating $I\ddot{\phi}=-MgR\sin\phi$ over a short time interval yields
        \begin{equation*}
            I\omega = dK
        \end{equation*}
        \item The velocity of the center of mass immediately after the blow is $\omega R$, so the integral of the EOM is
        \begin{align*}
            M\dot{R} &= \vec{Q}+\sum\vec{F}\\
            M\omega R &= -S+k
        \end{align*}
        \item It follows by combining the above two equations that
        \begin{equation*}
            S = \left(1-\frac{MdR}{I} \right)K
        \end{equation*}
    \end{itemize}
    \item \textbf{Couple}: A system of forces with a resultant/net/sum torque but no resultant force.
    \begin{itemize}
        \item A couple imparts angular momentum but no linear momentum.
        \item In rigid body dynamics, couples have an effect on a body that is independent of their point of application.
    \end{itemize}
    \item \textbf{Principal moment of inertia}: Any one of the eigenvalues of the moment of inertia tensor.
    \item Shift of origin equations.
    \begin{align*}
        I_{xx} &= M(Y^2+Z^2)+I_{xx}^*&
        I_{xy} &= -MXY+I_{xy}^*
    \end{align*}
    \item Derivation of Routh's rule.
    \begin{itemize}
        \item We know that
        \begin{align*}
            I_{xx} &= \iiint\rho(\vec{r})(y^2+z^2)\dd[3]{\vec{r}}&
            I_{yy} &= \iiint\rho(\vec{r})(x^2+z^2)\dd[3]{\vec{r}}&
            I_{zz} &= \iiint\rho(\vec{r})(x^2+y^2)\dd[3]{\vec{r}}
        \end{align*}
        \item Thus, letting
        \begin{equation*}
            K_i = \iiint_V\rho i^2\dv{x}\dv{y}\dv{z}
        \end{equation*}
        for $i=x,y,z$ and denoting $I_1^*:=I_{xx}$, $I_2^*:=I_{yy}$, and $I_3^*:=I_{zz}$, we have that
        \begin{align*}
            I_1^* &= K_y+K_z&
            I_2^* &= K_x+K_z&
            I_3^* &= K_x+K_y
        \end{align*}
        \item Note that the mass of the body in question is given by
        \begin{equation*}
            M = \iiint_V\rho\dd{x}\dd{y}\dd{z}
        \end{equation*}
        \item $M,K_i$ depend on the end-to-end lengths $2a,2b,2c$ of each class of symmetric rigid body (e.g., ellipsoids, parallelepipeds, etc.).
        \item Change variables from $x,y,z$ to
        \begin{align*}
            x &= a\xi&
            y &= b\eta&
            z &= c\zeta
        \end{align*}
        \item Thus,
        \begin{align*}
            M = \rho abc\iiint_{V_0}\dd{\xi}\dd{\eta}\dd{\zeta}
        \end{align*}
        where $V_0$ is a standard symmetric rigid body of the given type (i.e., with $\rho=1$ and $a=b=c=1$).
        \item It follows by a similar result for each $K_i$ that
        \begin{align*}
            M &\propto \rho abc&
            K_x &\propto \rho a^3bc&
            K_y &\propto \rho ab^3c&
            K_z &\propto \rho abc^3
        \end{align*}
        \item Thus, each $K_i$ equals $\lambda_iMi^2$ for some scalar $\lambda_i$, the same for all bodies of the given type.
        \item To summarize, we have Routh's rule as follows.
        \begin{align*}
            I_1^* &= M(\lambda_yb^2+\lambda_zc^2)&
            I_2^* &= M(\lambda_xa^2+\lambda_zc^2)&
            I_3^* &= M(\lambda_xa^2+\lambda_yb^2)
        \end{align*}
        \item Lastly, we compute the scalars using integrals.
        \item Example: The standard body for an ellipsoid is a sphere of uniform density.
        \begin{itemize}
            \item We have
            \begin{equation*}
                K_z = \iiint_{V_0}\zeta^2\dd{\xi}\dd{\eta}\dd{\zeta}
                = \int_{-1}^1\zeta^2\pi(1-\zeta^2)\dd{\zeta}
                = \frac{4\pi}{15}
            \end{equation*}
            \item We also have
            \begin{equation*}
                M_0 = \frac{4\pi}{3}
            \end{equation*}
            \item Thus,
            \begin{equation*}
                \lambda_z = \frac{K_z}{M_01^2}
                = \frac{1}{5}
            \end{equation*}
        \end{itemize}
    \end{itemize}
    \item Precession and gyroscopic motion.
    \item Up until now, we have considered rotation about a fixed axis. Now, we will consider the case in which only one point on the axis is fixed. (Later, we will consider the case in which no points on the axis are fixed.)
    \item The effect of a small force on an axis (precession), visualized.
    \begin{figure}[H]
        \centering
        \begin{tikzpicture}
            \footnotesize
            \draw (0,-1.3) -- (0,-0.7);
    
            \draw [pux,thick] ellipse (1cm and 4mm);
            \draw [pux,thick,yshift=-3mm] (1,0) arc[start angle=0,end angle=-180,x radius=1cm,y radius=4mm];
            \draw [pux,thick]
                (1,0)  -- ++(0,-0.3)
                (-1,0) -- ++(0,-0.3)
            ;
    
            \draw [line width=4pt,white] (0,0) -- (0,1);
            \draw [->] (0,0) -- (0,1) node[below right]{$\vec{r}$};
            \draw [->] (0,1) -- (0,1.7) node[below right]{$\vec{\omega}$};
            \draw [->] (0,1) -- ++(-0.8,0.3) node[above right]{$\vec{F}$};
            \draw [->] (-0.8,1.3) -- ++(-1,-0.3) node[below right]{$\vec{r}\times\vec{F}$};
        \end{tikzpicture}
        \caption{The origin of precession.}
        \label{fig:precession}
    \end{figure}
    \begin{itemize}
        \item Pushing one way on a rotating body induces a perpendicular change!
    \end{itemize}
    \item On the spinning top.
    \begin{itemize}
        \item There are similarities between the precession of a spinning top ($\dot{\vec{e}}_3=\vec{\Omega}\times\vec{e}_3$\footnote{In what cases do we use $\vec{e}_i$ for a principal axis?? Should it not always be $\hat{e}_i$?}), the Larmor precession, and the precession of a satellite orbit.
        \item $\vec{\Omega}$ is independent of the inclination of the axis!
        \item We discuss what happens when $MgR\not\ll I_3\omega^2/2=T$ later (in Sections 10.3 and 12.4).
        \item Since $\Omega\propto1/I_3\omega$, to minimize the effect of a force on the axis, we should use a fat and rapidly spinning body.
    \end{itemize}
    \item Precession of the equinoxes is covered.
    \item A rigorous proof that an instantaneous angular velocity vector $\vec{\omega}$ always exists for a rotating rigid body.
    \item On unstable motion.
    \begin{itemize}
        \item Recall that the ansatzs used are only valid as long as $\omega_1,\omega_2$ are small; thus, exponential blowup is not the \emph{actual} motion, but just tells us that the motion will not stay stable.
    \end{itemize}
    \item Reasoning for why $\vec{\omega}=\dot{\phi}\,\khat+\dot{\theta}\,\hat{e}_2'+\dot{\psi}\,\hat{e}_3$.
    \begin{itemize}
        \item A small change in any of $\phi,\theta,\psi$ --- independently --- corresponds to a small rotation about $\khat,\hat{e}_2',\hat{e}_3$, respectively.
        \item Thus, a change in all three is a linear combination.
    \end{itemize}
    \item On the \textbf{Chandler wobble}.
\end{itemize}




\end{document}