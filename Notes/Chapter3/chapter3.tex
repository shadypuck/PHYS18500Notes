\documentclass[../notes.tex]{subfiles}

\pagestyle{main}
\renewcommand{\chaptermark}[1]{\markboth{\chaptername\ \thechapter\ (#1)}{}}
\setcounter{chapter}{2}

\begin{document}




\chapter{Energy and Angular Momentum}
\section{Energy and Conservative Forces in 3D; Angular Momentum}
\begin{itemize}
    \item \marginnote{10/6:}Recap.
    \begin{itemize}
        \item If $F(x,\dot{x},t)=F(x)$, then we can define $V(x)$.
        \item A bit more on kinetic, potential, and total energy in 1D.
    \end{itemize}
    \item Question: Is $\vec{F}(\vec{r},\dot{\vec{r}},t)=F(\vec{r})$ sufficient for the force to be conservative?
    \begin{itemize}
        \item Answer: No, it is not.
    \end{itemize}
    \item What \emph{is} a necessary and sufficient condition, then?
    \begin{itemize}
        \item If $T+V=E$, a constant, then we should have $\dv*{t}(T+V)=0$.
        \item Since
        \begin{align*}
            \dot{T} &= m(\dot{x}\ddot{x}+\dot{y}\ddot{y}+\dot{z}\ddot{z})
                = m\dot{\vec{r}}\cdot\ddot{\vec{r}}
                = \dot{\vec{r}}\cdot\vec{F}&
            \dot{V} &= \pdv{V}{x}\dot{x}+\pdv{V}{y}\dot{y}+\pdv{V}{z}\dot{z}
            = \dot{r}\cdot\vec{\nabla}V
        \end{align*}
        stating that $\dot{T}+\dot{V}=\dv*{t}(T+V)=0$ is equivalent to stating that
        \begin{equation*}
            \dot{\vec{r}}\cdot(\vec{F}+\vec{\nabla}V)
        \end{equation*}
        \item But from here, it follows that we must have $\vec{F}=-\vec{\nabla}V$.
        \item Takeaway: Conservative forces depend on $\vec{r}$ and can be written as $-\vec{\nabla}V$ for some scalar function $V$.
    \end{itemize}
    \item Can we express this condition more nicely? Yes!
    \begin{itemize}
        \item Claim: $\text{curl}\,(\vec{F})=\vec{\nabla}\times\vec{F}=0$ iff $\vec{F}=-\vec{\nabla}V$ for some scalar function $V$.
        \item Suppose $F=-\vec{\nabla}V$ for some scalar function $V$.
        \begin{itemize}
            \item Then since the curl of a gradient field is zero,
            \begin{equation*}
                \vec{\nabla}\times\vec{F} = \vec{\nabla}\times\vec{\nabla}V = 0
            \end{equation*}
        \end{itemize}
        \item Suppose $\vec{\nabla}\times\vec{F}=0$.
        \begin{itemize}
            \item To prove that $\vec{F}=-\vec{\nabla}V$ for some $V$, it will suffice to show that
            \begin{equation*}
                V(\vec{r}) = -\int_{\vec{r}_0}^{\vec{r}}\vec{F}\cdot\dd\vec{r'}
            \end{equation*}
            \item In particular, it will suffice to show that the function above is well defined. To do so, we will need to prove that the line integral on the right-hand side above is \textbf{path-independent}.
            \item But then by the equivalent path independence condition below, we need
            \begin{equation*}
                \oint_C\vec{F}\cdot\dd\vec{r} = 0
            \end{equation*}
            for all $C$.
            \item Applying \textbf{Stokes' theorem}, we obtain the equivalent condition
            \begin{equation*}
                \oint_C\vec{F}\cdot\dd\vec{r} = \iint_S(\vec{\nabla}\times\vec{F})\cdot\dd{\vec{S}} = \iint_S0\cdot\dd{\vec{S}} = 0
            \end{equation*}
            as desired.
        \end{itemize}
    \end{itemize}
    \item \textbf{Path-independent} (line integral): A line integral $\int_{\vec{r}_0}^{\vec{r}_1}\vec{A}\cdot\dd\vec{r}$ over some vector field $\vec{A}$ such that if $C_1,C_2$ are any two curves connecting $\vec{r}_0$ and $\vec{r}_1$, then
    \begin{equation*}
        \int_{C_1}\vec{A}\cdot\dd\vec{r} = \int_{C_2}\vec{A}\cdot\dd\vec{r}
    \end{equation*}
    \begin{figure}[h!]
        \centering
        \begin{tikzpicture}
            \footnotesize
            \draw [stealth-stealth] (0,3) -- (0,0) -- (3.5,0);
    
            \draw [help lines, ->] (0.5,0.8) -- ++(0.3,0.3);
            \draw [help lines, ->] (0.2,2.2) -- ++(0.3,0.3);
            \draw [help lines, ->] (1,0.3) -- ++(0.3,0.3);
            \draw [help lines, ->] (0.8,2) -- ++(0.3,0.3);
            \draw [help lines, ->] (1.4,1.5) -- ++(0.3,0.3);
            \draw [help lines, ->] (1.2,2.5) -- ++(0.3,0.3);
            \draw [help lines, ->] (1.8,1.2) -- ++(0.3,0.3);
            \draw [help lines, ->] (2.3,0.4) -- ++(0.3,0.3);
            \draw [help lines, ->] (2.4,2.2) -- ++(0.3,0.3);
            \draw [help lines, ->] (2.5,1) -- ++(0.3,0.3);
    
            \draw [pux,thick]
                (0.9,0.8) .. controls (3,0.5) and (2,2) .. node[near start,below=2pt]{$C_1$} (2.4,2.7)
                (0.9,0.8) .. controls (0.8,3) and (2,2) .. node[near end,above=2pt]{$C_2$} (2.4,2.7)
            ;
    
            \fill (0.9,0.8) circle (2pt) node[below left]{$\vec{r}_0$};
            \fill (2.4,2.7) circle (2pt) node[above right]{$\vec{r}_1$};
        \end{tikzpicture}
        \caption{Path independent line integral.}
        \label{fig:pathIndependent}
    \end{figure}
    \begin{itemize}
        \item An equivalent path independence condition may be obtained via inspection of Figure \ref{fig:pathIndependent}.
        \item Indeed, saying that the path integral along $C_1$ (from $\vec{r}_0$ to $\vec{r}_1$) equals that along $C_2$ (from $\vec{r}_0$ to $\vec{r}_1$) is equivalent to saying that the difference of the path integrals is equal to zero. Equivalently, the path integral along $C_1$ (from $\vec{r}_0$ to $\vec{r}_1$) plus the path integral along $C_2$ (from $\vec{r}_1$ to $\vec{r}_0$) equals zero. But this sum of path integrals is just the closed loop integral $\oint_C$ around the oriented curve $C=C_1-C_2$.
        \item Thus, equivalently,
        \begin{equation*}
            \int_C\vec{A}\cdot\dd\vec{r} = 0
        \end{equation*}
        for all $C$ containing $\vec{r}_0$ and $\vec{r}_1$.
        \item Lastly, note that we do not need to constrain the curves to $\vec{r}_0$ and $\vec{r}_1$ but can let them freely range over the whole space. Thus, we can check the closed loop integral over all loops $C$ in the space.
    \end{itemize}
    \item \textbf{Stokes' theorem}: The following integral equality, where $C$ is a closed curve bounding the curved surface $S$ and $\vec{A}$ is a vector field. \emph{Given by}
    \begin{equation*}
        \oint_C\vec{F}\cdot\dd\vec{r} = \iint_S(\vec{\nabla}\times\vec{A})\cdot\dd{\vec{S}}
    \end{equation*}
    \item How do we find $V$ from $F$?
    \begin{itemize}
        \item First, we need an integral theorem.
        \item Theorem: For all scalar functions $\phi:\mathbb{R}^3\to\mathbb{R}$ defining conservative forces and all points $\vec{r}_0,\vec{r}_1\in\mathbb{R}^3$, the \textbf{line integral}
        \begin{equation*}
            \int_{\vec{r}_0}^{\vec{r}_1}\vec{\nabla}\phi\cdot\dd\vec{r} = \phi(\vec{r}_1)-\phi(\vec{r}_0)
        \end{equation*}
        \item It follows that if $F=-\nabla V$, then
        \begin{equation*}
            V(\vec{r}_1)-V(\vec{r}_0) = -\int_{\vec{r}_0}^{\vec{r}_1}\vec{\nabla}V\cdot\dd\vec{r}
        \end{equation*}
    \end{itemize}
    \item We now move onto rotation.
    \begin{itemize}
        \item We describe rotation in polar coordinates.
        \item Let $\ell_r$ be the length in the radial direction, and let $\ell_\theta$ be the length in the angular direction.
        \item Then
        \begin{align*}
            \dd\ell_r &= \dd{r}&
            \dd\ell_\theta &= r\dd\theta
        \end{align*}
        where
        \begin{align*}
            \hat{r} &= \ihat\cos\theta+\jhat\sin\theta&
            \hat{\theta} &= -\ihat\sin\theta+\jhat\cos\theta
        \end{align*}
        \item Coordinate-wise, we have
        \begin{align*}
            x &= r\cos\theta&
            y &= r\sin\theta
        \end{align*}
        \item Velocity-wise, we have $\vec{v}=v_x\ihat+v_y\jhat$ where
        \begin{align*}
            v_x &= \dot{r}\cos\theta-r\dot{\theta}\sin\theta&
                v_y &= \dot{r}\sin\theta+r\dot{\theta}\cos\theta\\
            v_r &= \vec{v}\cdot\hat{r} = \dot{r} = \dv{\ell_r}{t}&
                v_\theta &= \vec{v}\cdot\hat{\theta} = r\dot{\theta} = \dv{\ell_\theta}{t}
        \end{align*}
    \end{itemize}
    \item The analogy of force under rotation is \textbf{torque}.
    \item \textbf{Torque}: A twisting force that tends to cause rotation, quantified as follows. \emph{Also known as} \textbf{moment of force}. \emph{Denoted by} $\bm{\vec{g}}$. \emph{Given by}
    \begin{equation*}
        \vec{G} = \vec{r}\times\vec{F}
    \end{equation*}
    \begin{itemize}
        \item Componentwise, we have
        \begin{align*}
            G_x &= yF_z-zF_y&
            G_y &= zF_x-xF_z&
            G_z &= xF_y-yF_x
        \end{align*}
        \item We also have $\norm{\vec{G}}=rF\sin\theta$.
    \end{itemize}
    \item Momentum under rotation: Angular momentum.
    \item \textbf{Angular momentum}: The quantity of rotation of a body, quantified as follows. \emph{Denoted by} $\bm{\vec{J}}$. \emph{Given by}
    \begin{equation*}
        \vec{J} = \vec{r}\times\vec{p}
        = m\vec{r}\times\vec{r}
    \end{equation*}
    \begin{itemize}
        \item Derivative:
        \begin{equation*}
            \dot{\vec{J}} = \vec{G}
        \end{equation*}
    \end{itemize}
    \item \textbf{Central force}: A force that flows toward or away from the origin, i.e., is in the $\hat{r}$ direction.
    \begin{itemize}
        \item Identify with $\vec{r}\times\vec{F}=0$.
    \end{itemize}
    \item Under central forces, angular momentum is conserved.
    \begin{itemize}
        \item We have
        \begin{equation*}
            \vec{J} = mr^2\dot{\theta}\hat{z}
        \end{equation*}
        \item Sweeping out equal areas (Kepler's 2nd law): We have
        \begin{align*}
            \dd{A} &= \frac{1}{2}r^2\dd{\theta} = \pi r^2\frac{\dd{\theta}}{2\pi}\\
            \dv{A}{t} &= \frac{1}{2}r^2\dot{\theta}
        \end{align*}
    \end{itemize}
\end{itemize}



\section{Introduction to Variational Calculus and the Lagrangian}
\begin{itemize}
    \item \marginnote{10/9:}Recap points from last time, then variational calculus (different form of mechanics that is more powerful than Newton's laws, called Lagrangian mechanics).
    \item One particle feeling external conservative forces.
    \item We'll revisit this later when we learn Hamiltonian mechanics.
    \item Suppose we have one particle in three dimensions.
    \begin{itemize}
        \item Newton tells us that we can get EOM by figuring out all the forces on each particle and setting the net force equal to the mass times acceleration.
        \item This is often written componentwise.
        \item For the special case of a conservative force (requirement is that the curl vanishes, $\vec{\nabla}\times\vec{F}=0$), we can find a scalar potential energy function $V$ such that $\vec{F}=-\vec{\nabla}V$.
        \item Each
        \begin{equation*}
            -\pdv{V}{x_i} = F_i = m\vec{\ddot{r}}_i = \dot{p}_i
        \end{equation*}
    \end{itemize}
    \item Intro to variational calculus.
    \begin{itemize}
        \item We're not responsible for doing variational calculations, themselves, but we will use the results.
    \end{itemize}
    \item The variational problem.
    \begin{itemize}
        \item Define a family of curves in the space $t\oplus x$ connecting two points $(t_0,x_0)$ and $(t_1,x_1)$.
        \item We have a \textbf{functional}
        \begin{equation*}
            \Phi = \int_{t_0}^{t_1}f(x(t),\dot{x}(t),t)\dd{t}
        \end{equation*}
        \item The problem: Find the path $x(t)$ that makes $\Phi$ into an extremum (i.e., minimum or maximum).
        \item Example: Find the curve that minimizes the distance between the two points.
    \end{itemize}
    \item \textbf{Functional}: A function of curves (as opposed to points or values).
    \item Solving such problems.
    \begin{itemize}
        \item We want to find a way to differentiate functionals like $\Phi$ with respect to curves.
        \item Let $x(t)$ be the curve for which $\Phi$ is minimal or maximal (aka extremal or \textbf{stationary}).
        \item Let $\eta(t)$ be any smooth function with $\eta(t_0)=\eta(t_1)=0$.
        \item Define $x(t,0)=x(t)$ and $x(t,\alpha)=x(t,0)+\alpha\eta(t)$.
        \item Now, we can write $\Phi$ as a function of $\alpha$!
        \begin{equation*}
            \Phi(\alpha) = \int_{t_0}^{t_1}f(x(t,\alpha),\dot{x}(t,\alpha),t)\dd{t}
        \end{equation*}
        \item For $x(t)$ to be an extremum, we need
        \begin{equation*}
            \eval{\pdv{\Phi}{\alpha}}_{\alpha=0} = 0
        \end{equation*}
        for all $\eta(t)$.
        \item Now we take
        \begin{align*}
            \pdv{\Phi}{\alpha} &= \pdv{\alpha}\int_{t_0}^{t_1}f(x,\dot{x},t)\dd{t}\\
            &= \int_{t_0}^{t_1}\pdv{f}{\alpha}(x,\dot{x},t)\dd{t}\\
            &= \int_{t_0}^{t_1}\left( \pdv{f}{x}\pdv{x}{\alpha}+\pdv{f}{\dot{x}}\pdv{\dot{x}}{\alpha} \right)\dd{t}
        \end{align*}
        \item But we have that
        \begin{align*}
            x(t,\alpha) &= x(t)+\alpha\eta(t)&
            \dot{x}(t,\alpha) &= \dot{x}(t)+\alpha\dot{\eta}(t)
        \end{align*}
        so
        \begin{align*}
            \pdv{x}{\alpha} &= \eta(t)&
            \pdv{\dot{x}}{\alpha} &= \dot{\eta}(t)
        \end{align*}
        \item Thus, continuing from the above,
        \begin{equation*}
            \pdv{\Phi}{\alpha} = \int_{t_0}^{t_1}\left( \pdv{f}{x}\eta(t)+\pdv{f}{\dot{x}}\pdv{\eta}{t} \right)\dd{t}
        \end{equation*}
        \item We now integrate by parts.
        \begin{equation*}
            \int_{t_0}^{t_1}\pdv{f}{\dot{x}}\dv{\eta}{t}\dd{t} = \pdv{f}{\dot{x}}[\eta(t_1)-\eta(t_0)]-\int_{t_0}^{t_1}\dv{t}(\pdv{f}{\dot{x}})\eta(t)\dd{t}
        \end{equation*}
        \item The first term after the equals sign goes to zero by the definition of $\eta$.
        \item Thus, continuing from the above,
        \begin{align*}
            \pdv{\Phi}{\alpha} &= \int_{t_0}^{t_1}\left( \pdv{f}{x}\eta(t)-\dv{t}(\pdv{f}{\dot{x}})\eta(t) \right)\dd{t}\\
            &= \int_{t_0}^{t_1}\left( \pdv{f}{x}-\dv{t}(\pdv{f}{\dot{x}}) \right)\eta(t)\dd{t}
        \end{align*}
        \item Thus, since we want $\pdv*{\Phi}{\alpha}|_{\alpha=0}=0$, our condition that $f$ must satisfy is
        \begin{equation*}
            \int_{t_0}^{t_1}\left( \pdv{f}{x}-\dv{t}(\pdv{f}{\dot{x}}) \right)\eta(t)\dd{t} = 0
        \end{equation*}
        for any $\eta(t)$.
        \item In particular, if this is to be zero for all $\eta(t)$, then we must have
        \begin{equation*}
            \pdv{f}{x}-\dv{t}(\pdv{f}{\dot{x}}) = 0
        \end{equation*}
        \item This is called an \textbf{Euler Equation} within mathematics, and an \textbf{Euler-Lagrange Equation} within physics.
    \end{itemize}
    \item Variational example: What shape of curve minimizes the distance between two points.
    \begin{itemize}
        \item In the plane, we all know that this is a straight line, and we will prove this now.
        \begin{itemize}
            \item Aside: The problem is more interesting when applied to curved surfaces, such as geodesics or the sphere (great circle routes).
        \end{itemize}
        \item Recall that $\dd{\ell}=\sqrt{\dd{t}^2+\dd{x}^2}=\dd{t}\sqrt{1+\dot{x}^2}$.
        \item We want to minimize the sum of these distances along the curve (arc length), i.e., we want to minimize
        \begin{equation*}
            \Phi = \int_{t_0}^{t_1}\dd{t}\sqrt{1+\dot{x}^2}
        \end{equation*}
        \item From here, we may define
        \begin{equation*}
            f(x,\dot{x},t) = \sqrt{1+\dot{x}^2}
        \end{equation*}
        for substitution into the Euler-Lagrange equation.
        \item Substituting, we obtain
        \begin{align*}
            \dv{t}(\pdv{f}{\dot{x}}) &= \pdv{f}{x}\\
            \dv{t}(\frac{1}{2}(1+\dot{x}^2)^{-1/2}(2\dot{x})) &= 0\\
            \dv{t}(\frac{\dot{x}}{\sqrt{1+\dot{x}^2}}) &= 0\\
            \frac{\dot{x}}{\sqrt{1+\dot{x}^2}} &= C
        \end{align*}
        \item If the whole final expression is constant, then it must be that $\dot{x}$ is constant. From here, we can recover $x(t)=ct+b$.
        \item Note that we have not proven that this is the minimum (it could be a maximum of $\Phi$!). But \emph{if} there is a minimum, it is this.
    \end{itemize}
    \item In 3D, we can consider an equation of the form $f(x_1,x_2,x_3,\dot{x}_1,\dot{x}_2,\dot{x}_3,t)$.
    \begin{itemize}
        \item Running this back through the procedure, we get an Euler-Lagrange equation for each component.
        \begin{equation*}
            \pdv{f}{x_i}-\dv{t}(\pdv{f}{\dot{x}_i}) = 0
        \end{equation*}
    \end{itemize}
    \item We want a variational form of Newton's laws.
    \begin{itemize}
        \item Compare the Euler-Lagrange equation and an analogous form of Newton's law.
        \begin{align*}
            \dv{t}(\pdv{f}{\dot{x}_i}) &= \pdv{f}{x_i}&
            \dv{t}(m\dot{x}_i) &= -\pdv{V}{x_i}
        \end{align*}
        \item Let
        \begin{equation*}
            f = T-V
            = \sum_i\frac{1}{2}m\dot{x}_i^2-V(\{x_i\})
        \end{equation*}
        where $V(\{x_i\})$ denotes $V(x_1,x_2,x_3)$.
    \end{itemize}
    \item \textbf{Lagrangian function}: The function defined as follows. \emph{Denoted by} $\bm{L}$. \emph{Given by}
    \begin{equation*}
        L = T-V
    \end{equation*}
    \item \textbf{Action}: The following integral. \emph{Also known as} \textbf{action integral}. \emph{Denoted by} $\bm{S}$, $\bm{I}$. \emph{Given by}
    \begin{equation*}
        S = \int_{t_0}^{t_1}L(x_i,\dot{x}_i,t)\dd{t}
    \end{equation*}
    \item \textbf{Least action principle}: Particle trajectories are those for which $S$ is extremal.
    \begin{itemize}
        \item Not always needed or necessary.
    \end{itemize}
    \item Procedure for finding equations of motion.
    \begin{enumerate}
        \item Write down your Lagrangian for the system.
        \item Use the componentwise Euler-Lagrange equations to find the EOMs.
    \end{enumerate}
    \item Why do this?
    \begin{enumerate}
        \item We can use any coordinate system to define $L$.
        \begin{itemize}
            \item It's often easier to change coordinates at the stage of scalar functions rather than later when you're dealing with multiple derivatives, vectors, etc.
        \end{itemize}
        \item Much easier to specify constraints.
        \begin{itemize}
            \item We can also use this formalism (as we'll see next time) to go backwards and see what the original forces are.
        \end{itemize}
        \item Symmetries and conservation laws are often more transparent in this formulation.
    \end{enumerate}
    \item Example.
    \begin{itemize}
        \item Suppose we have a bead that is constrained to move under gravity along a parabolic wire.
        \item Let the equation of the wire be $z=ax^2$.
        \item The wire exerts normal forces; it's hard to figure out what these are because the curvature of the wire is constantly changing.
        \item Write
        \begin{align*}
            T &= \frac{1}{2}m(\dot{x}^2+\dot{z}^2)&
            V &= mgz
        \end{align*}
        \item We also need $\dot{z}=2ax\dot{x}$.
        \item Thus,
        \begin{align*}
            L &= T-V\\
            &= \frac{1}{2}m(\dot{x}^2+(2ax\dot{x})^2)-mgax^2\\
            &= \frac{1}{2}m(\dot{x}^2+4a^2x^2\dot{x}^2)-mgax^2
        \end{align*}
        \item We can now find the equations of motion with the Euler-Lagrange equation.
        \begin{align*}
            \dv{t}(\pdv{L}{\dot{x}}) &= \pdv{L}{x}\\
            \dv{t}(m\dot{x}+4ma^2x^2\dot{x}) &= 4ma^2x\dot{x}^2-2mgax\\
            m\ddot{x}+8ma^2x\dot{x}^2+4ma^2x^2\ddot{x} &= 4ma^2x\dot{x}^2-2mgax\\
            % \ddot{x}+4a^2(x\dot{x}^2+x^2\ddot{x})+2gax &= 0\\
            \ddot{x}(1+4a^2x^2)+\dot{x}^2(4a^2x)+2gax &= 0
        \end{align*}
        \item This final expression is pretty complicated! It would have been very complicated (perhaps prohibitively so) to arrive here with kinematics.
    \end{itemize}
    \item Imagine now that this wire is rotating at constant angular velocity $\omega$.
    \begin{itemize}
        \item We can solve this in rotating coordinates just as easily!
        \item This time, take
        \begin{equation*}
            T = \frac{1}{2}m(v_r^2+v_\theta^2+v_z^2)
        \end{equation*}
        where
        \begin{align*}
            v_r &= \dot{r}&
            v_\theta &= r\dot{\theta} = r\omega&
            v_z &= \dot{z}
        \end{align*}
    \end{itemize}
\end{itemize}



\section{Office Hours (Jerison)}
\begin{itemize}
    \item Phase offsets in the driven harmonic oscillator.
\end{itemize}



\section{Introduction to the Lagrangian: Examples and the Free Particle}
\begin{itemize}
    \item \marginnote{10/11:}Now that we have the Lagrangian, pretty soon, we will be able to prove why the kinetic energy has the form $mv^2/2$.
    \begin{itemize}
        \item We won't be required to reproduce this derivation, though.
    \end{itemize}
    \item Announcements.
    \begin{itemize}
        \item Midterm will be on a Wednesday during our section.
        \item No pset due Friday of midterm week; a smaller one will be due the following Monday.
        \item There will be another small one due that Friday.
        \item Some textbook chapters have been posted on Canvas with more background on the Lagrangian; they contain info that may be helpful for our homework.
    \end{itemize}
    \item Today: Pendulum and generalized coordinates.
    \item Next time: Lagrange multipliers and constraints; start central, conservative forces.
    \item Recap.
    \begin{itemize}
        \item $L=T-V=T(\{q_i\})-V(\{q_i\})$.
        \begin{itemize}
            \item We use $q$ instead of $x$ because these coordinates don't have to be positions!
        \end{itemize}
        \item Lagrange's equations of motion:
        \begin{equation*}
            \dv{t}(\pdv{L}{\dot{q}_i}) = \pdv{L}{q_i}
        \end{equation*}
        for $i=1,2,3$ for an unconstrained particle.
        \item Why use Lagrangian mechanics?
        \begin{enumerate}
            \item Constraints are easy to incorporate, e.g., bead on a quadratic wire.
            \item We can choose any generalized coordinates in which to express $T,V$.
            \item Symmetries are often more transparent.
        \end{enumerate}
        \item We talked about 1 last time; we'll talk about 2-3 today.
    \end{itemize}
    \item Generalized coordinates.
    \item Example (use of different coordinates): Simple pendulum.
    \begin{figure}[h!]
        \centering
        \begin{tikzpicture}
            \footnotesize
            \draw [dashed]
                (0,2.5) -- (0,-1) coordinate (A)
                (-1.5,2) -- (1.5,2)
            ;
    
            \draw [->] (1,0) -- ++(270:0.8) node[above left]{$mg$};
            \draw [->] (1,0) -- ++(300:0.6) node[below right=-1pt]{$\hat{r}$};
            \draw [->] (1,0) -- ++(30:0.6) node[above right=-1pt]{$\hat{\theta}$};
    
            \draw [gry,thick] (0,2) coordinate (B) node[circle,fill,black,inner sep=1pt,label={[above right=-1pt,black]$(0,0)$}]{} -- node[above right=-1pt,black]{$\ell$} (1,0) coordinate (C) node[circle,fill,grx,inner sep=2pt,label={[below right=1pt,black]m}]{} pic [draw,black,thin,angle radius=7mm,angle eccentricity=1.2,pic text={$\theta$}] {angle=A--B--C};
        \end{tikzpicture}
        \caption{Simple pendulum.}
        \label{fig:simplePendulum}
    \end{figure}
    \begin{itemize}
        \item A rigid, massless rod of length $\ell$ pinned at the top and connected to a bob of mass $m$ that makes angle $\theta$ with the vertical.
        \item EOM with Newton's laws.
        \begin{itemize}
            \item $\vec{F}=m\ddot{\vec{r}}$.
            \item This system has a plane polar symmetry, so we want an expression in plane polar coordinates.
            \item In particular, in these coordinates, $\ddot{\vec{r}}=(\ddot{r}-r\dot{\theta}^2)\hat{r}+(r\ddot{\theta}+2\dot{r}\dot{\theta})\hat{\theta}$.
            \item Using this acceleration vector, the EOMs are as follows:
            \begin{align*}
                F_{T,\text{rod}}+mg\cos\theta &= F_r = m(\ddot{r}-r\dot{\theta}^2)&
                -mg\sin\theta &= F_\theta = m(r\ddot{\theta}+2\dot{r}\dot{\theta})
            \end{align*}
            \item We know by inspection of Figure \ref{fig:simplePendulum} that $\ddot{r}=\dot{r}=0$ and $r=\ell$, so the above becomes
            \begin{align*}
                F_r &= -m\ell\dot{\theta}^2&
                F_\theta &= m\ell\ddot{\theta}
            \end{align*}
            \item Since the radial forces are balanced, we only need to worry about the angular ones going forward. In particular, by transitivity, the final EOM is
            \begin{align*}
                m\ell\ddot{\theta} &= -mg\sin\theta\\
                \ddot{\theta} &= -\frac{g}{\ell}\sin\theta
            \end{align*}
            as desired.
        \end{itemize}
        \item EOM with the Lagrangian.
        \begin{itemize}
            \item $L=T-V$, where
            \begin{align*}
                T &= \frac{1}{2}m(v_r^2+v_\theta^2)
                    = \frac{1}{2}m(\dot{r}^2+r^2\dot{\theta}^2)
                    = \frac{1}{2}m\ell^2\dot{\theta}^2&
                V &= -mg\ell\cos\theta
            \end{align*}
            \item Note that we can define the potential energy function as such instead of as $mg(\ell-\ell\cos\theta)$ since we may choose the zero of potential energy to be $mg\ell$!
            \item Thus, the complete Lagrangian is
            \begin{equation*}
                L = \frac{1}{2}m\ell^2\dot{\theta}^2+mg\ell\cos\theta
            \end{equation*}
            \item With only one of the two coordinates remaining (that is, $\theta$ not $r$), we only need an Euler-Lagrange equation in this one component to find the complete EOM:
            \begin{align*}
                \dv{t}(\pdv{L}{\dot{\theta}}) &= \pdv{L}{\theta}\\
                \dv{t}(m\ell^2\dot{\theta}) &= -mg\ell\sin\theta\\
                m\ell^2\ddot{\theta} &= -mg\ell\sin\theta\\
                \ddot{\theta} &= -\frac{g}{\ell}\sin\theta
            \end{align*}
        \end{itemize}
        \item Thus, we got the same result without having to derive the complicated transformation between Cartesian and polar coordinates!
    \end{itemize}
    \item The $\theta$ above is the first example we've seen thus far of a \textbf{generalized coordinate} (we'll see further examples later).
    \item $\pdv*{L}{\dot{q}_i}$ is often referred to as a \textbf{generalized momentum} and $\pdv*{L}{q_i}$ is often referred to as a \textbf{generalized force}.
    \begin{itemize}
        \item If we're in Cartesian coordinates, these things are \emph{actual} momenta and forces since\dots
        \begin{align*}
            \pdv{L}{\dot{x}_i} &= m\dot{x}_i
                = p_i&
            \pdv{L}{x_i} &= -\dv{V}{x_i}
                = F_i
        \end{align*}
        \item In the case of the pendulum, recall that we have
        \begin{align*}
            \pdv{L}{\dot{\theta}} &= m\ell^2\dot{\theta}&
            \pdv{L}{\theta} &= -mg\ell\sin\theta
        \end{align*}
        \begin{itemize}
            \item The left one can be recognized as the angular momentum $\vec{r}\times\vec{p}$.
            \item The right one can be recognized as the torque $\vec{r}\times\vec{F}$.
        \end{itemize}
        \item If $L$ is independent of $q_i$ for some $q_i$, then $\pdv*{L}{\dot{q}_i}$ is constant in time and hence we have a conserved force (in some sense).
        \begin{itemize}
            \item In particular, if $L$ is independent of some $q_i$, then $0=\pdv*{L}{q_i}=\dv*{t}(\pdv*{L}{\dot{q}_i})$, so $\pdv*{L}{\dot{q}_i}$ is constant in time.
        \end{itemize}
    \end{itemize}
    \item One last thing to keep in mind about coordinate systems.
    \item Cylindrical and spherical coordinates.
    \begin{itemize}
        \item Cylindrical:
        \begin{align*}
            x &= r\cos\phi&
            y &= r\sin\phi&
            z &= z
        \end{align*}
        \item Spherical:
        \begin{align*}
            x &= r\sin\theta\cos\phi&
            y &= r\sin\theta\sin\phi&
            z &= r\cos\theta
        \end{align*}
        \begin{itemize}
            \item In this case, $\theta$ comes down from the vertical, and $\phi$ sweeps around the $xy$-plane.
            \item Thus, $\theta=[0,\pi]$ and $\phi=[0,2\pi]$.
        \end{itemize}
    \end{itemize}
    \item Moving on: Symmetries.
    \item Why is $T=mv^2/2$? Let's look at the Lagrangian of a \textbf{free particle}.
    \begin{itemize}
        \item No external forces means that $V=0$ and thus $L=T-0=T$.
        \item If we believe Galileo's relativity principle, then the EOMs must be the same in any inertial reference frame.
        \item This is \emph{almost} the same as saying that the Lagrangian must be the same in any inertial reference frame, but not quite!
        \item In particular, if $L'=L+\dv*{t}f(q_i,t)$, then $L'$ and $L$ give the same EOMs, that is, they are equivalent.
        \begin{itemize}
            \item Note: We have just defined a notion of \emph{equivalence} for Lagrangians!
        \end{itemize}
        \item To see that they do give the same EOMs, start by expanding the definition of $L'$ above.
        \begin{equation*}
            L' = L+\sum_i\pdv{f}{q_i}\dot{q}_i+\pdv{f}{t}
        \end{equation*}
        \item Next, observe that
        \begin{align*}
            \pdv{L'}{\dot{q}_i} &= \pdv{L}{\dot{q}_i}+\pdv{f}{q_i}&
            \pdv{L'}{q_i} &= \pdv{L}{q_i}+\dv{t}(\pdv{f}{q_i})
        \end{align*}
        \begin{itemize}
            \item For the left equation above, we use the facts that $L$ may have a $\dot{q}_i$ term, $\pdv*{f}{q_i}\dot{q_i}$ does have a $\dot{q}_i$, and every other term does not contain a $\dot{q}_i$. This allows us to compute the partial derivative as written.
            \item For the right equation above, note that the partial and total derivatives $\pdv*{q_i}$ and $\dv*{t}$ do not commute in general. However, in this case, we know that
            \begin{equation*}
                \pdv{q_i}(\sum_j\pdv{f}{q_j}\dot{q}_j+\pdv{f}{t}) = \sum_i\dot{q}_j\cdot\pdv{j}\pdv{f}{q_i}+\pdv{t}\pdv{f}{q_i}
                = \dd{t}(\pdv{f}{q_i})
            \end{equation*}
            But how come $\pdv{q_i}\pdv{f}{q_j}\dot{q}_j=\dot{q}_j\cdot\pdv{j}\pdv{f}{q_i}$?? How do we know that $\dot{q}_j$ does not depend on $q_i$?
        \end{itemize}
        \item Last, it follows that the EOMs from $L'$ are
        \begin{align*}
            \dv{t}(\pdv{L}{\dot{q}_i}+\pdv{f}{q_i}) &= \pdv{L}{q_i}+\dv{t}\pdv{f}{q_i}\\
            \dv{t}(\pdv{L}{\dot{q}_i}) &= \pdv{L}{q_i}
        \end{align*}
        i.e., are the same as those from $L$, as desired.
    \end{itemize}
    \item \textbf{Free particle}: A particle moving with some velocity $v$ in a reference frame $K$ under the influence of no external forces.
    \item A teaser for next time.
    \begin{itemize}
        \item Suppose we have a free particle moving with velocity $\vec{v}$ so that $L=T$.
        \item What form can this take such that $L$ either doesn't change or changes by $\dv*{t}f(q_i,t)$ when we perform a Galilean transformation (that is, go to a new inertial reference frame)?
        \item What we'll see next time is that this constrains $T$ to be $\propto v^2$.
    \end{itemize}
\end{itemize}




\end{document}