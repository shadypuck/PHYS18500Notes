\documentclass[../notes.tex]{subfiles}

\pagestyle{main}
\renewcommand{\chaptermark}[1]{\markboth{\chaptername\ \thechapter\ (#1)}{}}
\setcounter{chapter}{8}

\begin{document}




\chapter{Rigid Body Motion}
\section{Introduction; Rotation About an Axis; Moments of Inertia}
\begin{itemize}
    \item \marginnote{11/3:}Announcements.
    \begin{itemize}
        \item We will now have \emph{seven} problem sets instead of \emph{eight}.
        \begin{itemize}
            \item Each problem set is now worth more (PSets still amount to 40\% of our grade).
            \item There will still be one makeup PSet at the end of the quarter.
        \end{itemize}
        \item PSet 5 is due next Friday.
    \end{itemize}
    \item Recap: Many-body motion.
    \begin{itemize}
        \item It's useful to introduce the center of mass coordinate, $\vec{R}=1/M\cdot\sum_\alpha m_\alpha\vec{r}_\alpha$, where $M=\sum_\alpha m_\alpha$.
        \item In the CM frame, $\vec{R}{\,}^*=0$ and $\vec{r}_\alpha=\vec{R}+\vec{r}_\alpha{}^*$.
        \begin{itemize}
            \item We also have $\vec{P}{\,}^*=0$, $T^*=\sum_\alpha m_\alpha(\dot{\vec{r}}_\alpha{}^*)^2/2$, and $\vec{J}{\,}^*=\sum_\alpha m_\alpha\vec{r}_\alpha{}^*\times\dot{\vec{r}}_\alpha{}^*$.
        \end{itemize}
        \item Then, going back into the lab frame, we have $\vec{P}=M\cdot\dot{\vec{R}}$, $T=M\dot{\vec{R}}^2/2+T^*$, and $\vec{J}=M\vec{R}\times\dot{\vec{R}}+\vec{J}{\,}^*$.
        \item One more note before we move onto rigid bodies: Suppose we're interested in the work, i.e., the rate of change of $T$ in the system.
        \begin{itemize}
            \item Recall that $m\ddot{\vec{r}}_\alpha=\sum_\beta\vec{F}_{\alpha\beta}+\vec{F}_\alpha$.
            \item Thus,
            \begin{equation*}
                \dot{T} = \sum_\alpha m_\alpha\dot{\vec{r}}_\alpha\cdot\ddot{\vec{r}}_\alpha
                = \sum_\alpha\sum_\beta\dot{\vec{r}}_\alpha\cdot\vec{F}_{\alpha\beta}+\sum_\alpha\dot{\vec{r}}_\alpha\cdot\vec{F}_\alpha
            \end{equation*}
            \item Note: Even letting $\vec{r}_{\alpha\beta}=\vec{r}_\alpha-\vec{r}_\beta$ and using $\vec{F}_{\alpha\beta}=-\vec{F}_{\beta\alpha}$, the left term above is often not equal to zero, i.e., there is no reason for it to vanish as in previous cases.
            \begin{itemize}
                \item This is not surprising, as it makes sense that the internal potential energy of the system would change in many cases.
            \end{itemize}
            \item However, if the $\vec{F}_{\alpha\beta}$ are conservative, then
            \begin{equation*}
                \dot{\vec{r}}_{\alpha\beta}\cdot\vec{F}_{\alpha\beta} = -\dv{t}V_{\text{int},\alpha\beta}
            \end{equation*}
            is the rate of internal forces doing work.
            \item Consequence: The rate of change of the kinetic plus internal potential energy is equal to the rate at which the external forces do work. That is,
            \begin{equation*}
                \dv{t}(T+V_\text{int}) = \sum_\alpha\dot{\vec{r}}_\alpha\cdot\vec{F}_\alpha
            \end{equation*}
            \item Additionally, we can find the rate of change of energy relative to the center of mass. In particular, in the CM frame, we have
            \begin{equation*}
                \dv{t}(\frac{1}{2}M\dot{\vec{R}}^2) = M\dot{\vec{R}}\cdot\ddot{\vec{R}}
                = \dot{\vec{R}}\cdot\sum_\alpha\vec{F}_\alpha
            \end{equation*}
            \item Subtracting the above equation from the one above it, we obtain
            \begin{align*}
                \dv{t}(T^*+V_\text{int}) &= \dv{t}(T-\frac{1}{2}M\dot{\vec{R}}^2+V_\text{int})\\
                &= \sum_\alpha\dot{\vec{r}}_\alpha\cdot\vec{F}_\alpha-\dot{\vec{R}}\cdot\sum_\alpha\vec{F}_\alpha\\
                &= \sum_\alpha\dot{\vec{r}}_\alpha{}^*\cdot\vec{F}_\alpha
            \end{align*}
            \item Note that in the leftmost term above, we are differentiating the total energy in the CM frame with respect to time. But since the time rate of change of energy is power, what we have expressed is the power.
        \end{itemize}
        \item Comparing this to $\dot{\vec{J}}{\,}^*=\sum_\alpha\vec{r}_\alpha{}^*\times\vec{F}_\alpha$, we see that we have a similar structure.
    \end{itemize}
    \item Today.
    \begin{itemize}
        \item Rigid bodies (a special case of many-body motion in which the particles are fixed relative to each other).
        \item Motion about an axis.
    \end{itemize}
    \item Today, we will primarily focus on rotation about an axis.
    \item The setup is as follows.
    \begin{itemize}
        \item We choose rotation to be in the $\hat{z}$ direction. We choose a shape (whatever we want), and it is rotating about this $\hat{z}$ axis.
        \item If is often useful to use cylindrical coordinates $(\rho,\phi,z)$. here because of the axial symmetry.
        \begin{itemize}
            \item Conversions: $x=\rho\cos\phi$, $y=\rho\sin\phi$, and $z=z$.
            \item Note that $\vec{r}=z\hat{z}+\rho\hat{\rho}$, much like in Figure \ref{fig:Vrotation}.
        \end{itemize}
        \item Recall that $\dv*{\vec{r}}{t}=\vec{\omega}\times\vec{r}=\dot{\vec{r}}$.
        \item We can now calculate our $\vec{J}$. It is equal to
        \begin{equation*}
            \vec{J} = \sum_\alpha m_\alpha\vec{r}_\alpha\times\dot{\vec{r}}_\alpha
            = \sum_\alpha m_\alpha\vec{r}_\alpha\times(\vec{\omega}\times\vec{r}_\alpha)
        \end{equation*}
        \item Expanding out the cross product, we obtain
        \begin{equation*}
            \begin{pmatrix}
                \hat{\rho} & \hat{\phi} & \hat{z}\\
                0 & 0 & \omega\\
                \rho & 0 & z\\
            \end{pmatrix}
            = \omega\rho\hat{\phi}
        \end{equation*}
        \item Expanding out our second cross product, we obtain
        \begin{equation*}
            \begin{pmatrix}
                \hat{\rho} & \hat{\phi} & \hat{z}\\
                \rho & 0 & z\\
                0 & \rho\omega & 0\\
            \end{pmatrix}
            = -z\rho\omega\hat{\rho}+\rho^2\omega\hat{z}
        \end{equation*}
    \end{itemize}
    \item Thus, we have that
    \begin{align*}
        \vec{J} &= \sum_\alpha m_\alpha(\rho_\alpha^2\omega\hat{z}-z_\alpha\omega\rho_\alpha\hat{\rho})\\
        &= \sum_\alpha m_\alpha[\rho_\alpha^2\omega\hat{z}-z_\alpha\omega(\rho_\alpha\cos\phi\hat{x}+\rho_\alpha\sin\phi\hat{y})]\\
        &= \omega\left( \sum_\alpha m_\alpha\rho_\alpha^2 \right)\hat{z}-\left( \omega\sum_\alpha m_\alpha z_\alpha x_\alpha \right)\hat{x}-\left( \omega\sum_\alpha m_\alpha z_\alpha y_\alpha \right)\hat{y}
    \end{align*}
    \begin{itemize}
        \item We can get this into a more familiar term via \textbf{moments of inertia}.
    \end{itemize}
    \item \textbf{Moment of inertia} (about the $z$-axis). \emph{Denoted by} $\bm{I_{zz}}$. \emph{Given by}
    \begin{equation*}
        I_{zz} = \sum_\alpha m_\alpha\rho_\alpha^2
        = \sum_\alpha m_\alpha(x_\alpha^2+y_\alpha^2)
    \end{equation*}
    \begin{itemize}
        \item In general, these are \textbf{second} moments about an axis. This just reflects the fact that the axial distance is \emph{squred}.
    \end{itemize}
    \item \textbf{Products of inertia}. Examples.
    \begin{itemize}
        \item $I_{xz}=-\sum_\alpha m_\alpha x_\alpha z_\alpha$.
        \item $I_{yz}=-\sum_\alpha m_\alpha y_\alpha z_\alpha$.
    \end{itemize}
    \item It follows from these definitions that, for $\vec{\omega}=\omega\hat{z}$, we have
    \begin{align*}
        J_z &= I_{zz}\omega&
        J_y &= I_{yz}\omega&
        J_x &= I_{xz}\omega
    \end{align*}
    \begin{itemize}
        \item Note that if $\vec{\omega}=\omega\hat{x}$, we have
    \end{itemize}
    \begin{align*}
        J_z &= I_{zx}\omega&
        J_y &= I_{yx}\omega&
        J_x &= I_{xx}\omega
    \end{align*}
    \item If we have $\vec{\omega}=\omega_x\hat{x}+\omega_y\hat{y}+\omega_z\hat{z}$, then the contributions to angular momentum add via
    \begin{equation*}
        \begin{bmatrix}
            J_x\\
            J_y\\
            J_z\\
        \end{bmatrix}
        = \underbrace{
            \begin{bmatrix}
                I_{xx} & I_{xy} & I_{xz}\\
                I_{yx} & I_{yy} & I_{yz}\\
                I_{zx} & I_{zy} & I_{zz}\\
            \end{bmatrix}
        }_I
        \begin{bmatrix}
            \omega_x\\
            \omega_y\\
            \omega_z\\
        \end{bmatrix}
    \end{equation*}
    \begin{itemize}
        \item $I$ is the \textbf{moment of inertia tensor}.
        \item It follows that, for example,
        \begin{equation*}
            J_x = I_{xx}\omega_x+I_{xy}\omega_y+I_{xz}\omega_z
        \end{equation*}
    \end{itemize}
    \item What's a tensor?
    \begin{itemize}
        \item It's like a matrix with a tiny bit more structure.
        \item For now, think of it as a $3\times 3$ matrix, and we'll talk more about it a little bit more next time.
    \end{itemize}
    \item Consider again $\vec{\omega}=\omega\hat{z}$.
    \begin{itemize}
        \item Then
        \begin{equation*}
            J_z = I_{zz}\omega
            = \sum_\alpha m_\alpha\rho_\alpha^2\omega
        \end{equation*}
        \item It follows that
        \begin{equation*}
            \dot{\vec{J}} = \sum_\alpha\vec{r}_\alpha\times\vec{F}_\alpha
        \end{equation*}
        \item Computing the cross product, we have
        \begin{equation*}
            \begin{pmatrix}
                \hat{\rho} & \hat{\phi} & \hat{z}\\
                \rho_\alpha & 0 & z_\alpha\\
                F_\rho & F_\phi & F_z\\
            \end{pmatrix}
            = -F_\phi z_\alpha\hat{\rho}+\rho_\alpha F_\phi\hat{z}
        \end{equation*}
        \item Then
        \begin{equation*}
            \dot{J}_z = I_{zz}\dot{\omega}
            = \sum_\alpha\rho_\alpha F_\phi
        \end{equation*}
        \begin{itemize}
            \item This is the equation of motion for rigid bodies.
            \item It gives $\omega(t)$ in terms of force $F_\phi$.
        \end{itemize}
    \end{itemize}
    \item Example: Equilibrium.
    \begin{figure}[h!]
        \centering
        \includegraphics[width=0.22\linewidth]{../ExtFiles/rectangularLamina.png}
        \caption{The rectangular lamina.}
        \label{fig:rectangularLamina}
    \end{figure}
    \begin{itemize}
        \item The \textbf{rectangular lamina}.
        \item We're pulling on two corners, and if it's in equilibrium, the thing is not rotating. This means that
        \begin{align*}
            bF-aMg &= 0\\
            F &= \frac{a}{b}Mg
        \end{align*}
    \end{itemize}
    \item Kinetic energy.
    \begin{itemize}
        \item We have that
        \begin{equation*}
            T = \sum_\alpha\frac{1}{2}m_\alpha(\rho_\alpha\omega)^2
            = \frac{1}{2}I\omega^2
        \end{equation*}
        \item It follows that the time rate of change of the kinetic energy is
        \begin{equation*}
            \dot{T} = I\omega\dot{\omega}
            = \sum_\alpha\omega\rho_\alpha F_\phi
            = \sum_\alpha(\rho\dot{\phi})F_\phi
            = \sum_\alpha\dot{\vec{r}}_\alpha\cdot\vec{F}_\alpha
        \end{equation*}
        \item Thus, in this case, the internal forces do no work (which in some sense makes sense for a rigid body).
        \item Thus, the KE is just related to these external forces as shown above.
    \end{itemize}
    \item We'll talk about pivot points next time.
\end{itemize}



\section{Euler's Angles; Freely Rotating Symmetric Body}
\begin{itemize}
    \item \marginnote{11/6:}Announcements.
    \begin{itemize}
        \item Our exams are graded; we can pick them up after class.
        \begin{itemize}
            \item High: 96\%.
            \item Median: 71\%.
        \end{itemize}
        \item Our course grades will be curved.
        \begin{itemize}
            \item $\text{A}^-$/$\text{B}^+$ cutoff is likely 83\%.
            \item $\text{B}^-$/$\text{C}^+$ cutoff is likely 60\%.
        \end{itemize}
        \item Office hours are back in her office today.
        \item Where we're going.
        \begin{itemize}
            \item Next week: Hamiltonians and conservation laws.
            \item Then Thanksgiving.
            \item Then a bit of dynamical systems.
        \end{itemize}
    \end{itemize}
    \item Recap.
    \begin{itemize}
        \item Rigid bodies --- rotation about a fixed axis.
        \item Moments and products of inertia.
        \begin{itemize}
            \item What is a tensor?
        \end{itemize}
    \end{itemize}
    \item Addressing a question from last time: Why do we call $T^*+V_\text{int}$ the "total energy" in the CM frame?
    \begin{itemize}
        \item It's tautological: This is the only possible definition of "total energy" in the CM frame.
        \item More specifically, recall that $\dv*{t}(T+V_\text{int})=\sum_\alpha\dot{\vec{r}}_\alpha\cdot\vec{F}_\alpha$ and $\dv*{t}(T^*+V_\text{int})=\sum_\alpha\dot{\vec{r}}_\alpha{}^*\cdot\vec{F}_\alpha$.
        \begin{itemize}
            \item If the $\vec{F}_\alpha$ are \emph{conservative}, then we can define $V_\text{ext}$ via
            \begin{equation*}
                -\dv{t}(V_\text{ext}(\{\vec{r}_\alpha\})) = -\sum_{\alpha,i}\pdv{V_\text{ext}}{r_{\alpha i}}\dv{r_{\alpha i}}{t}
                = -\sum_\alpha\dot{\vec{r}}_\alpha\cdot\vec{F}_\alpha
            \end{equation*}
            \item Plugging the above into the expression for $\dv*{t}(T+V_\text{int})$ given above yields
            \begin{equation*}
                \dv{t}(T+V_\text{int}+V_\text{ext}) = 0
            \end{equation*}
            \item But this is exactly the condition we expect for \emph{conservative} external forces.
        \end{itemize}
        \item Visualizing the system also helps make this definition of total energy more clear.
        \begin{itemize}
            \item Recall that the system is like a bunch of particles connected by springs, all of which are connected to some external potential like gravity.
            \item When we talk about the "total energy" in the CM frame, we're essentially just "diagonalizing" the system between external and internal forces.
        \end{itemize}
    \end{itemize}
    \item Back to rigid bodies now.
    \item Rigid body motion is completely specified by the following two equations of motion.
    \begin{enumerate}
        \item $\dot{\vec{P}}=M\ddot{\vec{R}}=\sum_\alpha\vec{F}_\alpha$.
        \begin{itemize}
            \item Looks like a particle of mass $M$ at the CM.
        \end{itemize}
        \item $\dot{\vec{J}}=\sum_\alpha\vec{r}_\alpha\times\vec{F}_\alpha$.
    \end{enumerate}
    \item Recap.
    \begin{itemize}
        \item Last time, we found that there's a huge simplification we can make because all the particles in a rigid body are locked together.
        \begin{itemize}
            \item The simplification is that $\vec{J}=\overleftrightarrow{I}\vec{\omega}$, where $\overleftrightarrow{I}$ is the moment of inertia tensor from last time.
            \item Jerison writes out the matrix formula all over again.
            \item Point to emphasize: $\overleftrightarrow{I}$ is an intrinsic property of the rigid body, and it plays the role of mass.
            \item If we have a continuous object, the sums over indices $\alpha$ turn into an integral! Recall this from prior courses.
            \item Compare to $\vec{P}=M\dot{\vec{R}}$ to see that there is a similar structure in the above equation.
        \end{itemize}
        \item Special case: Rotation about a fixed axis.
        \begin{itemize}
            \item We're headed toward the \textbf{compound pendulum}.
            \item For such a problem, we use cylindrical coordinates.
            \begin{itemize}
                \item Jerison redefines the coordinate conversions.
            \end{itemize}
            \item We take $\vec{\omega}$ to lie in the $\khat$ direction via $\vec{\omega}=\omega\khat$.
            \item The moment we're most concerned with is $I_{zz}$, defined as previously. Differentiating gets us to $J_z=I_{zz}\omega_z$ and $\dot{J}_z=I_{zz}\dot{\omega}$.
            \item From here, we can define the kinetic energy
            \begin{equation*}
                T = \sum_\alpha\frac{1}{2}m_\alpha\dot{\vec{r}}_\alpha{}^2
                = \sum_\alpha\frac{1}{2}m_\alpha(\rho_\alpha\omega)^2
                = \frac{1}{2}I_{zz}\omega^2
            \end{equation*}
            where we recall that $\dot{\vec{r}}_\alpha=\vec{\omega}\times\vec{r}_\alpha=\rho_\alpha\omega\,\hat{\phi}$.
        \end{itemize}
    \end{itemize}
    \item The EOMs for this system are given by $\dot{\vec{J}}=\sum_\alpha\vec{r}_\alpha\times\vec{F}_\alpha$.
    \begin{itemize}
        \item We're mostly interested in the $z$ component, i.e., $\dot{J}_z=\sum_\alpha\rho_\alpha F_\phi$.
        \item Sometimes, it can be useful to separate out the forces into axial forces and other forces via
        \begin{equation*}
            \dot{\vec{P}} = M\ddot{\vec{R}}
            = \vec{Q}+\sum_\alpha\vec{F}_\alpha
        \end{equation*}
        \item To make calculations, it will additionally be useful to have the following expression. For a rotating body, $\dot{\vec{R}}$ is given via $\dot{\vec{R}}=\vec{\omega}\times\vec{R}$ and $\ddot{\vec{R}}=\dot{\vec{\omega}}\times\vec{R}+\vec{\omega}\times\dot{\vec{R}}=\dot{\vec{\omega}}\times\vec{R}+\vec{\omega}\times(\vec{\omega}\times\vec{R})$.
    \end{itemize}
    \item This is true in general; if we specialize to our case of rotation about an axis\dots
    \begin{itemize}
        \item We first choose coordinates such that $z_\text{cm}=0$.
        \item Since this is rotation about an axis, the above equation simplifies to
        \begin{equation*}
            \ddot{\vec{R}} = R\dot{\omega}\,\hat{\phi}-\omega^2R\,\hat{\phi}
            = R\ddot{\phi}\,\hat{\phi}-\dot{\phi}^2R\,\hat{\rho}
        \end{equation*}
        \item In the right term above, the left term is tangential acceleration and the right term is centripetal acceleration.
    \end{itemize}
    \item Example: Compound pendulum.
    \begin{figure}[h!]
        \centering
        \begin{tikzpicture}
            \small
            \draw [stealth-stealth] (2,0) node[right]{$y$} -- (0,0) coordinate (o) -- (0,-2) coordinate (x) node[below]{$x$};
            \fill circle (1.5pt);
    
            \footnotesize
            \coordinate (c) at (-50:0.8);
            \draw [pux,thick,rotate around={-50:(c)}] (c) ellipse (1.5cm and 1cm);
            \draw [->] (0,0) -- node[right,yshift=1mm]{$\vec{R}$} (c);
            \draw [->] (c) -- node[right]{$\vec{F}$} ++(0,-1);
            \pic [draw,angle radius=4mm,angle eccentricity=1.4,pic text={$\phi$}] {angle=x--o--c};
        \end{tikzpicture}
        \caption{Compound pendulum.}
        \label{fig:compoundPendulum}
    \end{figure}
    \begin{itemize}
        \item We want to look at the force on the pivot.
        \item We define a new coordinate system as in Figure \ref{fig:compoundPendulum}. Explicitly, $\hat{x}$ points straight downwards and $\hat{y}$ points straight rightwards.
        \item We put our pendulum's center of mass such that it rotates through angle $\phi$.
        \item At this point, we have
        \begin{align*}
            T &= \frac{1}{2}I_{zz}\dot{\phi}^2&
            V &= M\vec{g}\cdot\vec{R} = -MgR\cos\phi
        \end{align*}
        \item Thus, our Lagrangian is
        \begin{equation*}
            L = T-V = \frac{1}{2}I_{zz}\dot{\phi}^2+MgR\cos\phi
        \end{equation*}
        \item It follows that our EOM is
        \begin{align*}
            I\ddot{\phi} &= -MgR\sin\phi\\
            \ddot{\phi} &= -\frac{MgR}{I}\sin\phi\\
            &= -\frac{g}{\ell}\sin\phi
        \end{align*}
        where $\ell=I/MR$.
        \begin{itemize}
            \item $\ell$ defines the \textbf{equivalent simple pendulum}.
        \end{itemize}
        \item From here, we can solve for the force on the pivot as a function of $\phi$ (we could also go through $\phi(t)$, and solve for $F(t)$ if we desired).
        \begin{itemize}
            \item We start with the conservation of energy
            \begin{equation*}
                \frac{1}{2}I\dot{\phi}^2-MgR\cos\phi = E
            \end{equation*}
            \item It follows that
            \begin{equation*}
                \dot{\phi}^2 = \frac{E+MgR\cos\phi}{I/2}
                = \frac{2E}{Mr\ell}+\frac{2g}{\ell}\cos\phi
            \end{equation*}
            \item We want to solve for $\vec{Q}$ from $M\ddot{\vec{R}}=\vec{Q}+\sum_\alpha\vec{F}_\alpha$.
            \item Here, the only relevant external force is our gravitational force $Mg\cos\phi\,\hat{\rho}-Mg\sin\phi\,\hat{\phi}$.
            \item We also found previously that $\ddot{\vec{R}}=R\ddot{\phi}\,\hat{\phi}-\dot{\phi}^2R\,\hat{\rho}$. Thus,
            \begin{equation*}
                MR\ddot{\phi}\,\hat{\phi}-MR\dot{\phi}^2\,\hat{\rho} = \vec{Q}+Mg\cos\phi\,\hat{\rho}-Mg\sin\phi\,\hat{\phi}
            \end{equation*}
            \item Splitting this vector equation into scalar equations, we obtain
            \begin{align*}
                Q_\rho &= -MR\dot{\phi}^2-Mg\cos\phi&
                Q_z &= 0&
                Q_\phi &= MR\ddot{\phi}+Mg\sin\phi
            \end{align*}
            \item Substituting from the conservation of energy, we obtain
            \begin{align*}
                Q_\rho &= -\frac{2E}{\ell}-Mg\left( 1+\frac{2R}{ell} \right)\cos\phi&
                Q_z &= 0&
                Q_\phi &= Mg\left( 1-\frac{R}{\ell} \right)\sin\phi
            \end{align*}
            \item These are the final formulae for the forces on pivot as a function of $\phi$.
        \end{itemize}
    \end{itemize}
    \item \textbf{Equivalent simple pendulum}: The simple pendulum having the same equation of motion as our extended body.
    \item What happens in a similar system when we have a "sudden blow" or impulse?
    \begin{figure}[h!]
        \centering
        \begin{tikzpicture}
            \small
            \draw [stealth-stealth] (2,0) node[right]{$y$} -- (1.5,0) -- (1.5,-0.5) node[below]{$x$};
            \node at (1.7,-0.2) {$+$};
    
            \footnotesize
            \fill
                (0,0) circle (1.5pt)
                (0,-1.6) circle (1.5pt)
            ;
            \coordinate (c) at (0,-0.8);
            \draw [pux,thick,rotate around={-90:(c)}] (c) ellipse (1.5cm and 1cm);
            \draw [dashed] (0,0) -- node[right]{$d$} ($(0,0)!2!(c)$);
    
            \draw [->] (-2.3,-1.6) -- node[above]{$\vec{K}$} ++(0.9,0);
            \draw [->,shorten >=2pt] (0.7,0) -- node[above]{$\vec{S}$} (0,0);
        \end{tikzpicture}
        \caption{The "sweet spot" of a compound pendulum.}
        \label{fig:compoundPendulum2}
    \end{figure}
    \begin{itemize}
        \item Such pendulums have a sweet spot or equilibrium where the CM is just hanging down.
        \item We imagine that we kick the pendulum with impulse $\vec{K}$ in the $\hat{y}$ direction (using our modified coordinate system), as shown above.
        \item We have that $K\hat{y}=\vec{K}=\vec{F}\Delta t$.
        \item Let $\vec{S}=\vec{Q}\Delta t$.
        \item What we'll see is that there is a special value of $d$ (between the pivot and CM) for which $\vec{\rho}$ vanishes!
        \item During the short interval,
        \begin{equation*}
            I\ddot{\phi} = -MgR\sin\phi+Fd
        \end{equation*}
        \item We make the approximation that $\ddot{\phi}$ is constant during $\Delta t$ and that $\sin\phi=0$.
        \item It follows that
        \begin{equation*}
            \omega_\text{final} = \ddot{\phi}\Delta t
            = F\Delta t\frac{d}{I}
            = \frac{Kd}{I}
        \end{equation*}
        \item Additionally, we have that $\dot{\vec{P}}=\vec{Q}+\vec{F}$ so that
        \begin{equation*}
            P_\text{final} = \dot{P}\Delta t
            = -Q\Delta t+F\Delta t
            = -S+K
        \end{equation*}
        \item But we also know that
        \begin{equation*}
            P_\text{final} = M\dot{R}_\text{final}
            = M\omega_\text{final}R
            = \frac{MKdR}{I}
        \end{equation*}
        \item Thus, putting everything together, we obtain
        \begin{align*}
            \frac{MKdR}{I} &= -S+K\\
            S &= K\left( 1-\frac{MdR}{I} \right)
        \end{align*}
        \item Thus, $S$ vanishes if we choose $d=\ell=I/MR$.
        \item Takeaway: Regardless of the shape of our pendulum, if we hit it at the distance of the equivalent simple pendulum, we'll have no impulse on the pivot.
        \item This is the "sweet spot" of our baseball bat or whatever.
    \end{itemize}
\end{itemize}



\section{Office Hours (Jerison)}
\begin{itemize}
    \item \marginnote{11/6:}The final will slant toward the second half of the course, but everything is fair game.
    \item Is there an abstract environment in which we can view mass vs. angular mass and momentum vs. angular momentum, etc. as special cases of the same generalized construct?
    \begin{itemize}
        \item Yes.
        \item One answer.
        \begin{itemize}
            \item We can get this mapping from a speed-type thing to a momentum-type thing with linear operators.
            \item A tensor is a mathematical object with some kind of geometrical meaning independent of the coordinate basis.
        \end{itemize}
        \item Another answer.
        \begin{itemize}
            \item These are both examples of equations of motion that come from the Lagrangian (think \emph{generalized} mass, \emph{generalized} momentum, \emph{generalized} force, etc.).
        \end{itemize}
    \end{itemize}
    \item Could you post the KE of a free particle derivation?
    \item There will not be another \emph{in-class} review session, but she will hold one outside of class.
    \item We will get to Euler angles on Friday.
\end{itemize}



\section{Chapter 8: Many-Body Systems}
\emph{From \textcite{bib:KibbleBerkshire}.}
\begin{itemize}
    \item \marginnote{11/3:}Wrapping up Section 8.4.
\end{itemize}



\section{Chapter 9: Rigid Bodies}
\emph{From \textcite{bib:KibbleBerkshire}.}
\begin{itemize}
    \item Covered a smattering of results from various sections.
\end{itemize}




\end{document}