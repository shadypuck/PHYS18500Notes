\documentclass[../psets.tex]{subfiles}

\pagestyle{main}
\renewcommand{\leftmark}{Problem Set \thesection}
\setcounter{section}{7}

\begin{document}




\section{Final Exam Review}
\begin{enumerate}
    \item \marginnote{12/4:}\textcite{bib:KibbleBerkshire}, Q7.2. Where is the center of mass of the Sun-Jupiter system? (The mass ratio is $M_S/M_J=1047$. The semi-major axis of Jupiter's orbit is \SI{5.20}{\astronomicalunit}, where $\SI{1}{\astronomicalunit}=\SI{1.50e8}{\kilo\meter}$ is the semi-major axis of the Earth's orbit.) Through what angle does the Sun's position --- as seen from the Earth --- oscillate because of the gravitational attraction of Jupiter?
    \item \textcite{bib:KibbleBerkshire}, Q9.6.
    \begin{enumerate}
        \item A simple pendulum supported by a light rigid rod of length $l$ is released from rest with the rod horizontal. Find the reaction at the pivot as a function of the angle of inclination.
        \item For the cube of Problem 9.1 (a uniform solid cube of edge length $2a$ suspended from a horizontal axis along one edge), find the horizontal and vertical components of the reaction on the axis as a function of its angular position. Compare your answer with the corresponding expressions for the equivalent simple pendulum.
    \end{enumerate}
    \item \textcite{bib:KibbleBerkshire}, Q9.16. A uniformly charged sphere is spinning freely with angular velocity $\vec{\omega}$ in a uniform magnetic field $\vec{B}$. Taking the $z$-axis in the direction of $\vec{\omega}$, and $\vec{B}$ in the $xz$-plane, write down the moment about the center of the magnetic force on a particle at $\vec{r}$. Evaluate the total moment of the magnetic force on the sphere, and show that it is equal to $(q/2M)\vec{J}\times\vec{B}$, where $q$ and $M$ are the total charge and mass, respectively. Hence show that the axis will precess around the direction of the magnetic field with precessional angular velocity equal to the Larmor frequency
    \begin{equation*}
        \omega_\text{L} = \frac{qB}{2M}
    \end{equation*}
    What difference would it make if the charge distribution were spherically symmetric, but non-uniform?
    \item \textcite{bib:KibbleBerkshire}, Q12.5. Consider a simple pendulum of mass $m$ and length $l$, hanging in a trolley of mass $M$ running on smooth horizontal rails. The pendulum swings in a plane parallel to the rails. Use the position $x$ of the trolley and the angle of inclination $\theta$ of the pendulum as generalized coordinates. Find the Hamiltonian of this pendulum. Show that $x$ is ignorable. To what symmetry does this correspond?
    \item A bead of mass $m$ is on a circular hoop of radius $R$, oriented vertically (i.e., with its radius lined up with $\khat$). The hoop is rotating at constant rate $\omega$ about $\khat$.
    \begin{enumerate}
        \item Find the Hamiltonian for the system.
        \item Find Hamilton's equations of motion.
        \item Find and classify the fixed points of the system for all values of $\omega>0$. For what value of $\omega$ does a bifurcation occur?
        \item Draw a bifurcation diagram using the parameter $\gamma=R\omega^2/g$, i.e., draw a plot in the $\gamma$-$\theta$ plane where solid lines represent stable fixed points, and dashed lines represent unstable fixed points. (Hint: This is called a \textbf{pitchfork bifurcation}.) Sketch an example trajectory $\theta(t)$ if $\omega(t)$ is being slowly turned up via
        \begin{equation*}
            \omega(t) = \sqrt{\frac{gat}{R}}
        \end{equation*}
        where $a\ll\sqrt{g/l}$.
    \end{enumerate}
\end{enumerate}




\end{document}