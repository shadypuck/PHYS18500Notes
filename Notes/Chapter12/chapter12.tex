\documentclass[../notes.tex]{subfiles}

\pagestyle{main}
\renewcommand{\chaptermark}[1]{\markboth{\chaptername\ \thechapter\ (#1)}{}}
\setcounter{chapter}{11}

\begin{document}




\chapter{Hamiltonian Mechanics}
\section{Free Rotation; Hamilton's Equations}
\begin{itemize}
    \item \marginnote{11/13:}Hamilton's equations and the Hamiltonian.
    \begin{itemize}
        \item Like Lagrange's formulation is slightly different than Newton's, so too is Hamilton's.
        \item Hamilton's formulation is --- once again --- more general, and hence applicable for certain dissipative systems that can't be (easily??) treated with the other two methods.
        \item It is also ubiquitous throughout physics.
    \end{itemize}
    \item We mainly consider \textbf{natural} systems, and natural-conservative systems at that.
    \begin{itemize}
        \item Thus, we can write $L=L(q_1,\dots,q_N;\dot{q}_1,\dots,\dot{q}_N)=L(q,\dot{q})$.
    \end{itemize}
    \item \textbf{Natural} (system): The Lagrangian does not depend explicitly on time.
    \item \textbf{Forced} (system): The Lagrangian does depend explicitly on time.
    \item Recall that
    \begin{align*}
        \dot{p}_\alpha &= \pdv{L}{q_\alpha}&
        p_\alpha &= \pdv{L}{\dot{q}_\alpha}
    \end{align*}
    where the $\alpha=1,\dots,N$ index generalized coordinates such as Cartesian coordinates or even Euler angles.
    \item We can also let $\dot{q}_\alpha=\dot{q}_\alpha(q,p)$, i.e., let $\dot{q}_\alpha$ be a function of $q$ and $p$.
    \begin{itemize}
        \item For example, for a particle in plane polar coordinates, our Lagrangian is
        \begin{equation*}
            L = \frac{1}{2}m(\dot{r}^2+r^2\dot{\theta}^2)-V(r,\theta)
        \end{equation*}
        \item Thus,
        \begin{align*}
            p_r &= m\dot{r}&
                p_\theta &= mr^2\dot{\theta}\\
            \dot{r} &= \frac{p_r}{m}&
                \dot{\theta} &= \frac{p_\theta}{mr^2}
        \end{align*}
    \end{itemize}
    \item \textbf{Hamiltonian}: The operator defined as follows. \emph{Given by}
    \begin{equation*}
        H(q,p) = \sum_{\beta=1}^np_\beta\dot{q}_\beta(q,p)-L(q,\dot{q}(q,p))
    \end{equation*}
    \item Thus,
    \begin{equation*}
        \pdv{H}{p_\alpha} = \dot{q}_\alpha+\sum_{\beta=1}^np_\beta\pdv{\dot{q}_\beta}{p_\alpha}-\sum_{\beta=1}^n\underbrace{\pdv{L}{\dot{q}_\beta}}_{p_\beta}\pdv{\dot{q}_\beta}{p_\alpha}
        = \dot{q}_\alpha
    \end{equation*}
    \item Additionally,
    \begin{equation*}
        \pdv{H}{q_\alpha} = \underbrace{-\pdv{L}{q_\alpha}}_{-\dot{p}_\alpha}+\sum_{\beta=1}^np_\beta\pdv{\dot{q}_\beta}{q_\alpha}-\sum_{\beta=1}^n\underbrace{\pdv{L}{\dot{q}_\beta}}_{p_\beta}\pdv{\dot{q}_\beta}{q_\alpha}
        = -\dot{p}_\alpha
    \end{equation*}
    \item Therefore, we get Hamilton's equations of motion:
    \begin{align*}
        \pdv{H}{p_\alpha} &= \dot{q}_\alpha&
        \pdv{H}{q_\alpha} &= -\dot{p}_\alpha
    \end{align*}
\end{itemize}



\section{Conservation of Energy; Ignorable Coordinates}
\begin{itemize}
    \item \marginnote{11/15:}Recap.
    \begin{itemize}
        \item Hamiltonian as total energy.
        \item Ignorable coordinates.
        \item Examples.
    \end{itemize}
    \item Logistics.
    \begin{itemize}
        \item HW 6 due Friday.
        \item HW 7 due at last class.
        \begin{itemize}
            \item A little bit long (Hamiltonians + dynamical systems stuff from after break).
        \end{itemize}
        \item HW 8 (optional) due at exam.
        \begin{itemize}
            \item Will be posted during Thanksgiving week.
            \item A mixture of newer material and then some review questions from the second half of the quarter.
        \end{itemize}
        \item The final will focus on second-half stuff. However, it may use stuff from the beginning of the quarter. There will not be a specific rotating reference frames or scattering question, but we may have to use knowledge of Lagrangians, etc.
    \end{itemize}
    \item Last time.
    \begin{itemize}
        \item We constructed the Hamiltonian $H(q,p)$.
    \end{itemize}
    \item Note: A Hamiltonian is an example of something called a \textbf{Legendre transform}, though that's not important for this class.
    \item Example: Central conservative force in the plane.
    \begin{itemize}
        \item Recall that the relevant Lagrangian is
        \begin{equation*}
            L = \frac{1}{2}m\dot{r}^2+\frac{1}{2}mr^2\dot{\theta}^2-V(r)
        \end{equation*}
        \item The expression for the generalized momentum yields the following two relations.
        \begin{align*}
            p_r &= \pdv{L}{\dot{r}} = m\dot{r}&
                p_\theta &= \pdv{L}{\dot{\theta}} = mr^2\dot{\theta}\\
            \dot{r} &= \frac{p_r}{m}&
                \dot{\theta} &= \frac{p_\theta}{mr^2}
        \end{align*}
        \item Substituting the above into the definition of the Hamiltonian, we obtain
        \begin{equation*}
            H = (p_r\dot{r}+p_\theta\dot{\theta})-\left[ \frac{1}{2}m\dot{r}^2+\frac{1}{2}mr^2\dot{\theta}^2-V(r) \right]
            = \frac{p_r^2}{2m}+\frac{p_\theta^2}{2mr^2}+V(r)
        \end{equation*}
        \begin{itemize}
            \item Observe that this is the kinetic plus potential energy! This is a recurring theme.
        \end{itemize}
        \item Using Hamilton's equations, we obtain
        \begin{align*}
            \dot{r} &= \pdv{H}{p_r} = \frac{p_r}{m}\\
            \dot{\theta} &= \pdv{H}{p_\theta} = \frac{p_\theta}{mr^2}\\
            -\dot{p}_r &= \pdv{H}{r} = -\frac{p_\theta^2}{mr^3}+\dv{V}{r}\\
            -\dot{p}_\theta &= \pdv{H}{\theta} = 0
        \end{align*}
        \item The first two equations provide relations we already knew.
        \item The last equation implies that $J=p_\theta$ is constant, as we'd expect for a central conservative force!
        \item The third equation can be arranged into the following form, which (when integrated) yields the radial energy equation.
        \begin{equation*}
            \dot{p}_r = m\ddot{r} = \frac{J^2}{mr^3}-\dv{V}{r}
        \end{equation*}
    \end{itemize}
    \item The Hamiltonian as total energy.
    \begin{itemize}
        \item Let's see why this is the general case.
        \item We have that
        \begin{equation*}
            T = \frac{1}{2}\sum_{\alpha=1}^nm_\alpha\dot{\vec{r}}_\alpha{}^2
            = \frac{1}{2}\sum_{\alpha=1}^nm_\alpha(\dot{x}_\alpha^2+\dot{y}_\alpha^2+\dot{z}_\alpha^2)
        \end{equation*}
        \item Notice that
        \begin{equation*}
            \sum_{\alpha=1}^n\pdv{T}{\dot{q}_\alpha}\dot{q}_\alpha = 2T
        \end{equation*}
        \item Here, we're summing over all generalized coordinates.
        \item This is true for generalized coordinates for natural systems ($T$ is independent of $t$).
        \begin{itemize}
            \item A proof can be found on \textcite[232-33]{bib:KibbleBerkshire}.
        \end{itemize}
        \item It follows that
        \begin{equation*}
            H = \sum_{\beta=1}^np_\beta\dot{q}_\beta-L
            = \sum_{\beta=1}^n\pdv{T}{\dot{q}_\beta}\dot{q}_\beta-L
            = 2T-(T-V)
            = T+V
            = E
        \end{equation*}
    \end{itemize}
    \item In general, for $H(q,p,t)$, we have
    \begin{equation*}
        \dv{H}{t} = \pdv{H}{t}+\sum_{\alpha=1}^n\pdv{H}{q_\alpha}\dot{q}_\alpha+\sum_{\alpha=1}^n\pdv{H}{p_\alpha}\dot{p}_\alpha
        = \pdv{H}{t}+\sum_{\alpha=1}^n\left( \pdv{H}{q_\alpha}\pdv{H}{p_\alpha}-\pdv{H}{p_\alpha}\pdv{H}{q_\alpha} \right)
        = \pdv{H}{t}
    \end{equation*}
    \begin{itemize}
        \item The substitutions from the second to the third equality above follow from Hamilton's equations.
    \end{itemize}
    \item Special case of the above: Natural, conservative systems.
    \begin{itemize}
        \item $H(q,p,t)=H(q,p)$, so $\pdv*{H}{t}=0$.
        \item It follows that in such a system, $\dv*{H}{t}=0$, hence $H=T+V=E$ is constant.
    \end{itemize}
    \item \textbf{Ignorable coordinate}: A coordinate $q_\alpha$ that does not appear in $H$.
    \begin{itemize}
        \item Thus, for an ignorable coordinate,
        \begin{equation*}
            -\dot{p}_\alpha = \pdv{H}{q_\alpha} = 0
        \end{equation*}
        so $p_\alpha$ is constant.
        \item Generally, $p_\alpha$ is in $H$.
    \end{itemize}
    \item Example: Central force in plane? Recall the Hamiltonian from the first example above and note that $\theta$ is ignorable because $\dot{p}_\theta=0$.
    \begin{itemize}
        \item Thus, we recover the radial energy equation.
        \item Hamilton's equations for this system:
        \begin{align*}
            \dot{r} &= \frac{p_r}{m}&
            -\dot{p}_r &= \pdv{H}{r} = \dv{U}{r}
        \end{align*}
        where $U(r)$ is the effective potential energy.
        \item Thus, the $r$ coordinate behaves just like a single particle that sees the potential energy function $U(r)$.
        \item The remaining Hamilton's equations tell us that
        \begin{align*}
            \dot{p}_\theta &= 0&
            \dot{\theta} &= \frac{p_\theta}{mr^2}
        \end{align*}
    \end{itemize}
    \item Example: Symmetric top.
    \begin{itemize}
        \item $2/3$ of our Euler angles are ignorable, so we can write an effective potential energy function for the third.
        \item Our slightly complicated expression for the Lagrangian here is
        \begin{equation*}
            L = \underbrace{\frac{1}{2}I_1\dot{\theta}^2\sin^2\theta+\frac{1}{2}I_1\dot{\theta}^2+\frac{1}{2}I_3(\dot{\psi}+\dot{\phi}\cos\theta)^2}_T-MgR\cos\theta
        \end{equation*}
        \item Thus,
        \begin{gather*}
            p_\phi = \pdv{L}{\dot{\phi}} = I_1\dot{\phi}\sin^2\theta+I_3(\dot{\psi}+\dot{\phi}\cos\theta)\cos\theta\\
            p_\theta = I_1\dot{\theta}\\
            p_\psi = I_3(\dot{\psi}+\dot{\phi}\cos\theta)
        \end{gather*}
        \item It follows that
        \begin{align*}
            \dot{\phi} &= \frac{p_\phi-p_\psi\cos\theta}{I_1\sin^2\theta}\\
            \dot{\phi} &= \frac{p_\theta}{I_1}\\
            \dot{\psi} &= \frac{p_\psi}{I_3}-\frac{p_\phi-p_\psi\cos\theta}{I_1\sin^2\theta}\cos\theta
        \end{align*}
        \item Thus,
        \begin{equation*}
            H = T+V
        \end{equation*}
        where $T$ is given in the Lagrangian above.
        \item It follows that
        \begin{equation*}
            H = \frac{(p_\phi-p_\psi\cos\theta)^2}{2I_1\sin^2\theta}+\frac{p_\theta^2}{2I_1}+\frac{p_\psi^2}{2I_3}+MgR\cos\theta
        \end{equation*}
        \item Since $\phi,\psi$ don't appear, they're ignorable. Thus, $p_\phi,p_\psi$ are constants.
        \item Consequently, we can rewrite this Hamiltonian in the simpler form
        \begin{equation*}
            H = \frac{p_\theta^2}{2I_1}+U(\theta)
        \end{equation*}
        where
        \begin{equation*}
            U(\theta) = MgR\cos\theta+\frac{(p_\phi-p_\psi\cos\theta)^2}{2I_1\sin^2\theta}+\frac{p_\psi^2}{2I_3}
        \end{equation*}
        \item $U(\theta)$ is pretty complicated, but once we fix $p_\phi,p_\psi$, it can be thought of as an effective potential energy function in $\theta$.
        \item We can now evaluate Hamilton's equations.
        \begin{equation*}
            -\dot{p}_\theta = -I_1\ddot{\theta} = \pdv{H}{\theta} = \dv{U}{\theta}
        \end{equation*}
        \item Evaluating the derivative of $U(\theta)$ would be very nasty, but we can learn some thing without evaluating it.
        \item We get the conservation law
        \begin{equation*}
            \frac{p_\theta^2}{2I_1}+U(\theta) = E
        \end{equation*}
        \item Thus, fixing $U(\theta)$, we get a parabola in $p_\theta$ with minimum at $\theta_0$ and we get a wiggling motion between $\theta_\text{min}$ and $\theta_\text{max}$. At $U=E_\text{min}$, $\theta=\theta_0$ and we have \emph{steady precession}.
        \item The precession rate
        \begin{equation*}
            \dot{\phi} = \frac{p_\phi-p_\psi\cos\theta}{I_1\sin^2\theta}
        \end{equation*}
        \item Then $\dot{\theta}=0$, $\cos\theta=p_\phi/p_\psi$. If $\arccos(p_\phi/p_\psi)<\theta_\text{min}$ or $>\theta_\text{max}$.
        \item So the thing is rotating on its own, and alternating back and forth \emph{see picture}
        \item In the case $\theta_\text{min}<\arccos(p_\phi/p_\psi)<\theta_\text{max}$, we get loop de loops. Importantly, $\dot{\phi}$ changes sign.
        \item If $\arccos(p_\phi/p_\psi)=\theta_\text{min}$, we get cusps corresponding to $\dot{\phi}=0$.
    \end{itemize}
\end{itemize}



\section{Symmetries and Conservation Laws}
\begin{itemize}
    \item \marginnote{11/17:}Recap.
    \begin{itemize}
        \item Conservation laws as symmetries of the Hamiltonian.
    \end{itemize}
    \item Review.
    \begin{itemize}
        \item The Hamiltonian is given by $H=\sum_{\beta=1}^np_\beta\dot{q}_\beta-L(p,q)$. This is true in general.
        \begin{itemize}
            \item If we have a natural, conservative system, then $H=T+V=E$.
        \end{itemize}
        \item Once the Hamiltonian is constructed, we can get Hamilton's equations $-\dot{p}_\alpha=\pdv*{H}{q_\alpha}$ and $\dot{q}_\alpha=\pdv{H}{p_\alpha}$.
    \end{itemize}
    \item Today:
    \begin{itemize}
        \item Something formulated mathematically by Emmy NOether in 1918. We will come up with conservation laws based on symmetries of the Hamiltonian.
        \item We will see how functions can be thought of as operators, and when those operators don't change the Hamiltonian, there is a conserved quantity within the function.
        \item We'll see how different functions like $H(q,p)$, $J(q,p)$, etc. can be thought of as generators of transformations.
        \item As mentioned, if $H$ is unchanged by the transformation generated by a function $G$, then $G$ is a conserved quantity.
        \item But what is a \textbf{symmetry}?
    \end{itemize}
    \item \textbf{Symmetry}: Something that is unchanged by a particular operation.
    \item \textbf{Transformation} (generated by a function $G(q,p,t)$):
    \begin{align*}
        \delta q_\alpha &= \pdv{G}{p_\alpha}\delta\lambda&
        \delta p_\alpha &= -\pdv{G}{q_\alpha}\delta\lambda
    \end{align*}
    where $\delta\lambda$ is an infinitesimal (with correct units).
    \item Examples.
    \begin{enumerate}
        \item $G=p_1$.
        \begin{itemize}
            \item Induces $\delta q_1=\delta\lambda$ and $\delta p_1=0$.
        \end{itemize}
        \item $G=H$.
        \begin{itemize}
            \item $\delta q_\alpha=\dot{q}_\alpha\delta\lambda$, $\delta p_\alpha=\dot{p}_\alpha\delta\lambda$.
            \item Take $\delta\lambda=\delta t$.
            \item Thus, the Hamiltonian is the function that evolves the system forward in time.
            \item Essentially, applying the Hamiltonian to a system does the same thing as waiting for the system to evolve for a little bit.
            \item The Hamiltonian is the \textbf{time evolution operator}.
        \end{itemize}
        \item $G=J_z=xp_y-yp_x$.
        \begin{itemize}
            \item $\var{x}=-y\var{\lambda}$, $\var{p_x}=-p_y\var{\lambda}$, $\var{y}=x\var{\lambda}$, $\var{p_\lambda}=p_x\var{\lambda}$.
            \item Taking $\var{\lambda}=\var{\theta}$, $J$ generates infinitesimal rotation.
            \item Indeed, we are mapping $\vec{r}\mapsto\vec{r}+r\var{\theta}\,\hat{\theta}=\vec{r}-r\sin\theta\,\hat{x}\var{\theta}+r\cos\theta\,\hat{y}\var{\theta}$.
            \item Equivalently,
            \begin{align*}
                (x,y) &\mapsto (x-y\var{\theta},y+x\var{\theta})&
                (p_x,p_y) &\mapsto (p_x-p_y\var{\theta},p_y+p_x\var{\theta})
            \end{align*}
        \end{itemize}
    \end{enumerate}
    \item How much does another function $F$ change under the transformation induced by $G$?
    \begin{itemize}
        \item So we applied $G$, and our coordinates and momenta all changed a bit. $F$ depends on these coordinates and momenta, so how did it change?
        \item What we find out is that
        \begin{equation*}
            \var{F} = \sum_{\alpha=1}^n\left( \pdv{F}{q_\alpha}\var{q_\alpha}+\pdv{F}{p_\alpha}\var{p_\alpha} \right)
            = \sum_{\alpha=1}^n\left( \pdv{F}{q_\alpha}\pdv{G}{p_\alpha}-\pdv{F}{p_\alpha}\pdv{G}{q_\alpha} \right)\var{\lambda}
        \end{equation*}
    \end{itemize}
    \item We now define a \textbf{Poisson bracket} $[F,G]$ which encapsulates this change. Let
    \begin{equation*}
        [F,G] = \sum_{\alpha=1}^n\left( \pdv{F}{q_\alpha}\pdv{G}{p_\alpha}-\pdv{F}{p_\alpha}\pdv{G}{q_\alpha} \right)
    \end{equation*}
    \item Therefore, to answer our original question,
    \begin{equation*}
        \var{F} = [F,G]\var{\lambda}
    \end{equation*}
    is the transformation (change) in $F$, as generated by $G$.
    \item Example: Transformations generated by $H$ (the time translation) are
    \begin{equation*}
        \dv{F}{t} = \pdv{F}{t}+\sum_{\alpha=1}^n\left( \pdv{F}{q_\alpha}\dot{q}_\alpha+\pdv{F}{p_\alpha}\dot{p}_\alpha \right)
        = \pdv{F}{t}+\sum_{\alpha=1}^n\left( \pdv{F}{q_\alpha}\pdv{H}{p_\alpha}-\pdv{F}{p_\alpha}\pdv{H}{q_\alpha} \right)
        = \pdv{F}{t}+[F,H]
    \end{equation*}
    \item Example: Suppose that $F=F(q,p,t)$ is the total momentum of the system, the total angular momentum, the total energy (Poisson bracket of this is zer), etc.
    \item Important note.
    \begin{itemize}
        \item Poisson brackets are \textbf{antisymmetric}, i.e., $[G,F]=-[F,G]$.
        \item Thus, in particular, if $[G,F]=0$, then $[F,G]=0$.
        \item Takeaway: If $F$ is unchanged under the transformation generated by $G$, then $G$ is unchanged under the transformation generated by $F$.
    \end{itemize}
    \item Now, let's suppose that we have some function $G$ such that its corresponding transformation does not change $H$. Essentially, we applied $G$, our $q_\alpha,p_\alpha$'s changed, but $H$ did not.
    \begin{itemize}
        \item We can choose $G$ to be time-independent.
        \item In other words, $G$ does not change $H$, so $[H,G]=0$ in
        \begin{equation*}
            \var{H} = [H,G]\var{\lambda} = 0
        \end{equation*}
        \item Moreover,
        \begin{equation*}
            \dv{G}{t} = [G,H] = 0
        \end{equation*}
        \item Thus, $G$ is a conserved quantity.
        \item Takeaway: Any function that does not change the Hamiltonian is constant in time in the system.
    \end{itemize}
    \item Given this, we'll now spend the rest of class on Galilean transformations relativistically and see what this gives us in terms of conserved quantities.
    \item Review: Galilean transformations and the relativity principle.
    \begin{itemize}
        \item Given an isolated system of $N$ particles, we want to find a function $G$ that produces the transformation that corresponds to a particular relativity principle. Then that function will be a conserved quantity.
    \end{itemize}
    \item Relativity principles.
    \begin{enumerate}
        \item There is no preferred $t=0$.
        \begin{itemize}
            \item What is the function that corresponds to translation in time? We've discussed that it's $H$.
            \item Thus, we want to show that $H$ is invariant under translation in time.
            \item $H$, itself, actually generates time translations.
            \item We already know from its antisymmetry that
            \begin{equation*}
                [H,H] = 0
            \end{equation*}
            \item Thus, unless the Hamiltonian explicitly depends on time,
            \begin{equation*}
                \dv{H}{t} = [H,H] = 0
            \end{equation*}
            and hence energy is conserved.
        \end{itemize}
        \item There is no preferred origin of space.
        \begin{itemize}
            \item If we think that this is true, $H$ should be invariant under spatial translation.
            \item Which operator generates a spatial translation? Translations of the whole system are generated by the total linear momentum operator $P$.
            \item Thus, in other words (for a general translation in the $x$-direction), $G=P_x=\sum_{\alpha=1}^Np_{x\alpha}$.
            \item Thus, if we differentiate with respect to $P$, we get
            \begin{align*}
                \var{x_\alpha} &= \var{x}&
                \var{p_{x\alpha}} &= 0
            \end{align*}
            that is, all other components are zero.
            \item So, for $H$ to be invariant, we need
            \begin{equation*}
                [H,P_x]\var{x} = 0 = \sum_{\alpha=1}^N\pdv{H}{x_\alpha}\var{x}
            \end{equation*}
            \begin{itemize}
                \item This requirement is fulfilled if $H$ only depends on relative coordinates (i.e., depends only on combinations like $x_\alpha-x_\beta$) because our difference goes like $x_\alpha+\var{x}-(x_\beta+\var{x}) = x_\alpha-x_\beta$
                \item Note that this applies to any direction!
                \item Translational invariance means that we have a conserved linear momentum of the system.
                \item We need the Poisson bracket to be 0, which is equivalent to requiring that $\pdv*{\vec{P}}{\alpha}=0$, i.e., that the total linear momentum is conserved.
            \end{itemize}
        \end{itemize}
        \item Isotropy of space.
        \begin{itemize}
            \item $H$ is invariant under rotations.
            \item The generators of rotations are the following if, WLOG, we take our rotations to be about the $z$-axis:
            \begin{equation*}
                J_z = \sum_{i=1}^N(x_ip_{y_i}-y_ip_{x_i})
            \end{equation*}
            \item More generally, we can write any infinitesimal rotation as
            \begin{align*}
                \var{\vec{r}_\alpha} &= \hat{n}\times\vec{r}_\alpha\var{\phi}&
                \var{\vec{p}_\alpha} &= \hat{n}\times\vec{p}_\alpha\var{\phi}
            \end{align*}
            \begin{itemize}
                \item Note that $\vec{n}$ is the axis of rotation.
            \end{itemize}
            \item Generator: $\hat{n}\cdot\vec{J}$.
            \item Requires $H$ only be a function of scalar products of $\vec{r}_\alpha\cdot\vec{p}_\alpha$ (e.g., $\vec{r}_\alpha\cdot\vec{r}_\beta$, etc.).
            \item By the same logic,
            \begin{equation*}
                \dv{\vec{J}}{t} = 0
            \end{equation*}
            so the angular momentum is conserved.
        \end{itemize}
        \item Boosts in velocity; the dynamics are the same in any inertial reference frame.
        \begin{itemize}
            \item We should be able to change to a frame that's moving at a constant velocity with respect to our own and have all the laws of physics stay the same.
            \item Under a boost in velocity, the Hamiltonian \emph{will} change! If you go into a particle's rest frame, the KE will disappear. But Hamilton's equations, importantly, are not changing.
            \item We want the EOMs to be invariant under a boost (say in $x$), i.e., we want
            \begin{align*}
                \var{x_\alpha} &= t\var{v}&
                \var{p_\alpha} &= m_\alpha\var{v}
            \end{align*}
            \item Thus, the generator for this transformation is
            \begin{equation*}
                G_x = \sum_{\alpha=1}^N(p_{x\alpha}t-m_\alpha x_\alpha) = P_xt-MX
            \end{equation*}
            where $X$ is the $x$-coordinate of the CM.
            \item Thus, in general,
            \begin{equation*}
                \vec{G} = \vec{P}t-M\vec{R}
            \end{equation*}
            \begin{itemize}
                \item In general, $H$ will change and the EOMs won't.
            \end{itemize}
            \item It can be proven that
            \begin{equation*}
                \dv{\vec{G}}{t} = 0
            \end{equation*}
            \item This yields the following conservation law.
            \begin{equation*}
                \dv{t}(\vec{P}t-M\vec{R}) = 0
            \end{equation*}
            \item This equation tells us that the total momentum equals the total mass times the CM mass times velocity; essentially,
            \begin{equation*}
                \vec{P}-M\dv{\vec{R}}{t} = 0
            \end{equation*}
        \end{itemize}
    \end{enumerate}
\end{itemize}



\section{Introduction to Dynamical Systems; Phase Portraits}
\begin{itemize}
    \item \marginnote{11/27:}Announcements.
    \begin{itemize}
        \item Office hours today 4:00-5:30, GCIS E231 are the last of the quarter.
        \begin{itemize}
            \item Possible last OH on Saturday.
        \end{itemize}
        \item Email her for exam accommodations.
        \item This week: M/W (dynamical systems), F (review).
    \end{itemize}
    \item Outline.
    \begin{itemize}
        \item Review of Lagrangian and Hamiltonian stuffs.
        \item A note on L + H for forced systems.
        \item Dynamical systems.
        \begin{itemize}
            \item Phase portraits.
            \item Fixed points and linear stability analysis.
            \item Conservative systems with 1 DOF.
        \end{itemize}
    \end{itemize}
    \item Recap.
    \begin{itemize}
        \item Prior to break, we learned about the Hamiltonian, which can be written from the Lagrangian.
        \item For a natural system, the Hamiltonian can also be interpreted as the total energy $H=T+V=E$.
        \item The Hamiltonian is another way of getting EOMs (Hamilton's equations) from the system; they're a nice set of symmetrical, first-order ODEs.
        \item A nice aspect of this structure is that \emph{ignorable} coordinates, which do not appear in $H$, are ones you don't have to worry about because the fact that $p_\alpha$ is conserved with respect to this coordinate follows from the Hamilton equation $-p_\alpha=\pdv*{H}{q_\alpha}$.
        \item Another thing we saw is that for a function $G(q,p)$, $[G,H]=0$ implies that $G$ is conserved.
        \item From the relativity principles, we also get some pieces of information.
        \begin{enumerate}
            \item There are constraints on the form of the Hamiltonian (e.g., depending on relative positions of particles).
            \item There are particular quantities that we expect to be conserved if these relativity principles are to be true.
        \end{enumerate}
    \end{itemize}
    \item Before more Hamiltonian systems, let's do forced systems. See \textcite[231-42]{bib:KibbleBerkshire}.
    \item Fact: Constraints with time dependence can do work.
    \item Example: Suppose we have a pendulum that we're rotating about the vertical axis at constant angular speed $\omega$.
    \begin{itemize}
        \item The general form of the kinetic energy for such a system is
        \begin{equation*}
            T = \frac{1}{2}m(\dot{r}^2+r^2\dot{\theta}^2+(r\sin\theta)^2\dot{\phi}^2)
        \end{equation*}
        \item There are 2 constraints on the system:
        \begin{align*}
            r &= \ell&
            \dot{\phi} &= \omega
        \end{align*}
        \begin{itemize}
            \item An \textbf{algebraic} constraint, like the one above on the left, changes directions and does no work.
            \item The other kind of constraint, which does depend on time via its alternate (integrated) form $\phi=\omega t$, can do work.
        \end{itemize}
        \item We can still write $L=T-V$ and substitute for constraints, obtaining
        \begin{equation*}
            L = \frac{1}{2}m(\ell^2\dot{\theta}^2+(\ell\sin\theta)\omega^2)-mg\ell(1-\cos\theta)
        \end{equation*}
        \begin{itemize}
            \item Because of the dependence on $\theta$, the above is not a natural system.
            \item Essentially, $T$ is not just a function of $\dot{q}_\alpha$ and $\dot{q}_\beta$!
            \item Thus, $H\neq T+V$
        \end{itemize}
        \item Use, for this system,
        \begin{equation*}
            H = \sum_\alpha p_\alpha\dot{q}_\alpha-L
        \end{equation*}
        \item Note that the effective kinetic and potential energies (i.e., $T',V'$ such that $H=T'+V'$) of this system are
        \begin{align*}
            T' &= \frac{1}{2}m\ell^2\dot{\theta}^2&
            V' &= \frac{\ell^2}{2m}\omega^2\sin^2\theta+mg\ell(1-\cos\theta)
        \end{align*}
    \end{itemize}
\end{itemize}




\end{document}