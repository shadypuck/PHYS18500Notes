\documentclass[../psets.tex]{subfiles}

\pagestyle{main}
\renewcommand{\leftmark}{Problem Set \thesection}
\setcounter{section}{3}

\begin{document}




\section{Orbits, Scattering, and Rotating Reference Frames}
\begin{enumerate}
    \item \marginnote{10/27:}Here, we will consider orbits and scattering from an isotropic harmonic oscillator potential
    \begin{equation*}
        V(r) = \frac{1}{2}kr^2
    \end{equation*}
    where $k>0$, as well as the corresponding repulsive potential ($k<0$).
    \begin{enumerate}
        \item Use the radial energy equation to determine the effective potential energy function $U(r)$ for this potential in the two cases, $k>0$ and $k<0$. Sketch this function and describe whether the orbits are bounded in each case. For the attractive case, find the minimum $U_\text{min}$ of $U(r)$ and describe the motion for $E=U_\text{min}$.
        \item Let $\gamma=J^2/2m$, $\beta=\sqrt{E^2/4\gamma^2-k/2\gamma}$, and $\alpha=E/2\gamma$. Use the orbit equation to show that the orbits of the potential $V(r)=kr^2/2$ can be written as
        \begin{equation*}
            1 = r^2[(\beta+\alpha)\cos^2\theta+(\alpha-\beta)\sin^2\theta]
        \end{equation*}
        Hint: To solve the differential equation, substitute $v=u^2$. You will need to complete the square as in class.
        \item What are the shapes of the orbits for the cases $\alpha<\beta$ and $\alpha>\beta$? We saw that for the attractive inverse square law, the orbits could be either ellipses or hyperbolas. Is a hyperbola possible for the attractive harmonic oscillator potential? Discuss this result in light of part (A).
        \item For the attractive case, show that the condition for a real orbit recovers the value of $E=U_\text{min}$ that you derived in part (A).
    \end{enumerate}
    \item In class, we found formulas for the change in angle of particles scattered via a hard sphere potential or an inverse square potential. Here, we will derive a general expression for the scattering angle as a function of the impact parameter.
    \begin{enumerate}
        \item Show that for a general force, the change in angle of the trajectory as it traverses from its smallest to its largest radial distance is given by
        \begin{equation*}
            \Delta\theta = 2\int_{r_\text{min}}^{r_\text{max}}\frac{J/r^2}{\sqrt{2m(E-V(r)-J^2/2mr^2)}}\dd{r}
        \end{equation*}
        Hint: Use the orbit equation to find an expression for $\dv*{\theta}{r}$, and integrate.
        \item Let the speed of the particle far from the scattering center be $v$. Explain why the angular momentum is $J=mvb$, where $b$ is the impact parameter.
        \item Show that the total angular change for an unbounded particle in a central force field is
        \begin{equation*}
            \Delta\theta = 2\int_{r_\text{min}}^\infty\frac{b/r^2}{\sqrt{1-V(r)/E-b^2/r^2}}\dd{r}
        \end{equation*}
        The scattering angle $\Theta$ is related to this angular change via $\Theta=\pi-\Delta\theta$. Write down the expression for the scattering angle in terms of $b$. This expression can be integrated to find $b(\theta)$, and hence the differential scattering cross-section, for a general potential $V(r)$.
    \end{enumerate}
    \item \textcite{bib:KibbleBerkshire}, Q5.4. Find the velocity relative to an inertial frame (in which the center of the Earth is at rest) of a point on the Earth's equator. Additionally, an aircraft is flying above the equator at \SI[per-mode=symbol]{1000.}{\kilo\meter\per\hour}. Assuming that it flies straight and level (i.e., at a constant altitude above the surface), give its velocity relative to the inertial frame\dots
    \begin{enumerate}
        \item If it flies north;
        \item If it flies west;
        \item If it flies east.
    \end{enumerate}
    \item A British warship fires a projectile due south near the Falkland Islands during World War I at latitude \ang{50}S. The shells are fired at \ang{37} elevation with a speed of \SI[per-mode=symbol]{400}{\meter\per\second}. If the projectile was aimed on the assumption that the latitude was \ang{50}N (i.e., the sailors accounted for the Coriolis force in the northern hemisphere by mistake), by how much did it miss? (This actually happened, though the precise numbers are not accurate.)
    \item \textcite{bib:KibbleBerkshire}, Q5.18. Find the equation of motion for a particle in a \emph{uniformly accelerated} frame with acceleration $\vec{a}$. Show that for a particle moving in a uniform gravitatinoal field, and subject to other forces, the gravitational field may be eliminated by a suitable choice of $\vec{a}$.
\end{enumerate}




\end{document}