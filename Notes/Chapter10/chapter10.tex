\documentclass[../notes.tex]{subfiles}

\pagestyle{main}
\renewcommand{\chaptermark}[1]{\markboth{\chaptername\ \thechapter\ (#1)}{}}
\setcounter{chapter}{9}

\begin{document}




\chapter{Lagrangian Mechanics}
\section{Free Rotation; Hamilton's Equations}
\begin{itemize}
    \item \marginnote{11/13:}Case 2: Gravity as an external force.
    \begin{itemize}
        \item The system we'll consider herein is the spinning top of Figure \ref{fig:topGyroscope}.
        \item In this case, it's easier to write down a Lagrangian.
        \item Luckily, we already have the kinetic energy, so
        \begin{equation*}
            L = \frac{1}{2}I_1\dot{\phi}^2\sin^2\theta+\frac{1}{2}I_1\dot{\theta}^2+\frac{1}{2}I_3(\dot{\psi}+\dot{\phi}\cos\theta)^2-MgR\cos\theta
        \end{equation*}
        \item Thus, our Euler-Lagrange equations will be
        \begin{align*}
            \dv{t}(I_1\dot{\theta}) &= I_1\dot{\phi}^2\sin\theta\cos\theta-I_3(\dot{\psi}+\dot{\phi}\cos\theta)\dot{\phi}\sin\theta+MgR\sin\theta\tag{$\theta$}\\
            \dv{t}[\underbrace{I_1\dot{\phi}\sin^2\theta+I_3(\dot{\psi}+\dot{\phi}\cos\theta)\cos\theta}_{P_\phi}] &= 0\tag{$\phi$}\\
            \dv{t}[\underbrace{I_3(\dot{\psi}+\dot{\phi}\cos\theta)}_{P_\psi}] &= 0\tag{$\psi$}
        \end{align*}
        \begin{itemize}
            \item Note that $P_\phi,P_\psi$ are generalized momenta.
        \end{itemize}
        \item Recall that
        \begin{equation*}
            \omega_3 = \dot{\psi}+\dot{\phi}\cos\theta
        \end{equation*}
        \begin{itemize}
            \item Equation ($\psi$) implies that this quantity is constant.
        \end{itemize}
        \item What are the conditions for steady procession at fixed angle $\theta$?
        \begin{itemize}
            \item If $\theta$ is constant, then Equations ($\phi$) and ($\psi$) imply that $\dot{\psi},\dot{\phi}$ are constant.
            \item Let $\Omega:=\dot{\phi}$ be the precession rate.
            \item Then it follows by Equation ($\theta$) above that for $\dot{\theta}=0$, we must assume $\sin\theta\neq 0$ to get a nontrivial solution.
            \item Substituting the definition of $\Omega$ into Equation ($\theta$), we have
            \begin{align*}
                0 &= I_1\Omega^2\cos\theta-I_3\omega_3\Omega+MgR\\
                \Omega &= \frac{I_3\omega_3\pm\sqrt{I_3^2\omega_3^2-4I_1\cos\theta MgR}}{2I_1\cos\theta}
            \end{align*}
            \item Thus, for real $\Omega$, we need $I_3^2\omega_3^2-4I_1\cos\theta MgR>0$.
            \item Thus, there is a minimum rotation speed $\omega_3$ to get steady precession for a given $\theta$ given by
            \begin{equation*}
                I_3^2\omega_3^2 = 4I_1\cos\theta MgR
            \end{equation*}
            \item Takeaway: The smaller the angle of inclination, the faster you have to be spinning to get steady procession at that rate.
            \item Next time, we'll analyze some even more general cases using the Hamiltonian.
        \end{itemize}
    \end{itemize}
    \item Problems with translation and rotation.
    \begin{itemize}
        \item Recall from our discussion of many-body systems that
        \begin{equation*}
            T = \frac{1}{2}M(\dot{X}^2+\dot{Y}^2+\dot{Z}^2)+T^*
        \end{equation*}
        where $\vec{R}=(X,Y,Z)$ is the center of mass and $T^*$ is the kinetic energy in the CM frame.
        \item For any general system, $T^*$ is given by
        \begin{equation*}
            T^* = \sum_\alpha\frac{1}{2}m_\alpha(\dot{\vec{r}}_\alpha{}^*)^2
        \end{equation*}
        \item Additionally, for a rigid body,
        \begin{equation*}
            T^* = \frac{1}{2}I_1^*\omega_1^2+\frac{1}{2}I_2^*\omega_2^2+\frac{1}{2}I_3^*\omega_3^2
        \end{equation*}
        \begin{itemize}
            \item Note that $I_1^*$ is the moment of inertia about principal axis 1 with CM at the origin.
            \item Explicitly,
            \begin{equation*}
                I_1^* = \iiint\rho_m(\vec{r}{\,}^*)(z^2+y^2)
            \end{equation*}
        \end{itemize}
    \end{itemize}
    \item We now leap to Chapter 12 to talk about the Hamiltonian!
\end{itemize}




\end{document}