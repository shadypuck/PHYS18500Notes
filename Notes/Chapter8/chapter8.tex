\documentclass[../notes.tex]{subfiles}

\pagestyle{main}
\renewcommand{\chaptermark}[1]{\markboth{\chaptername\ \thechapter\ (#1)}{}}
\setcounter{chapter}{7}

\begin{document}




\chapter{Many-Body Systems}
\section{The Many-Body Problem}
\begin{itemize}
    \item \marginnote{11/1:}Announcements.
    \begin{itemize}
        \item Exam room locations are on Canvas.
        \item Notice that we skipped \textcite{bib:KibbleBerkshire}, Chapter 6.
    \end{itemize}
    \item Recap: 2-body systems.
    \begin{itemize}
        \item In such a system, we have two particles: $m_1,\vec{r}_1$ and $m_2,\vec{r}_2$. Their mass sum is $M=m_1+m_2$, their center of mass is at $\vec{R}=(m_1\vec{r}_1+m_2\vec{r}_2)/(m_1+m_2)$, their reduced mass is $\mu=m_1m_2/(m_1+m_2)$, and their relative position is $\vec{r}=\vec{r}_1-\vec{r}_2$.
        \item Under a constant external force, their EOMs uncouple into $M\ddot{R}_i=Mg_i$ and $\mu\ddot{r}_i=-\pdv*{V_\text{int}}{r_i}$ where $V_\text{int}(\vec{r})$ is the interaction potential energy.
        \item Jerison will now give a better answer to last time's question, "what is the reduced mass?"
        \begin{itemize}
            \item Let's look at two important cases to start.
            \begin{enumerate}
                \item If $m_1=m_2$, $\mu=m_1/2=m_2/2$ and the particles are maximally affecting each other.
                \item If $m_1\ll m_2$, then
                \begin{equation*}
                    \mu = \frac{m_1m_2}{m_2(1+m_1/m_2)}
                    \approx m_1\left( 1-\frac{m_1}{m_2} \right)+\text{H.O.T.}
                    \to m_1
                \end{equation*}
                where H.O.T. stands for "higher order terms."
            \end{enumerate}
            \item Additionally, as $m_1/m_2\to 0$, we have $M\to m_2$, $\vec{R}\to\vec{r}_2$, $\vec{r}_2{}^*\to 0$, $\mu\to m_1$, and $\vec{r}\to\vec{r}_1{}^*$.
            \begin{itemize}
                \item Essentially, we approach the limit of 1 body orbiting a fixed object.
                \item This justifies the approximation made in earlier chapters of the Earth orbiting a fixed sun or a satellite orbiting the fixed Earth or more.
                \item Additional consideration of $\vec{r}_2{}^*=-m_2/M\cdot\vec{r}$??
            \end{itemize}
        \end{itemize}
    \end{itemize}
    \item Today: Many-body systems.
    \begin{itemize}
        \item Lagrangian, CM frame.
        \item Rockets.
    \end{itemize}
    \item Call our particle indices $\alpha=1,\dots,N$.
    \begin{itemize}
        \item \textcite{bib:KibbleBerkshire} uses a different notation! They just say $\vec{r}_i$.
        \item The mass sum in this case is
        \begin{equation*}
            M = \sum_\alpha m_\alpha
        \end{equation*}
        \item The center of mass in this case is
        \begin{equation*}
            \vec{R} = \frac{1}{M}\sum_\alpha m_\alpha\vec{r}_\alpha
        \end{equation*}
        \item The linear momentum in this case is
        \begin{equation*}
            \vec{P} = \sum_\alpha m_\alpha\dot{\vec{r}}_\alpha
            = M\dot{\vec{R}}
        \end{equation*}
    \end{itemize}
    \item In the CM frame (still denoted $*$), we have
    \begin{equation*}
        \vec{r}_\alpha = \vec{R}+\vec{r}_\alpha{}^*
    \end{equation*}
    \begin{itemize}
        \item Moreover, within the frame, we still have $\dot{\vec{R}}{}^*=0$ and hence $\vec{P}{\,}^*=0$.
    \end{itemize}
    \item Using the above, we may define the kinetic energy for the system
    \begin{align*}
        T &= \frac{1}{2}\sum_\alpha m_\alpha\dot{\vec{r}}_\alpha{}^2\\
        &= \frac{1}{2}\sum_\alpha m_\alpha(\dot{\vec{R}}+\dot{\vec{r}}_\alpha{}^*)^2\\
        &= \frac{1}{2}\Bigg( \dot{\vec{R}}{\,}^2\sum_\alpha m_\alpha+2\dot{\vec{R}}\cdot\underbrace{\sum_\alpha m_\alpha\dot{\vec{r}}_\alpha{}^*}_{0=\vec{P}{\,}^*}+\sum_\alpha m_\alpha(\dot{\vec{r}}_\alpha{}^*)^2 \Bigg)\\
        &= \frac{1}{2}M\dot{\vec{R}}{\,}^2+\frac{1}{2}\sum_\alpha m_\alpha(\dot{\vec{r}}_\alpha{}^*)^2\\
        &= T_\text{CM}+T^*
    \end{align*}
    \item We may now define the Lagrangian for the system.
    \begin{itemize}
        \item Note that
        \begin{align*}
            V &= -\sum_\alpha m_\alpha\vec{r}_\alpha\cdot\vec{g}+V_\text{int}(\{\vec{r}_\alpha-\vec{r}_\beta\})\\
            &= -M\vec{g}\cdot\vec{R}+V_\text{int}(\{\vec{r}_\alpha-\vec{r}_\beta\})
        \end{align*}
        where $\{\vec{r}_\alpha-\vec{r}_\beta\}$ denotes the vector with all pairwise differences.
        \item Combining this result with the above, we obtain
        \begin{align*}
            L &= T-V\\
            &= \frac{1}{2}M\dot{\vec{R}}{\,}^2+M\vec{g}\cdot\vec{R}+\frac{1}{2}\sum_\alpha m_\alpha(\dot{\vec{r}}_\alpha{}^*)^2-V_\text{int}(\{\vec{r}_\alpha-\vec{r}_\beta\})
        \end{align*}
    \end{itemize}
    \item Thus, the EOMs separate into
    \begin{align*}
        M\ddot{\vec{R}} &= M\vec{g}&
        m_\alpha\ddot{r}_{\alpha_i}{}^* &= -\pdv{V_\text{int}}{r_{\alpha_i}{}^*}
    \end{align*}
    where we have three of these, one for each $i=q_1,q_2,q_3$ component of particle $\alpha$.
    \item Moreover, we get two conservation laws.
    \begin{align*}
        \frac{1}{2}M\dot{\vec{R}}{\,}^2-M\vec{g}\cdot\vec{R} &= E&
        T^*+V_\text{int} &= E_\text{int}
    \end{align*}
    \item In the more general case wherein other forces act on the system, we have
    \begin{equation*}
        m_\alpha\ddot{\vec{r}}_\alpha = \sum_\beta\vec{F}_{\alpha\beta}+\vec{F}_\alpha
    \end{equation*}
    \begin{itemize}
        \item The $\vec{F}_{\alpha\beta}$ are internal pairwise forces.
        \item The singular $\vec{F}_\alpha$ represents an external force.
    \end{itemize}
    \item Linear momentum in this case.
    \begin{align*}
        \dot{\vec{P}} &= \sum_\alpha m_\alpha\ddot{\vec{r}}_\alpha\\
        &= \sum_\alpha\sum_\beta\vec{F}_{\alpha\beta}+\sum_\alpha\vec{F}_\alpha
    \end{align*}
    \begin{itemize}
        \item Since $\vec{F}_{\alpha\beta}=-\vec{F}_{\beta\alpha}$, the left term above cancels, leaving us with
        \begin{equation*}
            \dot{\vec{P}} = \sum_\alpha\vec{F}_\alpha
            = M\ddot{\vec{R}}
        \end{equation*}
        \item Recall that if there are no external forces, $\vec{P}$ is constant.
    \end{itemize}
    \item Angular momentum in this case.
    \begin{equation*}
        \vec{J} = \sum_\alpha m_\alpha\vec{r}_\alpha\times\dot{\vec{r}}_\alpha
    \end{equation*}
    \begin{itemize}
        \item It follows that
        \begin{align*}
            \dot{\vec{J}} &= \sum_\alpha m_\alpha\vec{r}_\alpha\times\ddot{\vec{r}}_\alpha\\
            &= \sum_\alpha\vec{r}_\alpha\times\sum_\beta\vec{F}_{\alpha\beta}+\sum_\alpha\vec{r}_\alpha\times\vec{F}_\alpha\\
            &= \sum_\alpha\sum_\beta\vec{r}_\alpha\times\vec{F}_{\alpha\beta}+\sum_\alpha\vec{r}_\alpha\times\vec{F}_\alpha
        \end{align*}
        \item If $\vec{F}_{\alpha\beta}$ are central (i.e., parallel to $\vec{r}_\alpha-\vec{r}_\beta$), then the left term above is zero.
        \item This leaves us with
        \begin{equation*}
            \dot{\vec{J}} = \sum_\alpha\vec{r}_\alpha\times\vec{F}_\alpha
        \end{equation*}
        i.e., $\dot{\vec{J}}$ is only affected by external forces in the central $\vec{F}_{\alpha\beta}$ case.
        \item Thus, if $\vec{F}_\alpha=0$, $\vec{J}$ is constant.
        \item Additionally, if $\vec{F}_\alpha$ are central, then $\vec{J}$ is constant because the cross product cancels.
    \end{itemize}
    \item In the CM frame\dots
    \begin{itemize}
        \item Recall that $\vec{r}_\alpha=\vec{R}+\vec{r}_\alpha{}^*$.
        \item Thus,
        \begin{align*}
            \vec{J} &= \sum_\alpha m_\alpha(\vec{R}+\vec{r}_\alpha)\times(\dot{\vec{R}}+\dot{\vec{r}}_\alpha)\\
            &= \left( \sum_\alpha m_\alpha \right)\vec{R}\times\dot{\vec{R}}+\underbrace{\left( \sum_\alpha m_\alpha\vec{r}_\alpha{}^* \right)}_{0=\vec{R}{\,}^*}\times\dot{\vec{R}}+\vec{R}\times\underbrace{\left( \sum_\alpha m_\alpha\dot{\vec{r}}_\alpha{}^* \right)}_{0=\vec{P}{\,}^*}+\sum_\alpha m_\alpha\vec{r}_\alpha{}^*\times\dot{\vec{r}}_\alpha{}^*\\
            &= M\vec{R}\times\dot{\vec{R}}+\vec{J}{\,}^*
        \end{align*}
        where
        \begin{equation*}
            \vec{J}{\,}^* = \sum_\alpha m_\alpha\vec{r}_\alpha{}^*\times\dot{\vec{r}}_\alpha{}^*
        \end{equation*}
        \item It follows that
        \begin{align*}
            \dot{\vec{J}}{\,}^* &= \dot{\vec{J}}-\dv{t}(M\vec{R}\times\dot{\vec{R}})\\
            &= \dot{\vec{J}}-M\vec{R}\times\ddot{\vec{R}}\\
            &= \dot{\vec{J}}-\vec{R}\times\sum_\alpha\vec{F}_\alpha\\
            &= \sum_\alpha\vec{r}_\alpha\times\vec{F}_\alpha-\vec{R}\times\sum_\alpha\vec{F}_\alpha\\
            &= \sum_\alpha\vec{r}_\alpha{}^*\times\vec{F}_\alpha
        \end{align*}
    \end{itemize}
    \item An application of these multi-body systems: Rockets!
    \begin{itemize}
        \item Consider a rocket traveling forward at velocity $v$.
        \item To propel itself forward, it ejects mass $\dd{m}$ at a constant speed $u$ relative to the rocket.
        \item After the ejection, the mass $\dd{m}$ travels backwards at speed $v-u$ and the remaining rocket $M-\dd{m}$ travels forward at velocity $v+\dd{v}$.
        \item We have conservation of momentum in this "explosion," so
        \begin{align*}
            (M-\dd{m})(V+\dd{v})+\dd{m}(v-u) &= Mv\\
            Mv+M\dd{v}-v\dd{m}-u\dd{m}+v\dd{m} &= Mv\\
            M\dd{v} &= u\dd{m}\\
            &= -u\dd{M}\\
            \frac{\dd{v}}{u} &= -\frac{\dd{M}}{M}\\
            \frac{v}{u} &= -\ln\frac{M}{M_0}\\
            M &= M_0\e[-v/u]
        \end{align*}
    \end{itemize}
\end{itemize}




\end{document}