\documentclass[../psets.tex]{subfiles}

\pagestyle{main}
\renewcommand{\leftmark}{Problem Set \thesection}
\setcounter{section}{4}

\begin{document}




\section{Multiple-Body Systems}
\begin{enumerate}
    \item \marginnote{11/10:}\textcite{bib:KibbleBerkshire}, Q7.3. The \textbf{parallax} of a star (the angle subtended at the star by the radius of the Earth's orbit) is $\bar{\omega}$. The star's position is observed to oscillate with angular amplitude $\alpha$ and period $\tau$. If the oscillation is interpreted as being due to the existence of a planet moving in a circular orbit around the star, show that the planet's mass $m_1$ is given by
    \begin{equation*}
        \frac{m_1}{M_\text{S}} = \frac{\alpha}{\bar{\omega}}\left( \frac{M\tau_\text{E}}{M_\text{S}\tau} \right)^{2/3}
    \end{equation*}
    where $M$ is the total mass of the star plus planet, $M_\text{S}$ is the Sun's mass, and $\tau_\text{E}=\SI{1}{\year}$. Evaluate the mass $m_1$ if $M=0.25M_\text{S}$, $\tau=\SI{16}{\year}$, $\bar{\omega}=0.5''$, and $\alpha=0.01''$. What conclusion can be drawn without making the assumption that the orbit is circular?\par
    See \textcite[164]{bib:KibbleBerkshire} for a discussion of the angular variation.
    \begin{proof}
        The setup for this problem is as follows, very much not to scale.
        \begin{center}
            \begin{tikzpicture}[
                every node/.style={black}
            ]
                \footnotesize
                \coordinate (sun) at (0,0);
                \coordinate (earth) at (83:1);
                \coordinate (cm) at (5,0);
                \coordinate (star) at ($(cm)+(-103:0.6)$);
                \coordinate (planet) at ($(cm)+(77:1.8)$);
        
                \fill [orx]       (sun)   circle (3pt) node[below left]{$M_\text{S}$};
                \draw [semithick] (sun)   circle (1cm);
                \fill [blx]       (earth) circle (2pt);
        
                \fill             (cm)     circle (1.5pt);
                \draw [semithick] (cm)     circle (6mm);
                \draw [semithick] (cm)     circle (1.8cm);
                \fill [orx]       (star)   circle (3pt) node[below]{$M-m_1$};
                \fill [grx]       (planet) circle (2pt) node[above right]{$m_1$};
        
                \draw [help lines]
                    (earth) -- (cm) -- node[right]{$a_2$} (star) -- (sun) -- node[left]{$a_\text{E}$} cycle
                    (earth) -- node[below]{$r$} (star)
                ;
                \draw [white,line width=4pt,decorate,decoration={brace}] ($(planet)+(-20:0.7)$) -- ($(star)+(-20:0.7)$);
                \draw [decorate,decoration={brace}] ($(planet)+(-13:0.7)$) -- node[right=1pt,yshift=-1pt]{$a$} ($(star)+(-13:0.7)$);
        
                \pic [pic text={$\bar{\omega}$},angle eccentricity=2.8] {angle=earth--star--sun};
                \pic [pic text={$\alpha$},angle eccentricity=3.5] {angle=star--earth--cm};
                \pic [draw,help lines,angle radius=2mm] {right angle=earth--cm--star};
                \pic [draw,help lines,angle radius=2mm] {right angle=star--sun--earth};
            \end{tikzpicture}
        \end{center}
        The structure of the equation we are looking to derive strongly suggests that Kepler's third law will be involved in the derivation in some manner. Thus, let's start by writing out the two iterations that we can for this setup, one for the orbit of the Earth about the fixed sun and one for the orbit of the star and planet about their center of mass.
        \begin{align*}
            \left( \frac{\tau_\text{E}}{2\pi} \right)^2 &= \frac{a_\text{E}^3}{GM_\text{S}}&
            \left( \frac{\tau}{2\pi} \right)^2 &= \frac{a^3}{GM}
        \end{align*}
        To combine these equations and make them look a bit more like the desired result, let's divide the left one by the right one.
        \begin{align*}
            \frac{\left( \frac{\tau_\text{E}}{2\pi} \right)^2}{\left( \frac{\tau}{2\pi} \right)^2} &= \frac{\frac{a_\text{E}^3}{GM_\text{S}}}{\frac{a^3}{GM}}\\
            \left( \frac{\tau_\text{E}}{\tau} \right)^2 &= \frac{Ma_\text{E}^3}{M_Sa^3}
        \end{align*}
        Now we know that $a_\text{E}$ is not in the final answer, so let's try to substitute it out. From the geometry of the setup, we can see that
        \begin{equation*}
            \frac{a_\text{E}}{r} = \sin\bar{\omega}
            \approx \bar{\omega}
        \end{equation*}
        This substitution would help a bit: It would get $a_\text{E}$ out of there and it would get $\bar{\omega}$ in there. However, it also introduces $r$. Howe, in turn, could we get $r$ out? Well, we can get $r$ out (and $\alpha$ in!) via
        \begin{equation*}
            \frac{a_2}{r} = \sin\alpha
            \approx \alpha
        \end{equation*}
        Once again, we have made progress, but now we have to deal with $a_2$. Additionally, we still have to find some way to get $m_1$ into the math. Fortunately, we can do \emph{both} of these things at the same time with the substitution
        \begin{equation*}
            a_2 = \frac{m_1}{M}a
        \end{equation*}
        Combining all of the above results, we obtain the desired equality.
        \begin{align*}
            \left( \frac{\tau_\text{E}}{\tau} \right)^2 &= \frac{Ma_\text{E}^3}{M_Sa^3}\\
            &= \frac{M\bar{\omega}^3r^3}{M_Sa^3}\\
            &= \frac{M\bar{\omega}^3a_2^3}{M_Sa^3\alpha^3}\\
            &= \frac{M\bar{\omega}^3m_1^3a^3}{M_Sa^3\alpha^3M^3}\\
            &= \frac{\bar{\omega}^3m_1^3}{M_S\alpha^3M^2}\\
            \left( \frac{M\tau_\text{E}}{M_\text{S}\tau} \right)^2 &= \frac{\bar{\omega}^3m_1^3}{M_S^3\alpha^3}\\
            &= \left( \frac{\bar{\omega}m_1}{M_S\alpha} \right)^3\\
            \frac{\alpha}{\bar{\omega}}\left( \frac{M\tau_\text{E}}{M_\text{S}\tau} \right)^{2/3} &= \frac{m_1}{M_\text{S}}
        \end{align*}
        For the second part of the question, we can plug and chug as follows.
        \begin{align*}
            m_1 &= \frac{\alpha}{\bar{\omega}}\left( \frac{M\tau_\text{E}}{M_\text{S}\tau} \right)^{2/3}M_S\\
            &= \frac{0.01''}{0.5''}\left( \frac{(0.25M_\text{S})(\SI{1}{\year})}{(M_\text{S})(\SI{16}{\year})} \right)^{2/3}M_S\\
            &= \frac{0.01}{0.5}\left( \frac{(0.25)(1)}{(1)(16)} \right)^{2/3}M_S\\
            &= \frac{0.01}{0.5}\left( \frac{(0.25)(1)}{(1)(16)} \right)^{2/3}M_S\\
            \Aboxed{m_1 &= 0.00125\,M_\text{S}}
        \end{align*}
        For the third part of the question, if the orbit is not circular, then our observation of $\alpha$ may \emph{underestimate} $a$ depending on the orientation of the orbit with respect to Earth. But if $\alpha$ is underestimated, then $m_1$ may be underestimated, too. It follows that if the orbit is elliptical, then
        \begin{equation*}
            \boxed{m_1 \geq 0.00125\,M_\text{S}}
        \end{equation*}
    \end{proof}
    \item \textcite{bib:KibbleBerkshire}, Q7.4. Two particles of masses $m_1$ and $m_2$ are attached to the ends of a light spring. The natural length of the spring is $l$, and its tension is $k$ times its extension. Initially, the particles are at rest, with $m_1$ at a height $l$ above $m_2$. At $t=0$, $m_1$ is projected vertically upward with velocity $v$. Find the positions of the particles at any subsequent time (assuming that $v$ is not so large that the spring is expanded or compressed beyond its elastic limit).
    \begin{proof}
        % Suppose that in an $xy$-coordinate system with gravity acting in the $-\hat{y}$ direction, $m_2$ is initially at $(0,0)$ and $m_1$

        The motion of this system will be one-dimensional. Call this one dimension "$\hat{y}$" so that gravity acts in the $-\hat{y}$ direction. Note that after $m_1$ is projected vertically upward by the impulse $m_1v$, neither particle is "held in place" any more, and both particles move solely under the force of gravity and the spring potential. Essentially, this system is \emph{not} a case of $m_1$ moving upward with constant velocity $v$ for all times after $t=0$ and $m_2$ being carried along. Lastly, the condition that the tension in the spring is $k$ times its extension identifies the spring as one that obeys Hooke's law.\par
        With all of this in mind, we may now begin solving for the positions of the particles in the coordinate system defined above. To do so, we will make use of the diagonalization of $m_1\ddot{y}_1=-m_1g-k(y_1-y_2-\ell)$ and $m_2\ddot{y}_2=-m_2g+k(y_1-y_2-\ell)$ into
        \begin{align*}
            M\ddot{Y} &= -Mg&
            \mu\ddot{y} &= -k(y-\ell)
        \end{align*}
        where
        \begin{align*}
            M &= m_1+m_2&
            Y &= \frac{m_1y_1+m_2y_2}{m_1+m_2}&
            \mu &= \frac{m_1m_2}{m_1+m_2}
            y &= y_1-y_2
        \end{align*}
        The initial conditions are
        \begin{align*}
            Y(0) &= \frac{m_1\ell+m_2\cdot 0}{m_1+m_2} = \frac{m_1}{M}\ell&
                y(0) &= \ell\\
            \dot{Y}(0) &= \frac{m_1v+m_2\cdot 0}{m_1+m_2} = \frac{m_1}{M}v&
                \dot{y}(0) &= v
        \end{align*}
        The left EOM above can be solved easily, yielding
        \begin{equation*}
            Y = -\frac{1}{2}gt^2+\frac{m_1}{M}vt+\frac{m_1}{M}\ell
        \end{equation*}
        The right EOM above will take slightly more work to solve. In particular, we need to change coordinates via $\Delta y=y-\ell$ and its consequence $\ddot{\Delta y}=\ddot{y}$. Thus, we obtain
        \begin{align*}
            \mu\ddot{\Delta y} &= -k\Delta y\\
            \ddot{\Delta y} &= -\frac{k}{\mu}\Delta y
        \end{align*}
        This equation is analogous to the SHO, so the solution is
        \begin{equation*}
            \Delta y = C\cos(\omega t)+D\sin(\omega t)
        \end{equation*}
        where $\omega=\sqrt{k/\mu}$, $C=\Delta y(0)=y(0)-\ell=0$, and $D=\dot{\Delta y}(0)/\omega=\dot{y}(0)/\omega=v/\omega$. Applying these substitutions and returning to our original center of mass coordinates, we obtain the final solution for the right EOM above.
        \begin{equation*}
            y = \ell+\frac{v}{\omega}\sin(\omega t)
        \end{equation*}
        We now return from our diagonalized coordinates to our original (and desired) coordinates via
        \begin{align*}
            y_1 &= Y+\frac{m_2}{M}y\\
            &= -\frac{1}{2}gt^2+\frac{m_1}{M}vt+\frac{m_1}{M}\ell+\frac{m_2}{M}\left( \frac{v}{\omega}\sin(\omega t)+\ell \right)\\
            \Aboxed{y_1 &= \ell+\frac{m_1vt}{M}-\frac{1}{2}gt^2+\frac{m_2v}{M\omega}\sin(\omega t)}
        \end{align*}
        \begin{align*}
            y_2 &= Y-\frac{m_1}{M}y\\
            &= -\frac{1}{2}gt^2+\frac{m_1}{M}vt+\frac{m_1}{M}\ell-\frac{m_1}{M}\left( \frac{v}{\omega}\sin(\omega t)+\ell \right)\\
            \Aboxed{y_2 &= \frac{m_1vt}{M}-\frac{1}{2}gt^2-\frac{m_1v}{M\omega}\sin(\omega t)}
        \end{align*}
        where, once again, $\omega=\sqrt{k/\mu}$.
    \end{proof}
    \item \textcite{bib:KibbleBerkshire}, Q8.15. Show that in a conservative $N$-body system, a state of minimal total energy for a given total $z$-component of angular momentum is necessarily one in which the system is rotating as a rigid body about the $z$-axis. \emph{Hint}: Use the method of Lagrange multipliers (see \textcite{bib:KibbleBerkshire}, QA.10), and treat the components of the positions $\vec{r}_i$ and velocities $\dot{\vec{r}}_i$ as independent variables.\par
    For more details on the setting of this problem, refer to Problem 4.\par
    Note that this is not a variational problem (i.e., a functional minimization problem), as we have discussed previously in the course. Rather, we have a system where the positions and velocities of the particle are changing, and energy is being dissipated, until the system reaches an equilibrium state of minimal energy. We would like to minimize the total energy $E$ subject to the constraint on the angular momentum, which can be accomplished by minimizing $E(\vec{r}_\alpha,\dot{\vec{r}}_\alpha)-\omega(J_z(\vec{r}_\alpha,\dot{\vec{r}}_\alpha)-J_{z,0})$, where $\omega$ is a scalar Lagrange multiplier, $J_z$ is the $z$-component of angular momentum, and $J_{z,0}$ is the constant (conserved) value of $J_z$. We assume the system can explore all configurations, so that this function can be minimized with respect to each component of velocity and position for each particle independently. (If you need a referesher on using Lagrange multipliers in this type of optimization problem, Wikipedia has a \href{https://en.wikipedia.org/wiki/Lagrange_multiplier}{good article}.) As indicated in the description for Problem 4, the rigid body result comes from minimization with respect to the velocity components $\dot{r}_{\alpha i}$.\par
    \emph{Hint}: Recall that vectors that satisfy $\dd{\vec{b}}/\dd{t}=\vec{\omega}\times\vec{b}$ are fixed in magnitude and rotate about the axis specified by the direction of $\vec{\omega}$ at rate $\omega$.
    \begin{proof}
        % The kinetic energy for this system is
        % \begin{equation*}
        %     T = \sum\frac{1}{2}m_\alpha\dot{\vec{r}}_\alpha{}^2
        % \end{equation*}
        % Let the potential energy of the system be denoted by
        % \begin{equation*}
        %     V(\{\vec{r}_\alpha\})
        % \end{equation*}
        % Thus, the Lagrangian for the system is
        % \begin{equation*}
        %     L = T-V
        %     = \sum_\alpha\frac{1}{2}m_\alpha\dot{\vec{r}}_\alpha{}^2-V(\{\vec{r}_\alpha\})
        % \end{equation*}
        % A holonomic constraint is $f(\vec{r}_\alpha,\dot{\vec{r}}_\alpha,t)=J_z(\vec{r}_\alpha,\dot{\vec{r}}_\alpha)-J_{z,0}=\sum_\alpha m_\alpha(x_\alpha\dot{y}_\alpha-y_\alpha\dot{x}_\alpha)-J_{z,0}=0$.

        % Let $E$ be one scalar field over $(\vec{r}_\alpha,\dot{\vec{r}}_\alpha)$. We want to find the minimum of $E$ on the surface $J_z(\vec{r}_\alpha,\dot{\vec{r}}_\alpha)-J_{z,0}=0$. Thus, to this end, we minimize $E-\omega(J_z(\vec{r}_\alpha,\dot{\vec{r}}_\alpha)-J_{z,0})$.
        % This is the "method of Lagrange multipliers," not the "method of Lagrange undetermined multipliers!!!" Is it possible to use the method of Lagrange undetermined multipliers here, or not at all??
        

        We seek a specific state of the described system in its phase space. Specifically, we mathematically seek the point in the $6N$-dimensional phase space $(\vec{r}_\alpha,\dot{\vec{r}}_\alpha)$ at which the scalar field $E(\vec{r}_\alpha,\dot{\vec{r}}_\alpha)$ is minimized on the surface of the manifold described by $J_z(\vec{r}_\alpha,\dot{\vec{r}}_\alpha)-J_{z,0}=0$. Taking the hint, we can solve such a minimization problem using the method of Lagrange multipliers\footnote{Note that this is the "method of Lagrange multipliers," not the "method of Lagrange undetermined multipliers!" These are two very much different (albeit related) things. Is it possible to use the method of Lagrange undetermined multipliers here, or not at all?? The original wording of the guidance document called $\omega$ a "scalar Lagrange \emph{undetermined} multiplier," which leads me to believe that it is possible. See \href{https://askfilo.com/user-question-answers-physics/show-that-in-a-conservative-body-system-a-state-of-minimal-35393438393537}{here} for a possible solution using Lagrange undetermined multipliers.} as described in QA.10, i.e., by minimizing the function
        \begin{equation*}
            w = E(\vec{r}_\alpha,\dot{\vec{r}}_\alpha)-\omega(J_z(\vec{r}_\alpha,\dot{\vec{r}}_\alpha)-J_{z,0})
        \end{equation*}
        over the $6N+1$ independent variables $(\vec{r}_\alpha,\dot{\vec{r}}_\alpha,\omega)$.\par
        Before we take partial derivatives of $w$ with respect to each of these independent variables, let's express it explicitly in terms of these variables. To do so, we find that
        \begin{align*}
            E &= T+V\\
            &= \frac{1}{2}\sum_\alpha m_\alpha\dot{\vec{r}}_\alpha{}^2+V(\{\vec{r}_\alpha\})\\
            &= \frac{1}{2}\sum_\alpha m_\alpha(\dot{x}_\alpha^2+\dot{y}_\alpha^2+\dot{z}_\alpha^2)+V(\{x_\alpha,y_\alpha,z_\alpha\})
        \end{align*}
        We also find that
        \begin{align*}
            J_z &= \left[ \sum_\alpha m_\alpha(\vec{r}_\alpha\times\dot{\vec{r}}_\alpha) \right]\cdot\khat\\
            &= \sum_\alpha m_\alpha[(x_\alpha\ihat+y_\alpha\jhat+z_\alpha\khat)\times(\dot{x}_\alpha\ihat+\dot{y}_\alpha\jhat+\dot{z}_\alpha\khat)]\cdot\khat\\
            &= \sum_\alpha m_\alpha(x_\alpha\dot{y}_\alpha-y_\alpha\dot{x}_\alpha)
        \end{align*}
        Thus, the function to minimize is
        \begin{equation*}
            w = \frac{1}{2}\sum_\alpha m_\alpha(\dot{x}_\alpha^2+\dot{y}_\alpha^2+\dot{z}_\alpha^2)+V(\{x_\alpha,y_\alpha,z_\alpha\})-\omega\left[ \sum_\alpha m_\alpha(x_\alpha\dot{y}_\alpha-y_\alpha\dot{x}_\alpha)-J_{z,0} \right]
        \end{equation*}\par
        Taking the next hint, we find the rigid body result by minimizing with respect to the velocity components. The velocity-coordinate partial derivatives for each $\alpha=1,\dots,N$ are
        \begin{align*}
            \dv{w}{\dot{x}_\alpha} &= m_\alpha\dot{x}_\alpha+\omega m_\alpha y_\alpha&
            \dv{w}{\dot{y}_\alpha} &= m_\alpha\dot{y}_\alpha-\omega m_\alpha x_\alpha&
            \dv{w}{\dot{z}_\alpha} &= m_\alpha\dot{z}_\alpha
        \end{align*}
        Setting these equations equal to zero, we learn that
        \begin{align*}
            \dot{x}_\alpha &= -\omega y_\alpha&
            \dot{y}_\alpha &= \omega x_\alpha&
            \dot{z}_\alpha &= 0
        \end{align*}
        Equivalently, defining $\vec{\omega}$ in the usual fashion (i.e., via $\vec{\omega}=(0,0,\omega)$), the above equations can be expressed in the form
        \begin{equation*}
            \dv{\vec{r}_\alpha}{t} = \vec{\omega}\times\vec{r}_\alpha
        \end{equation*}
        Therefore, taking the final hint, the above equation shows that $\vec{r}_\alpha$ is fixed in magnitude and rotates about the axis specified by the direction of $\vec{\omega}$ at rate $\omega$ for all particles $\alpha$, as desired.
    \end{proof}
    \item \textcite{bib:KibbleBerkshire}, Q8.16. A planet of mass $M$ is surrounded by a cloud of small particles in orbits around it. Their mutual gravitational attraction is negligible. Due to collisions between the particles, the energy will gradually decrease from its initial value, but the angular momentum will remain fixed at, say, $\vec{J}=\vec{J}_0$. The system will thus evolve toward a state of minimum energy, subject to this constraint. Show that the particles will tend to form a ring around the planet. What happens to the energy lost? Why does the argument not necessarily apply to a cloud of particles around a hot star? \emph{Hint}: As in Problem 3, the constraint may be imposed by the method of Lagrange multipliers. In this case, because there are three components of the constraint equation, we need three Lagrange multipliers, say $\omega_x,\omega_y,\omega_z$. We have to minimize the function $E-\vec{\omega}\cdot(\vec{J}-\vec{J}_0)$ with respect to variations of the position $\vec{r}_i$ and velocities $\dot{\vec{r}}_i$, and with respect to $\vec{\omega}$. Show by minimizing with respect to $\dot{\vec{r}}_i$ that once equilibrium has been reached, the cloud rotates as a rigid body, and by minimizing with respect to $\vec{r}_i$ that all particles occupy the same orbit.\par
    You can assume that the planet's mass is very large, so the planet is effectively fixed. Use the hints in Problem 3 for the rigid body result. The derivatives with respect to position components should give a relationship that looks like a force law for each particle. This force law can be used in conjunction with the rigid body result to show that the orbits of the particles are all the same.
    \begin{proof}
        % Lastly, we will show that all $r_\alpha$ are equal. Let $\omega=|\vec{\omega}|$ and take the magnitude of both sides of the above equation to find an explicit expression for $r_\alpha$ as follows.
        
        % argue that $\vec{\omega},\vec{r}_\alpha,\dot{\vec{r}}_\alpha$ are mutually orthogonal for all particles $\alpha$. Since $\dot{\vec{r}}_\alpha=\dv*{\vec{r}_\alpha}{t}=\vec{\omega}\times\vec{r}_\alpha$ and since the cross product of two vectors is, by definition, orthogonal to both vectors, we know that
        % \begin{align*}
        %     \dot{\vec{r}}_\alpha &\perp \vec{\omega}&
        %     \dot{\vec{r}}_\alpha &\perp \vec{r}_\alpha
        % \end{align*}
        % Additionally, since $(-GM/r_\alpha^3)\vec{r}_\alpha=\vec{\omega}\times\dot{\vec{r}}_\alpha$, we know that
        % \begin{equation*}
        %     r_\alpha \perp \vec{\omega}
        % \end{equation*}
        % Thus, we have the desired orthogonality relation. Since $\vec{\omega}$ is fixed, the fact that each $\vec{r}_\alpha$ ($\alpha=1,\dots,N$) is perpendicular to $\vec{\omega}$ implies that all $\vec{r}_\alpha$ lie in the same plane.
        
        % This implies that we can redefine the $x,y,z$-axes of our coordinate system to lie along these vectors


        Taking the hint, we once again seek to minimize a function
        \begin{equation*}
            w = E-\vec{\omega}\cdot(\vec{J}-\vec{J}_0)
        \end{equation*}
        Thus, as in Problem 3, let's begin by expressing $w$ in terms of the $6N$ independent variables. Once again, $E=T+V$, but while the kinetic energy expression $T$ is the same as last time, the potential energy expression $V$ is different. Indeed, the hypothesis that the mutual gravitational attraction between each of these particles is negligible means, mathematically, that
        \begin{equation*}
            V_\text{int}(\{\vec{r}_\alpha-\vec{r}_\beta\}) = 0
        \end{equation*}
        However, this still leaves the question of the external potential energy $V_\text{ext}$. This expression is nontrivial because each particle exists within the gravitational field of the planet. If we define the origin of our coordinate system to be the center of mass of the planet, then
        \begin{align*}
            V_\text{ext}(\{\vec{r}_\alpha\}) &= \sum_\alpha -\frac{GMm_\alpha}{r_\alpha}\\
            V(\{x_\alpha,y_\alpha,z_\alpha\}) &= \sum_\alpha -\frac{GMm_\alpha}{\sqrt{x_\alpha^2+y_\alpha^2+z_\alpha^2}}
        \end{align*}
        Next, there is the question of the angular momentum. This also differs from last time, but as we are taking the whole thing instead of just one component, the method to calculate it is very similar:
        \begin{align*}
            \vec{J} &= \sum_\alpha m_\alpha(\vec{r}_\alpha\times\dot{\vec{r}}_\alpha)\\
            &= \sum_\alpha m_\alpha[(x_\alpha\ihat+y_\alpha\jhat+z_\alpha\khat)\times(\dot{x}_\alpha\ihat+\dot{y}_\alpha\jhat+\dot{z}_\alpha\khat)]\\
            &= \sum_\alpha m_\alpha[(y_\alpha\dot{z}_\alpha-z_\alpha\dot{y}_\alpha)\ihat+(z_\alpha\dot{x}_\alpha-x_\alpha\dot{z}_\alpha)\jhat+(x_\alpha\dot{y}_\alpha-y_\alpha\dot{x}_\alpha)\khat]\\
            &= \sum_\alpha m_\alpha
            \begin{bmatrix}
                y_\alpha\dot{z}_\alpha-z_\alpha\dot{y}_\alpha\\
                z_\alpha\dot{x}_\alpha-x_\alpha\dot{z}_\alpha\\
                x_\alpha\dot{y}_\alpha-y_\alpha\dot{x}_\alpha\\
            \end{bmatrix}
        \end{align*}
        Lastly, we have that
        \begin{align*}
            \vec{\omega} &=
            \begin{bmatrix}
                \omega_x\\
                \omega_y\\
                \omega_z\\
            \end{bmatrix}&
            \vec{J}_0 &=
            \begin{bmatrix}
                J_{x,0}\\
                J_{y,0}\\
                J_{z,0}\\
            \end{bmatrix}
        \end{align*}
        Thus, the function to minimize is
        \begin{equation*}
            w = \underbrace{\underbrace{\frac{1}{2}\sum_\alpha m_\alpha(\dot{x}_\alpha^2+\dot{y}_\alpha^2+\dot{z}_\alpha^2)\vphantom{
                \begin{bmatrix}
                    \omega_x\\
                    \omega_y\\
                    \omega_z\\
                \end{bmatrix}
            }}_T+\underbrace{\sum_\alpha -\frac{GMm_\alpha}{\sqrt{x_\alpha^2+y_\alpha^2+z_\alpha^2}}\vphantom{
                \begin{bmatrix}
                    \omega_x\\
                    \omega_y\\
                    \omega_z\\
                \end{bmatrix}
            }}_V}_E-\underbrace{
                \begin{bmatrix}
                    \omega_x\\
                    \omega_y\\
                    \omega_z\\
                \end{bmatrix}
            }_{\vec{\omega}}\cdot\Bigg(
                \underbrace{
                    \sum_\alpha m_\alpha
                    \begin{bmatrix}
                        y_\alpha\dot{z}_\alpha-z_\alpha\dot{y}_\alpha\\
                        z_\alpha\dot{x}_\alpha-x_\alpha\dot{z}_\alpha\\
                        x_\alpha\dot{y}_\alpha-y_\alpha\dot{x}_\alpha\\
                    \end{bmatrix}
                }_{\vec{J}}-\underbrace{
                    \begin{bmatrix}
                        J_{x,0}\\
                        J_{y,0}\\
                        J_{z,0}\\
                    \end{bmatrix}
                }_{\vec{J}_0}
            \Bigg)
        \end{equation*}
        By taking the dot product and algebraically rearranging, we obtain
        \begin{equation*}
            \begin{aligned}
                w ={}& \sum_\alpha\left[ \frac{1}{2}m_\alpha(\dot{x}_\alpha^2+\dot{y}_\alpha^2+\dot{z}_\alpha^2)-\frac{GMm_\alpha}{\sqrt{x_\alpha^2+y_\alpha^2+z_\alpha^2}} \right]\\
                & -\sum_\alpha\left[ \omega_xm_\alpha(y_\alpha\dot{z}_\alpha-z_\alpha\dot{y}_\alpha)+\omega_ym_\alpha(z_\alpha\dot{x}_\alpha-x_\alpha\dot{z}_\alpha)+\omega_zm_\alpha(x_\alpha\dot{y}_\alpha-y_\alpha\dot{x}_\alpha) \right]\\
                & +\omega_xJ_{x,0}+\omega_yJ_{y,0}+\omega_zJ_{z,0}
            \end{aligned}
        \end{equation*}
        Taking the next hint, we prove by minimizing $w$ with respect to variations of the velocities $\dot{\vec{r}}_\alpha$ that once equilibrium has been reached, the cloud rotates as a rigid body. The velocity-coordinate partial derivatives for each $\alpha=1,\dots,N$ are
        \begin{align*}
            \pdv{w}{\dot{x}_\alpha} &= m_\alpha\dot{x}_\alpha-\omega_ym_\alpha z_\alpha+\omega_zm_\alpha y_\alpha\\
            \pdv{w}{\dot{y}_\alpha} &= m_\alpha\dot{y}_\alpha-\omega_zm_\alpha x_\alpha+\omega_xm_\alpha z_\alpha\\
            \pdv{w}{\dot{z}_\alpha} &= m_\alpha\dot{z}_\alpha-\omega_xm_\alpha y_\alpha+\omega_ym_\alpha x_\alpha
        \end{align*}
        Setting these equations equal to zero, we learn that
        \begin{align*}
            \dot{x}_\alpha &= \omega_yz_\alpha-\omega_zy_\alpha\\
            \dot{y}_\alpha &= \omega_zx_\alpha-\omega_xz_\alpha\\
            \dot{z}_\alpha &= \omega_xy_\alpha-\omega_yx_\alpha
        \end{align*}
        Equivalently, with $\vec{\omega}$ defined as in the first definition of the function $w$, the above equations can be expressed in the form
        \begin{equation*}
            \dv{\vec{r}_\alpha}{t} = \vec{\omega}\times\vec{r}_\alpha
        \end{equation*}
        Therefore, taking the final hint from Problem 3, the above equation shows that $\vec{r}_\alpha$ is fixed in magnitude and rotates about the axis specified by the direction of $\vec{\omega}$ at rate $\omega$ for all particles $\alpha$. Hence, the cloud rotates as a rigid body.\par
        Taking the following hint, we prove by minimizing with respect to the variations of the positions $\vec{r}_\alpha$ that all particles occupy the same orbit. The position-coordinate partial derivatives for each $\alpha=1,\dots,N$ are
        \begin{align*}
            \pdv{w}{x_\alpha} &= \frac{GMm_\alpha x_\alpha}{(x_\alpha^2+y_\alpha^2+z_\alpha^2)^{3/2}}+\omega_ym_\alpha\dot{z}_\alpha-\omega_zm_\alpha\dot{y}_\alpha\\
            \pdv{w}{y_\alpha} &= \frac{GMm_\alpha y_\alpha}{(x_\alpha^2+y_\alpha^2+z_\alpha^2)^{3/2}}+\omega_zm_\alpha\dot{x}_\alpha-\omega_xm_\alpha\dot{z}_\alpha\\
            \pdv{w}{z_\alpha} &= \frac{GMm_\alpha z_\alpha}{(x_\alpha^2+y_\alpha^2+z_\alpha^2)^{3/2}}+\omega_xm_\alpha\dot{y}_\alpha-\omega_ym_\alpha\dot{x}_\alpha
        \end{align*}
        Setting these equations equal to zero, we learn that
        \begin{align*}
            -\frac{GMx_\alpha}{(x_\alpha^2+y_\alpha^2+z_\alpha^2)^{3/2}} &= \omega_y\dot{z}_\alpha-\omega_z\dot{y}_\alpha\\
            -\frac{GMy_\alpha}{(x_\alpha^2+y_\alpha^2+z_\alpha^2)^{3/2}} &= \omega_z\dot{x}_\alpha-\omega_x\dot{z}_\alpha\\
            -\frac{GMz_\alpha}{(x_\alpha^2+y_\alpha^2+z_\alpha^2)^{3/2}} &= \omega_x\dot{y}_\alpha-\omega_y\dot{x}_\alpha
        \end{align*}
        Equivalently, the above equations can be expressed in the form
        \begin{equation*}
            -\frac{GM}{r_\alpha^3}\vec{r}_\alpha = \vec{\omega}\times\dot{\vec{r}}_\alpha
        \end{equation*}
        This equation gives us everything we need to finish the problem. But before we can move any closer to proving that "all particles $\alpha$ occupy the same orbit," we need a more precise understanding of exactly what this condition means mathematically. First off, since we have proven that the particles rotate as a rigid body, each "orbit" will be circular. Since a circular orbit is specified by a plane in which the orbit lies and a radius, this condition further translates to all $\vec{r}_\alpha$ lying in the same plane and all $|\vec{r}_\alpha|=r_\alpha$ being equal.\par
        We will now show that all $\vec{r}_\alpha$ lie in the same plane. To do so, we will prove that they are all perpendicular to a single vector, namely $\vec{\omega}$. Let $\alpha$ be arbitrary. Then this relation follows immediately from the facts that $(-GM/r_\alpha^3)\vec{r}_\alpha=\vec{\omega}\times\dot{\vec{r}}_\alpha$ and that the cross product of two vectors ($\vec{r}_\alpha$) is, by definition, orthogonal to both vectors (including $\vec{\omega}$).\par
        We will show that all $r_\alpha$ are equal by taking the magnitude of both sides of the above vector equation and solving for an explicit expression for $r_\alpha$. However, before we can do this, we need the definition $\omega:=|\vec{\omega}|$. We also need to recall that $\dot{\vec{r}}_\alpha=\dv*{\vec{r}_\alpha}{t}=\vec{\omega}\times\vec{r}_\alpha$. Lastly, we need to establish the orthogonality not just between $\vec{\omega}$ and $\vec{r}_\alpha$, but between every vector in the set $\vec{\omega},\vec{r}_\alpha,\dot{\vec{r}}_\alpha$; fortunately, this can be done by adding to the previously determined fact that $\vec{\omega}\perp\vec{r}_\alpha$ the relation $\dot{\vec{r}}_\alpha=\vec{\omega}\times\vec{r}_\alpha$ and the aforementioned fact that the cross product of two vectors is orthogonal to both vectors. Therefore,
        \begingroup
        \allowdisplaybreaks
        \begin{align*}
            \left| -\frac{GM}{r_\alpha^3}\vec{r}_\alpha \right| &= \left| \vec{\omega}\times\dot{\vec{r}}_\alpha \right|\\
            \frac{GM}{r_\alpha^3}|\vec{r}_\alpha| &= |\vec{\omega}|\,|\dot{\vec{r}}_\alpha|\sin\ang{90}\\
            \frac{GM}{r_\alpha^3}r_\alpha &= |\vec{\omega}|\,|\vec{\omega}\times\vec{r}_\alpha|\,1\\
            \frac{GM}{r_\alpha^2} &= |\vec{\omega}|\,|\vec{\omega}|\,|\vec{r}_\alpha|\sin\ang{90}\\
            &= \omega^2r_\alpha\\
            \sqrt[3]{\frac{GM}{\omega^2}} &= r_\alpha
        \end{align*}
        \endgroup
    \end{proof}
    \item \textcite{bib:KibbleBerkshire}, Q9.11. A long, thin, hollow cylinder of radius $a$ is balanced on a horizontal knife edge, with its axis parallel to it. It is given a small displacement. Calculate the angular displacement at the moment when the cylinder ceases to touch the knife edge. \emph{Hint}: This is the moment when the radial component of the reaction falls to zero.
    \begin{proof}
        The setup for this problem is as follows.
        \begin{center}
            \begin{tikzpicture}[
                every node/.style={black}
            ]
                \footnotesize
                \draw [help lines] (30:1.5) coordinate (a) -- (-150:1) coordinate (b) -- ++(0,1.5) coordinate (c);
                \pic [draw,help lines,pic text={$\phi$}] {angle=a--b--c};
                \draw [help lines] (0,0) -- node[right]{$a$} (100:1);
        
                \draw [pux,thick] circle (1cm);
                \fill (-150:1) circle (2pt);
                \draw [thick] (-150:1) -- ++(0,-2);
        
                \fill circle (1.5pt);
                \draw [->] (0,0) -- (30:1.3) node[above]{$\hat{\rho}$};
                \draw [->] (0,0) -- (-60:1.3) node[below]{$\hat{\phi}$};
                \draw [->] (0,0) -- (-90:1.3) node[below left=-1pt]{$\vec{F}_g$};
            \end{tikzpicture}
        \end{center}
        Taking the hint, we look to derive an expression for the radial component $Q_\rho$ of the force of the pivot point on the rotating cylinder. $Q_\rho$ appears in the vector equation of motion
        \begin{equation*}
            M\ddot{\vec{R}} = \vec{Q}+\vec{F}_g
        \end{equation*}
        describing the system, so this will be our starting point. Expanding based on the picture for the setup, we obtain
        \begin{equation*}
            M(a\ddot{\phi}\,\hat{\phi}-Ma\dot{\phi}^2\,\hat{\rho}) = (Q_\phi\,\hat{\phi}+Q_\rho\,\hat{\rho})+(Mg\sin\phi\,\hat{\phi}-Mg\cos\phi\,\hat{\rho})
        \end{equation*}
        From here, we may isolate the radial component to yield
        \begin{equation*}
            Q_\rho = Mg\cos\phi-Ma\dot{\phi}^2
        \end{equation*}
        If we were to set $Q_\rho=0$ now, we could solve for $\phi$ in terms of the variables of the above equation. But we can do better! $\dot{\phi}^2$ also occurs in the expression for the kinetic energy of the rotating cylinder, so we now build up to a way to substitute this term out via said conservation law. Let's begin.\par
        By the parallel axis theorem, the moment of inertia of the rotating cylinder about the pivot point is
        \begin{equation*}
            I = Ma^2+I^*
            = Ma^2+Ma^2
            = 2Ma^2
        \end{equation*}
        Additionally, we know from class that its kinetic and potential energies are
        \begin{align*}
            T &= \frac{1}{2}I\dot{\phi}^2
                = Ma^2\dot{\phi}^2&
            V &= Mga\cos\phi
        \end{align*}
        Thus, we can calculate the total energy $E$ of the system by using the starting position of the cylinder at rest on top of the knife. In particular,
        \begin{equation*}
            E = V = Mga\cos(0) = Mga
        \end{equation*}
        It follows by the conservation of energy that
        \begin{align*}
            Ma^2\dot{\phi}^2+Mga\cos\phi &= Mga\\
            \dot{\phi}^2 &= \frac{Mga-Mga\cos\phi}{Ma^2}\\
            &= \frac{g}{a}(1-\cos\phi)
        \end{align*}
        Substituting this into the original expression for $Q_\rho$ and simplifying yields the desired result as follows.
        \begin{align*}
            Mg\cos\phi-Ma\cdot\frac{g}{a}(1-\cos\phi) &= 0\\
            % &= \cos\phi-(1-\cos\phi)\\
            % &= 2\cos\phi-1\\
            \cos\phi &= \frac{1}{2}\\
            \Aboxed{\phi &= \ang{60}}
        \end{align*}
    \end{proof}
\end{enumerate}




\end{document}